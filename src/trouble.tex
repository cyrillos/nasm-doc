%
% vim: ts=4 sw=4 et
%
\xchapter{trouble}{Troubleshooting}

This chapter describes some of the common problems that users have
been known to encounter with NASM, and answers them. If you think you
have found a bug in NASM, please see \nref{bugs}.

\xsection{problems}{Common Problems}

\xsubsection{inefficient}{NASM Generates \textindexlc{Inefficient Code}}

We sometimes get `bug' reports about NASM generating inefficient, or
even `wrong', code on instructions such as \code{ADD ESP,8}. This is a
deliberate design feature, connected to predictability of output:
NASM, on seeing \code{ADD ESP,8}, will generate the form of the
instruction which leaves room for a 32-bit offset. You need to code
\indexcode{BYTE}\code{ADD ESP,BYTE 8} if you want the space-efficient
form of the instruction. This isn't a bug, it's user error: if you
prefer to have NASM produce the more efficient code automatically enable
optimization with the \code{-O} option (see \nref{opt-O}).

\xsubsection{jmprange}{My Jumps are Out of Range}
\index{out of range!jumps}

Similarly, people complain that when they issue \textindex{conditional
jumps} (which are \code{SHORT} by default) that try to jump too far,
NASM reports `short jump out of range' instead of making the jumps
longer.

This, again, is partly a predictability issue, but in fact has a
more practical reason as well. NASM has no means of being told what
type of processor the code it is generating will be run on; so it
cannot decide for itself that it should generate \codeindex{Jcc NEAR}
type instructions, because it doesn't know that it's working for a 386 or
above. Alternatively, it could replace the out-of-range short
\code{JNE} instruction with a very short \code{JE} instruction that jumps
over a \code{JMP NEAR}; this is a sensible solution for processors
below a 386, but hardly efficient on processors which have good
branch prediction \emph{and} could have used \code{JNE NEAR} instead. So,
once again, it's up to the user, not the assembler, to decide what
instructions should be generated. See \nref{opt-O}.

\xsubsection{proborg}{\codeindex{ORG} Doesn't Work}

People writing \textindex{boot sector} programs in the \code{bin} format often
complain that \code{ORG} doesn't work the way they'd like: in order to
place the \code{0xAA55} signature word at the end of a 512-byte boot
sector, people who are used to MASM tend to code

\begin{lstlisting}
    ORG 0

    ; some boot sector code

    ORG 510
    DW 0xAA55
\end{lstlisting}

This is not the intended use of the \code{ORG} directive in NASM, and
will not work. The correct way to solve this problem in NASM is to
use the \codeindex{TIMES} directive, like this:

\begin{lstlisting}
    ORG 0

    ; some boot sector code

    TIMES 510-($-$$) DB 0
    DW 0xAA55
\end{lstlisting}

The \code{TIMES} directive will insert exactly enough zero bytes into
the output to move the assembly point up to 510. This method also
has the advantage that if you accidentally fill your boot sector too
full, NASM will catch the problem at assembly time and report it, so
you won't end up with a boot sector that you have to disassemble to
find out what's wrong with it.

\xsubsection{probtimes}{\codeindex{TIMES} Doesn't Work}

The other common problem with the above code is people who write the
\code{TIMES} line as

\begin{lstlisting}
    TIMES 510-$ DB 0
\end{lstlisting}

by reasoning that \code{\$} should be a pure number, just like 510, so
the difference between them is also a pure number and can happily be
fed to \code{TIMES}.

NASM is a \emph{modular} assembler: the various component parts are
designed to be easily separable for re-use, so they don't exchange
information unnecessarily. In consequence, the \code{bin} output
format, even though it has been told by the \code{ORG} directive that
the \code{.text} section should start at 0, does not pass that
information back to the expression evaluator. So from the
evaluator's point of view, \code{\$} isn't a pure number: it's an offset
from a section base. Therefore the difference between \code{\$} and 510
is also not a pure number, but involves a section base. Values
involving section bases cannot be passed as arguments to \code{TIMES}.

The solution, as in the previous section, is to code the \code{TIMES}
line in the form

\begin{lstlisting}
    TIMES 510-($-$$) DB 0
\end{lstlisting}

in which \code{\$} and \code{\$\$} are offsets from the same section base,
and so their difference is a pure number. This will solve the
problem and generate sensible code.
