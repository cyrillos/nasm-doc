\chapter{\textindexlc{Standard Macro Packages}}
\label{ch:macropkg}

The \codeindex{\%use} directive (see \fullref{sec:use}) includes one of
the standard macro packages included with the NASM distribution and compiled
into the NASM binary. It operates like the \code{\%include} directive (see
\fullref{sec:include}), but the included contents is provided by NASM itself.

The names of standard macro packages are case insensitive, and can be
quoted or not.

\section{\codeindex{altreg}: \textindexlc{Alternate Register Names}}
\label{sec:pkgaltreg}

The \code{altreg} standard macro package provides alternate register
names. It provides numeric register names for all registers (not just
\code{R8}-\code{R15}), the Intel-defined aliases \code{R8L}-\code{R15L}
for the low bytes of register (as opposed to the NASM/AMD standard names
\code{R8B}-\code{R15B}), and the names \code{R0H}-\code{R3H} (by analogy
with \code{R0L}-\code{R3L}) for \code{AH}, \code{CH}, \code{DH},
and \code{BH}.

Example use:

\begin{lstlisting}
%use altreg

proc:
    mov r0l,r3h     ; mov al,bh
    ret
\end{lstlisting}

See also \fullref{sec:reg64}.

\section{\codeindex{smartalign}\index{align, smart}: Smart \code{ALIGN} Macro}
\label{sec:pkgsmartalign}

The \code{smartalign} standard macro package provides for an
\codeindex{ALIGN} macro which is more powerful than the default (and
backwards-compatible) one (see \fullref{sec:align}). When the
\code{smartalign} package is enabled, when \code{ALIGN} is used without
a second argument, NASM will generate a sequence of instructions more
efficient than a series of \code{NOP}. Furthermore, if the padding
exceeds a specific threshold, then NASM will generate a jump over
the entire padding sequence.

The specific instructions generated can be controlled with the
new \codeindex{ALIGNMODE} macro. This macro takes two parameters: one mode,
and an optional jump threshold override. If (for any reason) you need
to turn off the jump completely just set jump threshold value to -1
(or set it to \code{nojmp}). The following modes are possible:

\begin{itemize}
    \item{\code{generic}: Works on all x86 CPUs and should have
    reasonable performance. The default jump threshold is 8.
    This is the default.}

    \item{\code{nop}: Pad out with \code{NOP} instructions. The only
    difference compared to the standard \code{ALIGN} macro is that NASM
    can still jump over a large padding area. The default jump
    threshold is 16.}

    \item{\code{k7}: Optimize for the AMD K7 (Athlon/Althon XP).
    These instructions should still work on all x86 CPUs. The default
    jump threshold is 16.}

    \item{\code{k8}: Optimize for the AMD K8 (Opteron/Althon 64).
    These instructions should still work on all x86 CPUs. The default
    jump threshold is 16.}

    \item{\code{p6}: Optimize for Intel CPUs. This uses the long
    \code{NOP} instructions first introduced in Pentium Pro. This
    is incompatible with all CPUs of family 5 or lower, as well as
    some VIA CPUs and several virtualization solutions. The default
    jump threshold is 16.}
\end{itemize}

The macro \codeindex{\_\_ALIGNMODE\_\_} is defined to contain the
current alignment mode. A number of other macros beginning with
\code{\_\_ALIGN\_} are used internally by this macro package.
