%
% vim: ts=4 sw=4 et
%
\xchapter{changelog}{\textindexlc{NASM Version History}}

\xsection{cl-2.xx}{NASM 2 Series}

The NASM 2 series supports x86-64, and is the production version of NASM
since 2007.

\xsubsection{cl-2.14.02}{Version 2.14.02}

\begin{itemize}
    \item{Fix crash due to multiple errors or warnings during the code
        generation pass if a list file is specified.}
\end{itemize}

\xsubsection{cl-2.14.01}{Version 2.14.01}

\begin{itemize}
    \item{Create all system-defined macros defore processing command-line
        given preprocessing directives (\code{-p}, \code{-d}, \code{-u}, \code{--pragma},
        \code{--before}).}

    \item{If debugging is enabled, define a \code{\_\_DEBUG\_FORMAT\_\_} predefined
        macro. See \nref{dfmtm}.}

    \item{Fix an assert for the case in the \code{obj} format when a \code{SEG}
        operator refers to an \code{EXTERN} symbol declared further down in the
        code.}

    \item{Fix a corner case in the floating-point code where a binary, octal
        or hexadecimal floating-point having at least 32, 11, or 8 mantissa
        digits could produce slightly incorrect results under very specific
        conditions.}

    \item{Support \code{-MD} without a filename, for \code{gcc} compatibility.
        \code{-MF} can be used to set the dependencies output filename.
        See \nref{opt-MD}.}

    \item{Fix \code{-E} in combination with \code{-MD}. See \nref{opt-E}.}

    \item{Fix missing errors on redefined labels; would cause convergence
        failure instead which is very slow and not easy to debug.}

    \item{Duplicate definitions of the same label \emph{with the same value} is now
        explicitly permitted (2.14 would allow it in some circumstances.)}

    \item{Add the option \code{--no-line} to ignore \code{\%line} directives in the
        source. See \nref{opt-no-line} and \nref{line}.}
\end{itemize}

\xsubsection{cl-2.14}{Version 2.14}

\begin{itemize}
    \item{Changed \code{-I} option semantics by adding a trailing path
        separator unconditionally.}

    \item{Fixed null dereference in corrupted invalid single line macros.}

    \item{Fixed division by zero which may happen if source code is malformed.}

    \item{Fixed out of bound access in processing of malformed segment override.}

    \item{Fixed out of bound access in certain \code{EQU} parsing.}

    \item{Fixed buffer underflow in float parsing.}

    \item{Added \code{SGX} (Intel Software Guard Extensions) instructions.}

    \item{Added \code{+n} syntax for multiple contiguous registers.}

    \item{Fixed \code{subsections\_via\_symbols} for \code{macho} object format.}

    \item{Added the \code{--gprefix}, \code{--gpostfix}, \code{--lprefix}, and
        \code{--lpostfix} command line options, to allow command line base symbol
        renaming. See \nref{opt-pfix}.}

    \item{Allow label renaming to be specified by \code{\%pragma} in addition to
        from the command line. See \nref{mangling}.}

    \item{Supported generic \code{\%pragma} namespaces, \code{output} and \code{debug}.
        See \nref{gen-namespace}.}

    \item{Added the \code{--pragma} command line option to inject a \code{\%pragma}
        directive. See \nref{opt-pragma}.}

    \item{Added the \code{--before} command line option to accept preprocess
        statement before input. See \nref{opt-before}.}

    \item{Added \code{AVX512} \code{VBMI2} (Additional Bit Manipulation), \code{VNNI}
        (Vector Neural Network), \code{BITALG} (Bit Algorithm), and \code{GFNI} (Galois
        Field New Instruction) instructions.}

    \item{Added the \code{STATIC} directive for local symbols that should be
        renamed using global-symbol rules. See \nref{static}.}

    \item{Allow a symbol to be defined as \code{EXTERN} and then later
        overridden as \code{GLOBAL} or \code{COMMON}. Furthermore, a symbol
        declared \code{EXTERN} and then defined will be treated as \code{GLOBAL}.
        See \nref{extern}.}

    \item{The \code{GLOBAL} directive no longer is required to precede the
        definition of the symbol.}

    \item{Support \code{private\_extern} as \code{macho} specific extension to the
        \code{GLOBAL} directive. See \nref{macho-pext}.}

    \item{Updated \code{UD0} encoding to match with the specification}

    \item{Added the \code{--limit-X} command line option to set execution
        limits. See \nref{opt-limit}.}

    \item{Updated the \code{Codeview} version number to be aligned with \code{MASM}.}

    \item{Added the \code{--keep-all} command line option to preserve output
        files. See \nref{opt-keep-all}.}

    \item{Added the \code{--include} command line option, an alias to \code{-P}
        (\nref{opt-p}).}

    \item{Added the \code{--help} command line option as an alias to \code{-h}
        (\nref{syntax}).}

    \item{Added \code{-W}, \code{-D}, and \code{-Q} suffix aliases for \code{RET}
        instructions so the operand sizes of these instructions can be
        encoded without using \code{o16}, \code{o32} or \code{o64}.}
\end{itemize}

%\xsubsection{cl-2.13.03} Version 2.13.03
%
%\item{Added AVX and AVX512 \code{VAES*} and \code{VPCLMULQDQ} instructions.
%
%\item{Fixed missing dwarf record in x32 ELF output format.
%
%\xsubsection{cl-2.13.02} Version 2.13.02
%
%\item{Fix false positive in testing of numeric overflows.
%
%\item{Fix generation of \code{PEXTRW} instruction.
%
%\item{Fix \code{smartalign} package which could trigger an error during
%   optimization if the alignment code expanded too much due to
%   optimization of the previous code.
%
%\item{Fix a case where negative value in \code{TIMES} directive causes
%   panic instead of an error.
%
%\item{Always finalize \code{.debug_abbrev} section with a null in
%   \code{dwarf} output format.
%
%\item{Support \code{debug} flag in section attributes for \code{macho}
%   output format.  See \nref{machosect}.
%
%\item{Support up to 16 characters in section names for \code{macho}
%   output format.
%
%\item{Fix missing update of global \code{BITS} setting if \code{SECTION}
%   directive specified a bit size using output format-specific
%   extensions (e.g. \code{USE32} for the \code{obj} output format.)
%
%\item{Fix the incorrect generation of VEX-encoded instruction when static
%   mode decorators are specified on scalar instructions, losing the
%   decorators as they require EVEX encoding.
%
%\item{Option \code{-MW} to quote dependency outputs according to Watcom
%   Make conventions instead of POSIX Make conventions.  See \nref{opt-MW}.
%
%\item{The \code{obj} output format now contains embedded dependency file
%   information, unless disabled with \code{%pragma obj nodepend}.  See
%   \nref{objdepend}.
%
%\item{Fix generation of dependency lists.
%
%\item{Fix a number of null pointer reference and memory allocation errors.
%
%\item{Always generate symbol-relative relocations for the \code{macho64}
%   output format; at least some versions of the XCode/LLVM linker fails
%   for section-relative relocations.
%
%\xsubsection{cl-2.13.01} Version 2.13.01
%
%\item{Fix incorrect output for some types of \code{FAR} or \code{SEG}
%   references in the \code{obj} output format, and possibly other 16-bit
%   output formats.
%
%\item{Fix the address in the list file for an instruction containing a
%   \code{TIMES} directive.
%
%\item{Fix error with \code{TIMES} used together with an instruction which
%   can vary in size, e.g. \code{JMP}.
%
%\item{Fix breakage on some uses of the \code{DZ} pseudo-op.
%
%\xsubsection{cl-2.13} Version 2.13
%
%\item{Support the official forms of the \code{UD0} and \code{UD1} instructions.
%
%\item{Allow self-segment-relative expressions in immediates and
%   displacements, even when combined with an external or otherwise
%   out-of-segment special symbol, e.g.:
%
%\c      extern foo
%\c      mov eax,[foo - $ + ebx]               ; Now legal
%
%\item{Handle a 64-bit origin in NDISASM.
%
%\item{NASM can now generate sparse output files for relevant output
%   formats, if the underlying operating system supports them.
%
%\item{The \code{macho} object format now supports the \code{subsections_via_symbols}
%   and \code{no_dead_strip} directives, see \nref{macho-ssvs}.
%
%\item{The \code{macho} object format now supports the \code{no_dead_strip},
%  \code{live_support} and \code{strip_static_syms} section flags, see
%  \nref{machosect}.
%
%\item{The \code{macho} object format now supports the \code{dwarf} debugging
%  format, as required by newer toolchains.
%
%\item{All warnings can now be suppressed if desired; warnings not
%   otherwise part of any warning class are now considered its own
%   warning class called \code{other} (e.g. \code{-w-other}).  Furthermore,
%   warning-as-error can now be controlled on a per warning class
%   basis, using the syntax \code{-w+error=}\e{warning-class} and its
%   equivalent for all other warning control options.  See \nref{opt-w}
%   for the command-line options and warning classes and
%   \nref{asmdir-warning} for the \code{[WARNING]} directive.
%
%\item{Fix a number of bugs related to AVX-512 decorators.
%
%\item{Significant improvements to building NASM with Microsoft Visual
%   Studio via \code{Mkfiles/msvc.mak}.  It is now possible to build the
%   full Windows installer binary as long as the necessary
%   prerequisites are installed; see \code{Mkfiles/README}
%
%\item{To build NASM with custom modifications (table changes) or from the
%   git tree now requires Perl 5.8 at the very minimum, quite possibly
%   a higher version (Perl 5.24.1 tested.)  There is no requirement to
%   have Perl on your system at all if all you want to do is build
%   unmodified NASM from source archives.
%
%\item{Fix the \code{\{z\}} decorator on AVX-512 \code{VMOVDQ*} instructions.
%
%\item{Add new warnings for certain dangerous constructs which never ought
%   to have been allowed.  In particular, the \code{RESB} family of
%   instructions should have been taking a critical expression all
%   along.
%
%\item{Fix the EVEX (AVX-512) versions of the \code{VPBROADCAST}, \code{VPEXTR},
%   and \code{VPINSR} instructions.
%
%\item{Support contracted forms of additional instructions.  As a general
%   rule, if an instruction has a non-destructive source immediately
%   after a destination register that isn't used as an input, NASM
%   supports omitting that source register, using the destination
%   register as that value.  This among other things makes it easier to
%   convert SSE code to the equivalent AVX code:
%
%\c      addps xmm1,xmm0                       ; SSE instruction
%\c      vaddps ymm1,ymm1,ymm0                 ; AVX official long form
%\c      vaddps ymm1,ymm0                      ; AVX contracted form
%
%\item{Fix Codeview malformed compiler version record.
%
%\item{Add the \code{CLWB} and \code{PCOMMIT} instructions.  Note that the
%   \code{PCOMMIT} instruction has been deprecated and will never be
%   included in a shipping product; it is included for completeness
%   only.
%
%\item{Add the \code{%pragma} preprocessor directive for soft-error directives.
%
%\item{Add the \code{RDPID} instruction.
%
%\xsubsection{cl-2.12.02} Version 2.12.02
%
%\item{Fix preprocessor errors, especially \code{%error} and \code{%warning},
%   inside \code{%if} statements.
%
%\item{Fix relative relocations in 32-bit Mach-O.
%
%\item{More Codeview debug format fixes.
%
%\item{If the MASM \code{PTR} keyword is encountered, issue a warning.  This is
%   much more likely to indicate a MASM-ism encountered in NASM than it
%   is a valid label.  This warning can be suppressed with \code{-w-ptr},
%   the \code{[warning]} directive (see \nref{opt-w}) or by the macro
%   definition \code{%idefine ptr $%?} (see \nref{selfref%?}).
%
%\item{When an error or a warning comes from the expansion of a multi-line
%   macro, display the file and line numbers for the expanded macros.
%   Macros defined with \code{.nolist} do not get displayed.
%
%\item{Add macros \code{ilog2fw()} and \code{ilog2cw()} to the \code{ifunc} macro
%   package.  See \nref{ilog2}.
%
%
%\xsubsection{cl-2.12.01} Version 2.12.01
%
%\item{Portability fixes for some platforms.
%
%\item{Fix error when not specifying a list file.
%
%\item{Correct the handling of macro-local labels in the Codeview
%   debugging format.
%
%\item{Add \code{CLZERO}, \code{MONITORX} and \code{MWAITX} instructions.
%
%
%\xsubsection{cl-2.12} Version 2.12
%
%\item{Major fixes to the \code{macho} backend (\nref{machofmt}); earlier versions
%   would produce invalid symbols and relocations on a regular basis.
%
%\item{Support for thread-local storage in Mach-O.
%
%\item{Support for arbitrary sections in Mach-O.
%
%\item{Fix wrong negative size treated as a big positive value passed into
%   backend causing NASM to crash.
%
%\item{Fix handling of zero-extending unsigned relocations, we have been printing
%   wrong message and forgot to assign segment with predefined value before
%   passing it into output format.
%
%\item{Fix potential write of oversized (with size greater than allowed in
%   output format) relative relocations.
%
%\item{Portability fixes for building NASM with the LLVM compiler.
%
%\item{Add support of Codeview version 8 (\code{cv8}) debug format for
%   \code{win32} and \code{win64} formats in the \code{COFF} backend,
%   see \nref{codeview}.
%
%\item{Allow 64-bit outputs in 16/32-bit only backends.  Unsigned 64-bit
%   relocations are zero-extended from 32-bits with a warning
%   (suppressible via \code{-w-zext-reloc}); signed 64-bit relocations are
%   an error.
%
%\item{Line numbers in list files now correspond to the lines in the source
%   files, instead of simply being sequential.
%
%\item{There is now an official 64-bit (x64 a.k.a. x86-64) build for Windows.
%
%
%\xsubsection{cl-2.11.09} Version 2.11.09
%
%\item{Fix potential stack overwrite in \code{macho32} backend.
%
%\item{Fix relocation records in \code{macho64} backend.
%
%\item{Fix symbol lookup computation in \code{macho64} backend.
%
%\item{Adjust \code{.symtab} and \code{.rela.text} sections alignments to 8 bytes
%   in \code{elf64} backed.
%
%\item{Fix section length computation in \code{bin} backend which leaded in incorrect
%   relocation records.
%
%\xsubsection{cl-2.11.08} Version 2.11.08
%
%\item{Fix section length computation in \code{bin} backend which leaded in incorrect
%   relocation records.
%
%\item{Add a warning for numeric preprocessor definitions passed via command
%   line which might have unexpected results otherwise.
%
%\item{Add ability to specify a module name record in \code{rdoff} linker with
%   \code{-mn} option.
%
%\item{Increase label length capacity up to 256 bytes in \code{rdoff} backend for
%   FreePascal sake, which tends to generate very long labels for procedures.
%
%\item{Fix segmentation failure when rip addressing is used in \code{macho64} backend.
%
%\item{Fix access on out of memory when handling strings with a single
%   grave. We have sixed similar problem in previous release but not
%   all cases were covered.
%
%\item{Fix NULL dereference in disassembled on \code{BND} instruction.
%
%\xsubsection{cl-2.11.07} Version 2.11.07
%
%\item{Fix 256 bit \code{VMOVNTPS} instruction.
%
%\item{Fix \code{-MD} option handling, which was rather broken in previous
%release changing command line api.
%
%\item{Fix access to unitialized space when handling strings with
%a single grave.
%
%\item{Fix nil dereference in handling memory reference parsing.
%
%\xsubsection{cl-2.11.06} Version 2.11.06
%
%\item{Update AVX512 instructions based on the Extension Reference (319433-021 Sept
%2014).
%
%\item{Fix the behavior of \code{-MF} and \code{-MD} options (Bugzilla 3392280)
%
%\item{Updated Win32 Makefile to fix issue with build
%
%\xsubsection{cl-2.11.05} Version 2.11.05
%
%\item{Add \code{--v} as an alias for \code{-v} (see \nref{opt-v}), for
%command-line compatibility with Yasm.
%
%\item{Fix a bug introduced in 2.11.03 whereby certain instructions would
%contain multiple REX prefixes, and thus be corrupt.
%
%\xsubsection{cl-2.11.04} Version 2.11.04
%
%\item{Removed an invalid error checking code. Sometimes a memref only with
%a displacement can also set an evex flag. For example:
%
%\c       vmovdqu32 [0xabcd]{k1}, zmm0
%
%\item{Fixed a bug in disassembler that EVEX.L'L vector length was not matched
%when EVEX.b was set because it was simply considered as EVEC.RC.
%Separated EVEX.L'L case from EVEX.RC which is ignored in matching.
%
%\xsubsection{cl-2.11.03} Version 2.11.03
%
%\item{Fix a bug there REX prefixes were missing on instructions inside a
%\code{TIMES} statement.
%
%\xsubsection{cl-2.11.02} Version 2.11.02
%
%\item{Add the \code{XSAVEC}, \code{XSAVES} and \code{XRSTORS} family instructions.
%
%\item{Add the \code{CLFLUSHOPT} instruction.
%
%\xsubsection{cl-2.11.01} Version 2.11.01
%
%\item{Allow instructions which implicitly uses \code{XMM0} (\code{VBLENDVPD},
%\code{VBLENDVPS}, \code{PBLENDVB} and \code{SHA256RNDS2}) to be specified
%without an explicit \code{xmm0} on the assembly line.  In other words,
%the following two lines produce the same output:
%
%\c      vblendvpd xmm2,xmm1,xmm0      ; Last operand is fixed xmm0
%\c      vblendvpd xmm2,xmm1           ; Implicit xmm0 omitted
%
%\item{In the ELF backends, don't crash the assembler if \code{section align}
%is specified without a value.
%
%\xsubsection{cl-2.11} Version 2.11
%
%\item{Add support for the Intel AVX-512 instruction set:
%
%\item{16 new, 512-bit SIMD registers. Total 32 \code{(ZMM0 ~ ZMM31)}
%
%\item{8 new opmask registers \code{(K0 ~ K7)}. One of 7 registers \code{(K1 ~ K7)} can
%be used as an opmask for conditional execution.
%
%\item{A new EVEX encoding prefix. EVEX is based on VEX and provides more
%capabilities: opmasks, broadcasting, embedded rounding and compressed
%displacements.
%
%\c  - opmask
%\c      VDIVPD zmm0{k1}{z}, zmm1, zmm3  ; conditional vector operation
%\c                                      ; using opmask k1.
%\c                                      ; {z} is for zero-masking
%\c  - broadcasting
%\c      VDIVPS zmm4, zmm5, [rbx]{1to16} ; load single-precision float and
%\c                                      ; replicate it 16 times. 32 * 16 = 512
%\c  - embedded rounding
%\c      VCVTSI2SD xmm6, xmm7, {rz-sae}, rax ; round toward zero. note that it
%\c                                       ; is used as if a separate operand.
%\c                                       ; it comes after the last SIMD operand
%
%\item{Add support for \code{ZWORD} (512 bits), \code{DZ} and \code{RESZ}.
%
%\item{Add support for the MPX and SHA instruction sets.
%
%\item{Better handling of section redefinition.
%
%\item{Generate manpages when running \code{'make dist'}.
%
%\item{Handle all token chains in mmacro params range.
%
%\item{Support split [base,index] effective address:
%
%\c      mov eax,[eax+8,ecx*4]   ; eax=base, ecx=index, 4=scale, 8=disp
%
%This is expected to be most useful for the MPX instructions.
%
%\item{Support \code{BND} prefix for branch instructions (for MPX).
%
%\item{The \code{DEFAULT} directive can now take \code{BND} and \code{NOBND}
%options to indicate whether all relevant branches should be getting
%\code{BND} prefixes.  This is expected to be the normal for use in MPX
%code.
%
%\item{Add \code{{evex}}, \code{{vex3}} and \code{{vex2}} instruction prefixes to
%have NASM encode the corresponding instruction, if possible, with an EVEX,
%3-byte VEX, or 2-byte VEX prefix, respectively.
%
%\item{Support for section names longer than 8 bytes in Win32/Win64 COFF.
%
%\item{The \code{NOSPLIT} directive by itself no longer forces a single
%register to become an index register, unless it has an explicit
%multiplier.
%
%\c      mov eax,[nosplit eax]       ; eax as base register
%\c      mov eax,[nosplit eax*1]     ; eax as index register
%
%\xsubsection{cl-2.10.09} Version 2.10.09
%
%\item{Pregenerate man pages.
%
%\xsubsection{cl-2.10.08} Version 2.10.08
%
%\item{Fix \code{VMOVNTDQA}, \code{MOVNTDQA} and \code{MOVLPD} instructions.
%
%\item{Fix collision for \code{VGATHERQPS}, \code{VPGATHERQD} instructions.
%
%\item{Fix \code{VPMOVSXBQ}, \code{VGATHERQPD}, \code{VSPLLW} instructions.
%
%\item{Add a bunch of AMD TBM instructions.
%
%\item{Fix potential stack overwrite in numbers conversion.
%
%\item{Allow byte size in \code{PREFETCHTx} instructions.
%
%\item{Make manual pages up to date.
%
%\item{Make \code{F3} and \code{F2} SSE prefixes to override \code{66}.
%
%\item{Support of AMD SVM instructions in 32 bit mode.
%
%\item{Fix near offsets code generation for \code{JMP}, \code{CALL} instrictions
%in long mode.
%
%\item{Fix preprocessor parse regression when id is expanding to a whitespace.
%
%\xsubsection{cl-2.10.07} Version 2.10.07
%
%\item{Fix line continuation parsing being broken in previous version.
%
%\xsubsection{cl-2.10.06} Version 2.10.06
%
%\item{Always quote the dependency source names when using the automatic
%dependency generation options.
%
%\item{If no dependency target name is specified via the \code{-MT} or
%\code{-MQ} options, quote the default output name.
%
%\item{Fix assembly of shift operations in \code{CPU 8086} mode.
%
%\item{Fix incorrect generation of explicit immediate byte for shift by 1
%under certain circumstances.
%
%\item{Fix assembly of the \code{VPCMPGTQ} instruction.
%
%\item{Fix RIP-relative relocations in the \code{macho64} backend.
%
%\xsubsection{cl-2.10.05} Version 2.10.05
%
%\item{Add the \code{CLAC} and \code{STAC} instructions.
%
%\xsubsection{cl-2.10.04} Version 2.10.04
%
%\item{Add back the inadvertently deleted 256-bit version of the \code{VORPD}
%instruction.
%
%\item{Correct disassembly of instructions starting with byte \code{82} hex.
%
%\item{Fix corner cases in token pasting, for example:
%
%\c    %define N 1e%++%+ 5
%\c            dd N, 1e+5
%
%\xsubsection{cl-2.10.03} Version 2.10.03
%
%\item{Correct the assembly of the instruction:
%
%\c XRELEASE MOV [absolute],AL
%
%\> Previous versions would incorrectly generate \code{F3 A2} for this
%instruction and issue a warning; correct behavior is to emit \code{F3 88
%05}.
%
%\xsubsection{cl-2.10.02} Version 2.10.02
%
%\item{Add the \code{ifunc} macro package with integer functions, currently
%only integer logarithms.  See \nref{pkg_ifunc}.
%
%\item{Add the \code{RDSEED}, \code{ADCX} and \code{ADOX} instructions.
%
%\xsubsection{cl-2.10.01} Version 2.10.01
%
%\item{Add missing VPMOVMSKB instruction with reg32, ymmreg operands.
%
%\xsubsection{cl-2.10} Version 2.10
%
%\item{When optimization is enabled, \code{mov r64,imm} now optimizes to the
%  shortest form possible between:
%
%\c      mov r32,imm32                   ;  5 bytes
%\c      mov r64,imm32                   ;  7 bytes
%\c      mov r64,imm64                   ; 10 bytes
%
%\> To force a specific form, use the \code{STRICT} keyword, see \nref{strict}.
%
%\item{Add support for the Intel AVX2 instruction set.
%
%\item{Add support for Bit Manipulation Instructions 1 and 2.
%
%\item{Add support for Intel Transactional Synchronization Extensions (TSX).
%
%\item{Add support for x32 ELF (32-bit ELF with the CPU in 64-bit mode.)
%   See \nref{elffmt}.
%
%\item{Add support for bigendian UTF-16 and UTF-32.  See \nref{unicode}.
%
%\xsubsection{cl-2.09.10} Version 2.09.10
%
%\item{Fix up NSIS script to protect uninstaller against registry keys
%   absence or corruption. It brings in a few additional questions
%   to a user during deinstallation procedure but still it is better
%   than unpredictable file removal.
%
%\xsubsection{cl-2.09.09} Version 2.09.09
%
%\item{Fix initialization of section attributes of \code{bin} output format.
%
%\item{Fix \code{mach64} output format bug that crashes NASM due to NULL symbols.
%
%
%\xsubsection{cl-2.09.08} Version 2.09.08
%
%\item{Fix \code{__OUTPUT_FORMAT__} assignment when output driver alias
%   is used. For example when \code{-f elf} is used \code{__OUTPUT_FORMAT__}
%   must be set to \code{elf}, if \code{-f elf32} is used \code{__OUTPUT_FORMAT__}
%   must be assigned accordingly, i.e. to \code{elf32}. The rule applies to
%   all output driver aliases. See \nref{ofmtm}.
%
%
%\xsubsection{cl-2.09.07} Version 2.09.07
%
%\item{Fix attempts to close same file several times
%   when \code{-a} option is used.
%
%\item{Fixes for VEXTRACTF128, VMASKMOVPS encoding.
%
%
%\xsubsection{cl-2.09.06} Version 2.09.06
%
%\item{Fix missed section attribute initialization in \code{bin} output target.
%
%
%\xsubsection{cl-2.09.05} Version 2.09.05
%
%\item{Fix arguments encoding for VPEXTRW instruction.
%
%\item{Remove invalid form of VPEXTRW instruction.
%
%\item{Add \code{VLDDQU} as alias for \code{VLDQQU} to
%   match specification.
%
%
%\xsubsection{cl-2.09.04} Version 2.09.04
%
%\item{Fix incorrect labels offset for VEX intructions.
%
%\item{Eliminate bogus warning on implicit operand size override.
%
%\item{\code{%if} term could not handle 64 bit numbers.
%
%\item{The COFF backend was limiting relocations number to 16 bits even if
%   in real there were a way more relocations.
%
%
%\xsubsection{cl-2.09.03} Version 2.09.03
%
%\item{Print \code{%macro} name inside \code{%rep} blocks on error.
%
%\item{Fix preprocessor expansion behaviour. It happened sometime
%   too early and sometime simply wrong. Move behaviour back to
%   the origins (down to NASM 2.05.01).
%
%\item{Fix unitialized data dereference on OMF output format.
%
%\item{Issue warning on unterminated \code{%{} construct.
%
%\item{Fix for documentation typo.
%
%
%\xsubsection{cl-2.09.02} Version 2.09.02
%
%\item{Fix reversed tokens when \code{%deftok} produces more than one output token.
%
%\item{Fix segmentation fault on disassembling some VEX instructions.
%
%\item{Missing \code{%endif} did not always cause error.
%
%\item{Fix typo in documentation.
%
%\item{Compound context local preprocessor single line macro identifiers
%  were not expanded early enough and as result lead to unresolved
%  symbols.
%
%
%\xsubsection{cl-2.09.01} Version 2.09.01
%
%\item{Fix NULL dereference on missed %deftok second parameter.
%
%\item{Fix NULL dereference on invalid %substr parameters.
%
%
%\xsubsection{cl-2.09} Version 2.09
%
%\item{Fixed assignment the magnitude of \code{%rep} counter. It is limited
%  to 62 bits now.
%
%\item{Fixed NULL dereference if argument of \code{%strlen} resolves
%  to whitespace. For example if nonexistent macro parameter is used.
%
%\item{\code{%ifenv}, \code{%elifenv}, \code{%ifnenv}, and \code{%elifnenv} directives
%  introduced.  See \nref{ifenv}.
%
%\item{Fixed NULL dereference if environment variable is missed.
%
%\item{Updates of new AVX v7 Intel instructions.
%
%\item{\code{PUSH imm32} is now officially documented.
%
%\item{Fix for encoding the LFS, LGS and LSS in 64-bit mode.
%
%\item{Fixes for compatibility with OpenWatcom compiler and DOS 8.3 file
%  format limitation.
%
%\item{Macros parameters range expansion introduced. See \nref{mlmacrange}.
%
%\item{Backward compatibility on expanging of local sigle macros restored.
%
%\item{8 bit relocations for \code{elf} and \code{bin} output formats are introduced.
%
%\item{Short intersegment jumps are permitted now.
%
%\item{An alignment more than 64 bytes are allowed for \code{win32},
%  \code{win64} output formats.
%
%\item{\code{SECTALIGN} directive introduced. See \nref{sectalign}.
%
%\item{\code{nojmp} option introduced in \code{smartalign} package. See
%  \nref{pkg_smartalign}.
%
%\item{Short aliases \code{win}, \code{elf} and \code{macho} for output formats are
%  introduced.  Each stands for \code{win32}, \code{elf32} and \code{macho32}
%  accordingly.
%
%\item{Faster handling of missing directives implemented.
%
%\item{Various small improvements in documentation.
%
%\item{No hang anymore if unable to open malloc.log file.
%
%\item{The environments without vsnprintf function are able to build nasm again.
%
%\item{AMD LWP instructions updated.
%
%\item{Tighten EA checks. We warn a user if there overflow in EA addressing.
%
%\item{Make \code{-Ox} the default optimization level.  For the legacy
%  behavior, specify \code{-O0} explicitly.  See \nref{opt-O}.
%
%\item{Environment variables read with \code{%!} or tested with \code{%ifenv}
%  can now contain non-identifier characters if surrounded by quotes.
%  See \nref{getenv}.
%
%\item{Add a new standard macro package \code{%use fp} for floating-point
%  convenience macros.  See \nref{pkg_fp}.
%
%
%\xsubsection{cl-2.08.02} Version 2.08.02
%
%\item{Fix crash under certain circumstances when using the \code{%+} operator.
%
%
%\xsubsection{cl-2.08.01} Version 2.08.01
%
%\item{Fix the \code{%use} statement, which was broken in 2.08.
%
%
%\xsubsection{cl-2.08} Version 2.08
%
%\item{A number of enhancements/fixes in macros area.
%
%\item{Support for converting strings to tokens.  See \nref{deftok}.
%
%\item{Fuzzy operand size logic introduced.
%
%\item{Fix COFF stack overrun on too long export identifiers.
%
%\item{Fix Macho-O alignment bug.
%
%\item{Fix crashes with -fwin32 on file with many exports.
%
%\item{Fix stack overrun for too long [DEBUG id].
%
%\item{Fix incorrect sbyte usage in IMUL (hit only if optimization
%  flag passed).
%
%\item{Append ending token for \code{.stabs} records in the ELF output format.
%
%\item{New NSIS script which uses ModernUI and MultiUser approach.
%
%\item{Visual Studio 2008 NASM integration (rules file).
%
%\item{Warn a user if a constant is too long (and as result will be stripped).
%
%\item{The obsoleted pre-XOP AMD SSE5 instruction set which was never actualized
%  was removed.
%
%\item{Fix stack overrun on too long error file name passed from the command line.
%
%\item{Bind symbols to the .text section by default (ie in case if SECTION
%  directive was omitted) in the ELF output format.
%
%\item{Fix sync points array index wrapping.
%
%\item{A few fixes for FMA4 and XOP instruction templates.
%
%\item{Add AMD Lightweight Profiling (LWP) instructions.
%
%\item{Fix the offset for \code{%arg} in 64-bit mode.
%
%\item{An undefined local macro (\code{%$}) no longer matches a global macro
%  with the same name.
%
%\item{Fix NULL dereference on too long local labels.
%
%
%\xsubsection{cl-2.07} Version 2.07
%
%\item{NASM is now under the 2-clause BSD license.  See \nref{legal}.
%
%\item{Fix the section type for the \code{.strtab} section in the \code{elf64}
%  output format.
%
%\item{Fix the handling of \code{COMMON} directives in the \code{obj} output format.
%
%\item{New \code{ith} and \code{srec} output formats; these are variants of the
%  \code{bin} output format which output Intel hex and Motorola S-records,
%  respectively.  See \nref{ithfmt} and \nref{srecfmt}.
%
%\item{\code{rdf2ihx} replaced with an enhanced \code{rdf2bin}, which can output
%  binary, COM, Intel hex or Motorola S-records.
%
%\item{The Windows installer now puts the NASM directory first in the
%  \code{PATH} of the "NASM Shell".
%
%\item{Revert the early expansion behavior of \code{%+} to pre-2.06 behavior:
%  \code{%+} is only expanded late.
%
%\item{Yet another Mach-O alignment fix.
%
%\item{Don't delete the list file on errors.  Also, include error and
%  warning information in the list file.
%
%\item{Support for 64-bit Mach-O output, see \nref{machofmt}.
%
%\item{Fix assert failure on certain operations that involve strings with
%  high-bit bytes.
%
%
%\xsubsection{cl-2.06} Version 2.06
%
%\item{This release is dedicated to the memory of Charles A. Crayne, long
%  time NASM developer as well as moderator of \code{comp.lang.asm.x86} and
%  author of the book \e{Serious Assembler}.  We miss you, Chuck.
%
%\item{Support for indirect macro expansion (\code{%[...]}).  See \nref{indmacro}.
%
%\item{\code{%pop} can now take an argument, see \nref{pushpop}.
%
%\item{The argument to \code{%use} is no longer macro-expanded.  Use
%  \code{%[...]} if macro expansion is desired.
%
%\item{Support for thread-local storage in ELF32 and ELF64.  See \nref{elftls}.
%
%\item{Fix crash on \code{%ifmacro} without an argument.
%
%\item{Correct the arguments to the \code{POPCNT} instruction.
%
%\item{Fix section alignment in the Mach-O format.
%
%\item{Update AVX support to version 5 of the Intel specification.
%
%\item{Fix the handling of accesses to context-local macros from higher
%  levels in the context stack.
%
%\item{Treat \code{WAIT} as a prefix rather than as an instruction, thereby
%  allowing constructs like \code{O16 FSAVE} to work correctly.
%
%\item{Support for structures with a non-zero base offset. See \nref{struc}.
%
%\item{Correctly handle preprocessor token concatenation (see \nref{concat})
%   involving floating-point numbers.
%
%\item{The \code{PINSR} series of instructions have been corrected and
%   rationalized.
%
%\item{Removed AMD SSE5, replaced with the new XOP/FMA4/CVT16 (rev 3.03)
%   spec.
%
%\item{The ELF backends no longer automatically generate a \code{.comment} section.
%
%\item{Add additional "well-known" ELF sections with default attributes.  See
%   \nref{elfsect}.
%
%
%\xsubsection{cl-2.05.01} Version 2.05.01
%
%\item{Fix the \code{-w}/\code{-W} option parsing, which was broken in NASM 2.05.
%
%
%\xsubsection{cl-2.05} Version 2.05
%
%\item{Fix redundant REX.W prefix on \code{JMP reg64}.
%
%\item{Make the behaviour of \code{-O0} match NASM 0.98 legacy behavior.
%  See \nref{opt-O}.
%
%\item{\code{-w-user} can be used to suppress the output of \code{%warning} directives.
%  See \nref{opt-w}.
%
%\item{Fix bug where \code{ALIGN} would issue a full alignment datum instead of
%  zero bytes.
%
%\item{Fix offsets in list files.
%
%\item{Fix \code{%include} inside multi-line macros or loops.
%
%\item{Fix error where NASM would generate a spurious warning on valid
%  optimizations of immediate values.
%
%\item{Fix arguments to a number of the \code{CVT} SSE instructions.
%
%\item{Fix RIP-relative offsets when the instruction carries an immediate.
%
%\item{Massive overhaul of the ELF64 backend for spec compliance.
%
%\item{Fix the Geode \code{PFRCPV} and \code{PFRSQRTV} instruction.
%
%\item{Fix the SSE 4.2 \code{CRC32} instruction.
%
%
%\xsubsection{cl-2.04} Version 2.04
%
%\item{Sanitize macro handing in the \code{%error} directive.
%
%\item{New \code{%warning} directive to issue user-controlled warnings.
%
%\item{\code{%error} directives are now deferred to the final assembly phase.
%
%\item{New \code{%fatal} directive to immediately terminate assembly.
%
%\item{New \code{%strcat} directive to join quoted strings together.
%
%\item{New \code{%use} macro directive to support standard macro directives.  See
%  \nref{use}.
%
%\item{Excess default parameters to \code{%macro} now issues a warning by default.
%  See \nref{mlmacro}.
%
%\item{Fix \code{%ifn} and \code{%elifn}.
%
%\item{Fix nested \code{%else} clauses.
%
%\item{Correct the handling of nested \code{%rep}s.
%
%\item{New \code{%unmacro} directive to undeclare a multi-line macro.
%  See \nref{unmacro}.
%
%\item{Builtin macro \code{__PASS__} which expands to the current assembly pass.
%  See \nref{pass_macro}.
%
%\item{\code{__utf16__} and \code{__utf32__} operators to generate UTF-16 and UTF-32
%  strings.  See \nref{unicode}.
%
%\item{Fix bug in case-insensitive matching when compiled on platforms that
%  don't use the \code{configure} script.  Of the official release binaries,
%  that only affected the OS/2 binary.
%
%\item{Support for x87 packed BCD constants.  See \nref{bcdconst}.
%
%\item{Correct the \code{LTR} and \code{SLDT} instructions in 64-bit mode.
%
%\item{Fix unnecessary REX.W prefix on indirect jumps in 64-bit mode.
%
%\item{Add AVX versions of the AES instructions (\code{VAES}...).
%
%\item{Fix the 256-bit FMA instructions.
%
%\item{Add 256-bit AVX stores per the latest AVX spec.
%
%\item{VIA XCRYPT instructions can now be written either with or without
%  \code{REP}, apparently different versions of the VIA spec wrote them
%  differently.
%
%\item{Add missing 64-bit \code{MOVNTI} instruction.
%
%\item{Fix the operand size of \code{VMREAD} and \code{VMWRITE}.
%
%\item{Numerous bug fixes, especially to the AES, AVX and VTX instructions.
%
%\item{The optimizer now always runs until it converges.  It also runs even
%  when disabled, but doesn't optimize.  This allows most forward references
%  to be resolved properly.
%
%\item{\code{%push} no longer needs a context identifier; omitting the context
%  identifier results in an anonymous context.
%
%
%\xsubsection{cl-2.03.01} Version 2.03.01
%
%\item{Fix buffer overflow in the listing module.
%
%\item{Fix the handling of hexadecimal escape codes in `...` strings.
%
%\item{The Postscript/PDF documentation has been reformatted.
%
%\item{The \code{-F} option now implies \code{-g}.
%
%
%\xsubsection{cl-2.03} Version 2.03
%
%\item{Add support for Intel AVX, CLMUL and FMA instructions,
%including YMM registers.
%
%\item{\code{dy}, \code{resy} and \code{yword} for 32-byte operands.
%
%\item{Fix some SSE5 instructions.
%
%\item{Intel \code{INVEPT}, \code{INVVPID} and \code{MOVBE} instructions.
%
%\item{Fix checking for critical expressions when the optimizer is enabled.
%
%\item{Support the DWARF debugging format for ELF targets.
%
%\item{Fix optimizations of signed bytes.
%
%\item{Fix operation on bigendian machines.
%
%\item{Fix buffer overflow in the preprocessor.
%
%\item{\code{SAFESEH} support for Win32, \code{IMAGEREL} for Win64 (SEH).
%
%\item{\code{%?} and \code{%??} to refer to the name of a macro itself.  In particular,
%\code{%idefine keyword $%?} can be used to make a keyword "disappear".
%
%\item{New options for dependency generation: \code{-MD}, \code{-MF},
%\code{-MP}, \code{-MT}, \code{-MQ}.
%
%\item{New preprocessor directives \code{%pathsearch} and \code{%depend}; INCBIN
%reimplemented as a macro.
%
%\item{\code{%include} now resolves macros in a sane manner.
%
%\item{\code{%substr} can now be used to get other than one-character substrings.
%
%\item{New type of character/string constants, using backquotes (\code{`...`}),
%which support C-style escape sequences.
%
%\item{\code{%defstr} and \code{%idefstr} to stringize macro definitions before
%creation.
%
%\item{Fix forward references used in \code{EQU} statements.
%
%
%\xsubsection{cl-2.02} Version 2.02
%
%\item{Additional fixes for MMX operands with explicit \code{qword}, as well as
%  (hopefully) SSE operands with \code{oword}.
%
%\item{Fix handling of truncated strings with \code{DO}.
%
%\item{Fix segfaults due to memory overwrites when floating-point constants
%  were used.
%
%\item{Fix segfaults due to missing include files.
%
%\item{Fix OpenWatcom Makefiles for DOS and OS/2.
%
%\item{Add autogenerated instruction list back into the documentation.
%
%\item{ELF: Fix segfault when generating stabs, and no symbols have been
%  defined.
%
%\item{ELF: Experimental support for DWARF debugging information.
%
%\item{New compile date and time standard macros.
%
%\item{\code{%ifnum} now returns true for negative numbers.
%
%\item{New \code{%iftoken} test for a single token.
%
%\item{New \code{%ifempty} test for empty expansion.
%
%\item{Add support for the \code{XSAVE} instruction group.
%
%\item{Makefile for Netware/gcc.
%
%\item{Fix issue with some warnings getting emitted way too many times.
%
%\item{Autogenerated instruction list added to the documentation.
%
%
%\xsubsection{cl-2.01} Version 2.01
%
%\item{Fix the handling of MMX registers with explicit \code{qword} tags on
%  memory (broken in 2.00 due to 64-bit changes.)
%
%\item{Fix the PREFETCH instructions.
%
%\item{Fix the documentation.
%
%\item{Fix debugging info when using \code{-f elf}
%(backwards compatibility alias for \code{-f elf32}).
%
%\item{Man pages for rdoff tools (from the Debian project.)
%
%\item{ELF: handle large numbers of sections.
%
%\item{Fix corrupt output when the optimizer runs out of passes.
%
%
%\xsubsection{cl-2.00} Version 2.00
%
%\item{Added c99 data-type compliance.
%
%\item{Added general x86-64 support.
%
%\item{Added win64 (x86-64 COFF) output format.
%
%\item{Added \code{__BITS__} standard macro.
%
%\item{Renamed the \code{elf} output format to \code{elf32} for clarity.
%
%\item{Added \code{elf64} and \code{macho} (MacOS X) output formats.
%
%\item{Added Numeric constants in \code{dq} directive.
%
%\item{Added \code{oword}, \code{do} and \code{reso} pseudo operands.
%
%\item{Allow underscores in numbers.
%
%\item{Added 8-, 16- and 128-bit floating-point formats.
%
%\item{Added binary, octal and hexadecimal floating-point.
%
%\item{Correct the generation of floating-point constants.
%
%\item{Added floating-point option control.
%
%\item{Added Infinity and NaN floating point support.
%
%\item{Added ELF Symbol Visibility support.
%
%\item{Added setting OSABI value in ELF header directive.
%
%\item{Added Generate Makefile Dependencies option.
%
%\item{Added Unlimited Optimization Passes option.
%
%\item{Added \code{%IFN} and \code{%ELIFN} support.
%
%\item{Added Logical Negation Operator.
%
%\item{Enhanced Stack Relative Preprocessor Directives.
%
%\item{Enhanced ELF Debug Formats.
%
%\item{Enhanced Send Errors to a File option.
%
%\item{Added SSSE3, SSE4.1, SSE4.2, SSE5 support.
%
%\item{Added a large number of additional instructions.
%
%\item{Significant performance improvements.
%
%\item{\code{-w+warning} and \code{-w-warning} can now be written as -Wwarning and
% -Wno-warning, respectively.  See \nref{opt-w}.
%
%\item{Add \code{-w+error} to treat warnings as errors.  See \nref{opt-w}.
%
%\item{Add \code{-w+all} and \code{-w-all} to enable or disable all suppressible
% warnings. See \nref{opt-w}.
%
%
%\xchapter{cl-0.98.xx} NASM 0.98 Series
%
%The 0.98 series was the production versions of NASM from 1999 to 2007.
%
%\xsubsection{cl-0.98.39} Version 0.98.39
%
%\item{fix buffer overflow
%
%\item{fix outas86's \code{.bss} handling
%
%\item{"make spotless" no longer deletes config.h.in.
%
%\item{\code{%(el)if(n)idn} insensitivity to string quotes difference (#809300).
%
%\item{(nasm.c)\code{__OUTPUT_FORMAT__} changed to string value instead of symbol.
%
%\xsubsection{cl-0.98.38} Version 0.98.38
%
%\item{Add Makefile for 16-bit DOS binaries under OpenWatcom, and modify
%  \code{mkdep.pl} to be able to generate completely pathless dependencies, as
%  required by OpenWatcom wmake (it supports path searches, but not
%  explicit paths.)
%
%\item{Fix the \code{STR} instruction.
%
%\item{Fix the ELF output format, which was broken under certain
%    circumstances due to the addition of stabs support.}
%
%\item{Quick-fix Borland format debug-info for \code{-f obj}}
%
%\item{Fix for \code{\%rep} with no arguments (#560568)}
%
%\item{Fix concatenation of preprocessor function call (#794686)}
%
%\item{Fix long label causes coredump (#677841)}
%
%\item{Use autoheader as well as autoconf to keep configure from generating
%  ridiculously long command lines.}
%
%\item{Make sure that all of the formats which support debugging output
%  actually will suppress debugging output when \code{-g} not specified.}
%
%\xsubsection{cl-0.98.37} Version 0.98.37
%
%
%\item{Paths given in \code{-I} switch searched for \code{incbin}-ed as
%  well as \code{%include}-ed files.
%
%\item{Added stabs debugging for the ELF output format, patch from
%  Martin Wawro.
%
%\item{Fix \code{output/outbin.c} to allow origin > 80000000h.
%
%\item{Make \code{-U} switch work.
%
%\item{Fix the use of relative offsets with explicit prefixes, e.g.
%\code{a32 loop foo}.
%
%\item{Remove \code{backslash()}.
%
%\item{Fix the \code{SMSW} and \code{SLDT} instructions.
%
%\item{\code{-O2} and \code{-O3} are no longer aliases for \code{-O10} and \code{-O15}.
%If you mean the latter, please say so! :)
%
%\xsubsection{cl-0.98.36} Version 0.98.36
%
%
%\item{Update rdoff - librarian/archiver - common rec - docs!
%
%\item{Fix signed/unsigned problems.
%
%\item{Fix \code{JMP FAR label} and \code{CALL FAR label}.
%
%\item{Add new multisection support - map files - fix align bug
%
%\item{Fix sysexit, movhps/movlps reg,reg bugs in insns.dat
%
%\item{\code{Q} or \code{O} suffixes indicate octal
%
%\item{Support Prescott new instructions (PNI).
%
%\item{Cyrix \code{XSTORE} instruction.
%
%
%\xsubsection{cl-0.98.35} Version 0.98.35
%
%\item{Fix build failure on 16-bit DOS (Makefile.bc3 workaround for compiler bug.)
%
%\item{Fix dependencies and compiler warnings.
%
%\item{Add "const" in a number of places.
%
%\item{Add -X option to specify error reporting format (use -Xvc to
%  integrate with Microsoft Visual Studio.)
%
%\item{Minor changes for code legibility.
%
%\item{Drop use of tmpnam() in rdoff (security fix.)
%
%
%\xsubsection{cl-0.98.34} Version 0.98.34
%
%\item{Correct additional address-size vs. operand-size confusions.
%
%\item{Generate dependencies for all Makefiles automatically.
%
%\item{Add support for unimplemented (but theoretically available)
%  registers such as tr0 and cr5.  Segment registers 6 and 7 are called
%  segr6 and segr7 for the operations which they can be represented.
%
%\item{Correct some disassembler bugs related to redundant address-size prefixes.
%  Some work still remains in this area.
%
%\item{Correctly generate an error for things like "SEG eax".
%
%\item{Add the JMPE instruction, enabled by "CPU IA64".
%
%\item{Correct compilation on newer gcc/glibc platforms.
%
%\item{Issue an error on things like "jmp far eax".
%
%
%\xsubsection{cl-0.98.33} Version 0.98.33
%
%\item{New __NASM_PATCHLEVEL__ and __NASM_VERSION_ID__ standard macros to
%  round out the version-query macros.  version.pl now understands
%  X.YYplWW or X.YY.ZZplWW as a version number, equivalent to
%  X.YY.ZZ.WW (or X.YY.0.WW, as appropriate).
%
%\item{New keyword "strict" to disable the optimization of specific
%  operands.
%
%\item{Fix the handing of size overrides with JMP instructions
%  (instructions such as "jmp dword foo".)
%
%\item{Fix the handling of "ABSOLUTE label", where "label" points into a
%  relocatable segment.
%
%\item{Fix OBJ output format with lots of externs.
%
%\item{More documentation updates.
%
%\item{Add -Ov option to get verbose information about optimizations.
%
%\item{Undo a braindead change which broke \code{%elif} directives.
%
%\item{Makefile updates.
%
%
%\xsubsection{cl-0.98.32} Version 0.98.32
%
%\item{Fix NASM crashing when \code{%macro} directives were left unterminated.
%
%\item{Lots of documentation updates.
%
%\item{Complete rewrite of the PostScript/PDF documentation generator.
%
%\item{The MS Visual C++ Makefile was updated and corrected.
%
%\item{Recognize .rodata as a standard section name in ELF.
%
%\item{Fix some obsolete Perl4-isms in Perl scripts.
%
%\item{Fix configure.in to work with autoconf 2.5x.
%
%\item{Fix a couple of "make cleaner" misses.
%
%\item{Make the normal "./configure && make" work with Cygwin.
%
%
%\xsubsection{cl-0.98.31} Version 0.98.31
%
%\item{Correctly build in a separate object directory again.
%
%\item{Derive all references to the version number from the version file.
%
%\item{New standard macros __NASM_SUBMINOR__ and __NASM_VER__ macros.
%
%\item{Lots of Makefile updates and bug fixes.
%
%\item{New \code{%ifmacro} directive to test for multiline macros.
%
%\item{Documentation updates.
%
%\item{Fixes for 16-bit OBJ format output.
%
%\item{Changed the NASM environment variable to NASMENV.
%
%
%\xsubsection{cl-0.98.30} Version 0.98.30
%
%\item{Changed doc files a lot: completely removed old READMExx and
%  Wishlist files, incorporating all information in CHANGES and TODO.
%
%\item{I waited a long time to rename zoutieee.c to (original) outieee.c
%
%\item{moved all output modules to output/ subdirectory.
%
%\item{Added 'make strip' target to strip debug info from nasm & ndisasm.
%
%\item{Added INSTALL file with installation instructions.
%
%\item{Added -v option description to nasm man.
%
%\item{Added dist makefile target to produce source distributions.
%
%\item{16-bit support for ELF output format (GNU extension, but useful.)
%
%
%\xsubsection{cl-0.98.28} Version 0.98.28
%
%\item{Fastcooked this for Debian's Woody release:
%Frank applied the INCBIN bug patch to 0.98.25alt and called
%it 0.98.28 to not confuse poor little apt-get.
%
%
%\xsubsection{cl-0.98.26} Version 0.98.26
%
%\item{Reorganised files even better from 0.98.25alt
%
%
%\xsubsection{cl-0.98.25alt} Version 0.98.25alt
%
%\item{Prettified the source tree. Moved files to more reasonable places.
%
%\item{Added findleak.pl script to misc/ directory.
%
%\item{Attempted to fix doc.
%
%
%\xsubsection{cl-0.98.25} Version 0.98.25
%
%\item{Line continuation character \code{\\}.
%
%\item{Docs inadvertantly reverted - "dos packaging".
%
%
%\xsubsection{cl-0.98.24p1} Version 0.98.24p1
%
%\item{FIXME: Someone, document this please.
%
%
%\xsubsection{cl-0.98.24} Version 0.98.24
%
%\item{Documentation - Ndisasm doc added to Nasm.doc.
%
%
%\xsubsection{cl-0.98.23} Version 0.98.23
%
%\item{Attempted to remove rdoff version1
%
%\item{Lino Mastrodomenico's patches to preproc.c (%$$ bug?).
%
%
%\xsubsection{cl-0.98.22} Version 0.98.22
%
%\item{Update rdoff2 - attempt to remove v1.
%
%
%\xsubsection{cl-0.98.21} Version 0.98.21
%
%\item{Optimization fixes.
%
%
%\xsubsection{cl-0.98.20} Version 0.98.20
%
%\item{Optimization fixes.
%
%
%\xsubsection{cl-0.98.19} Version 0.98.19
%
%\item{H. J. Lu's patch back out.
%
%
%\xsubsection{cl-0.98.18} Version 0.98.18
%
%\item{Added ".rdata" to "-f win32".
%
%
%\xsubsection{cl-0.98.17} Version 0.98.17
%
%\item{H. J. Lu's "bogus elf" patch. (Red Hat problem?)
%
%
%\xsubsection{cl-0.98.16} Version 0.98.16
%
%\item{Fix whitespace before "[section ..." bug.
%
%
%\xsubsection{cl-0.98.15} Version 0.98.15
%
%\item{Rdoff changes (?).
%
%\item{Fix fixes to memory leaks.
%
%
%\xsubsection{cl-0.98.14} Version 0.98.14
%
%\item{Fix memory leaks.
%
%
%\xsubsection{cl-0.98.13} Version 0.98.13
%
%\item{There was no 0.98.13
%
%
%\xsubsection{cl-0.98.12} Version 0.98.12
%
%\item{Update optimization (new function of "-O1")
%
%\item{Changes to test/bintest.asm (?).
%
%
%\xsubsection{cl-0.98.11} Version 0.98.11
%
%\item{Optimization changes.
%
%\item{Ndisasm fixed.
%
%
%\xsubsection{cl-0.98.10} Version 0.98.10
%
%\item{There was no 0.98.10
%
%
%\xsubsection{cl-0.98.09} Version 0.98.09
%
%\item{Add multiple sections support to "-f bin".
%
%\item{Changed GLOBAL_TEMP_BASE in outelf.c from 6 to 15.
%
%\item{Add "-v" as an alias to the "-r" switch.
%
%\item{Remove "#ifdef" from Tasm compatibility options.
%
%\item{Remove redundant size-overrides on "mov ds, ex", etc.
%
%\item{Fixes to SSE2, other insns.dat (?).
%
%\item{Enable uppercase "I" and "P" switches.
%
%\item{Case insinsitive "seg" and "wrt".
%
%\item{Update install.sh (?).
%
%\item{Allocate tokens in blocks.
%
%\item{Improve "invalid effective address" messages.
%
%
%\xsubsection{cl-0.98.08} Version 0.98.08
%
%\item{Add "\code{%strlen}" and "\code{%substr}" macro operators
%
%\item{Fixed broken c16.mac.
%
%\item{Unterminated string error reported.
%
%\item{Fixed bugs as per 0.98bf
%
%
%\xsubsection{cl-0.98.09b with John Coffman patches released 28-Oct-2001} Version 0.98.09b with John Coffman patches released 28-Oct-2001
%
%Changes from 0.98.07 release to 98.09b as of 28-Oct-2001
%
%\item{More closely compatible with 0.98 when -O0 is implied
%or specified.  Not strictly identical, since backward
%branches in range of short offsets are recognized, and signed
%byte values with no explicit size specification will be
%assembled as a single byte.
%
%\item{More forgiving with the PUSH instruction.  0.98 requires
%a size to be specified always.  0.98.09b will imply the size
%from the current BITS setting (16 or 32).
%
%\item{Changed definition of the optimization flag:
%
%\c    -O0     strict two-pass assembly, JMP and Jcc are
%\c            handled more like 0.98, except that back-
%\c            ward JMPs are short, if possible.
%\c
%\c    -O1     strict two-pass assembly, but forward
%\c            branches are assembled with code guaranteed
%\c            to reach; may produce larger code than
%\c            -O0, but will produce successful assembly
%\c            more often if branch offset sizes are not
%\c            specified.
%\c
%\c    -O2     multi-pass optimization, minimize branch
%\c            offsets; also will minimize signed immed-
%\c            iate bytes, overriding size specification.
%\c
%\c    -O3     like -O2, but more passes taken, if needed
%
%
%\xsubsection{cl-0.98.07 released 01/28/01} Version 0.98.07 released 01/28/01
%
%\item{     Added Stepane Denis' SSE2 instructions to a *working*
%        version of the code - some earlier versions were based on
%        broken code - sorry 'bout that. version "0.98.07"
%
%\item{      Cosmetic modifications to nasm.c, nasm.h,
%        AUTHORS, MODIFIED
%
%
%\xsubsection{cl-0.98.06f released 01/18/01} Version 0.98.06f released 01/18/01
%
%
%\item{Add "metalbrain"s jecxz bug fix in insns.dat
%
%\item{Alter nasmdoc.src to match - version "0.98.06f"
%
%
%\xsubsection{cl-0.98.06e released 01/09/01} Version 0.98.06e released 01/09/01
%
%
%\item{      Removed the "outforms.h" file - it appears to be
%        someone's old backup of "outform.h". version "0.98.06e"
%
%\item{fbk - finally added the fix for the "multiple %includes bug",
%        known since 7/27/99 - reported originally (?) and sent to
%        us by Austin Lunnen - he reports that John Fine had a fix
%        within the day. Here it is...
%
%\item{Nelson Rush resigns from the group. Big thanks to Nelson for
%  his leadership and enthusiasm in getting these changes
%  incorporated into Nasm!
%
%\item{fbk - [list +], [list -] directives - ineptly implemented, should
%        be re-written or removed, perhaps.
%
%\item{Brian Raiter / fbk - "elfso bug" fix - applied to aoutb format
%                       as well - testing might be desirable...
%
%\item{James Seter - -postfix, -prefix command line switches.
%
%\item{Yuri Zaporozhets - rdoff utility changes.
%
%
%\xsubsection{cl-0.98p1} Version 0.98p1
%
%\item{GAS-like palign (Panos Minos)
%
%\item{FIXME: Someone, fill this in with details
%
%
%\xsubsection{cl-0.98bf (bug-fixed)} Version 0.98bf (bug-fixed)
%
%\item{Fixed - elf and aoutb bug - shared libraries
%        - multiple "%include" bug in "-f obj"
%        - jcxz, jecxz bug
%        - unrecognized option bug in ndisasm
%
%\xsubsection{cl-0.98.03 with John Coffman's changes released 27-Jul-2000} Version 0.98.03 with John Coffman's changes released 27-Jul-2000
%
%\item{Added signed byte optimizations for the 0x81/0x83 class
%of instructions: ADC, ADD, AND, CMP, OR, SBB, SUB, XOR:
%when used as 'ADD reg16,imm' or 'ADD reg32,imm.'  Also
%optimization of signed byte form of 'PUSH imm' and 'IMUL
%reg,imm'/'IMUL reg,reg,imm.'  No size specification is needed.
%
%\item{Added multi-pass JMP and Jcc offset optimization.  Offsets
%on forward references will preferentially use the short form,
%without the need to code a specific size (short or near) for
%the branch.  Added instructions for 'Jcc label' to use the
%form 'Jnotcc $+3/JMP label', in cases where a short offset
%is out of bounds.  If compiling for a 386 or higher CPU, then
%the 386 form of Jcc will be used instead.
%
%\> This feature is controlled by a new command-line switch: "O",
%(upper case letter O).  "-O0" reverts the assembler to no
%extra optimization passes, "-O1" allows up to 5 extra passes,
%and "-O2"(default), allows up to 10 extra optimization passes.
%
%\item{Added a new directive:  'cpu XXX', where XXX is any of:
%8086, 186, 286, 386, 486, 586, pentium, 686, PPro, P2, P3 or
%Katmai.  All are case insensitive.  All instructions will
%be selected only if they apply to the selected cpu or lower.
%Corrected a couple of bugs in cpu-dependence in 'insns.dat'.
%
%\item{Added to 'standard.mac', the "use16" and "use32" forms of
%the "bits 16/32" directive. This is nothing new, just conforms
%to a lot of other assemblers. (minor)
%
%\item{Changed label allocation from 320/32 (10000 labels @ 200K+)
%to 32/37 (1000 labels); makes running under DOS much easier.
%Since additional label space is allocated dynamically, this
%should have no effect on large programs with lots of labels.
%The 37 is a prime, believed to be better for hashing. (minor)
%
%
%\xsubsection{cl-0.98.03} Version 0.98.03
%
%"Integrated patchfile 0.98-0.98.01.  I call this version 0.98.03 for
%historical reasons: 0.98.02 was trashed." --John Coffman
%<johninsd@san.rr.com>, 27-Jul-2000
%
%\item{Kendall Bennett's SciTech MGL changes
%
%\item{Note that you must define "TASM_COMPAT" at compile-time
%to get the Tasm Ideal Mode compatibility.
%
%\item{All changes can be compiled in and out using the TASM_COMPAT macros,
%and when compiled without TASM_COMPAT defined we get the exact same
%binary as the unmodified 0.98 sources.
%
%\item{standard.mac, macros.c: Added macros to ignore TASM directives before
%first include
%
%\item{nasm.h: Added extern declaration for tasm_compatible_mode
%
%\item{nasm.c: Added global variable tasm_compatible_mode
%
%\item{Added command line switch for TASM compatible mode (-t)
%
%\item{Changed version command line to reflect when compiled with TASM additions
%
%\item{Added response file processing to allow all arguments on a single
%line (response file is @resp rather than -@resp for NASM format).
%
%\item{labels.c: Changes islocal() macro to support TASM style @@local labels.
%
%\item{Added islocalchar() macro to support TASM style @@local labels.
%
%\item{parser.c: Added support for TASM style memory references (ie: mov
%[DWORD eax],10 rather than the NASM style mov DWORD [eax],10).
%
%\item{preproc.c: Added new directives, \code{%arg}, \code{%local}, \code{%stacksize} to directives
%table
%
%\item{Added support for TASM style directives without a leading % symbol.
%
%\item{Integrated a block of changes from Andrew Zabolotny <bit@eltech.ru>:
%
%\item{A new keyword \code{%xdefine} and its case-insensitive counterpart \code{%ixdefine}.
%They work almost the same way as \code{%define} and \code{%idefine} but expand
%the definition immediately, not on the invocation. Something like a cross
%between \code{%define} and \code{%assign}. The "x" suffix stands for "eXpand", so
%"xdefine" can be deciphered as "expand-and-define". Thus you can do
%things like this:
%
%\c      %assign ofs     0
%\c
%\c      %macro  arg     1
%\c              %xdefine %1 dword [esp+ofs]
%\c              %assign ofs ofs+4
%\c      %endmacro
%
%\item{Changed the place where the expansion of %$name macros are expanded.
%Now they are converted into ..@ctxnum.name form when detokenizing, so
%there are no quirks as before when using %$name arguments to macros,
%in macros etc. For example:
%
%\c      %macro  abc     1
%\c              %define %1 hello
%\c      %endm
%\c
%\c      abc     %$here
%\c      %$here
%
%\>    Now last line will be expanded into "hello" as expected. This also allows
%    for lots of goodies, a good example are extended "proc" macros included
%    in this archive.
%
%\item{Added a check for "cstk" in smacro_defined() before calling get_ctx() -
%    this allows for things like:
%
%\c      %ifdef %$abc
%\c      %endif
%
%\>    to work without warnings even in no context.
%
%\item{Added a check for "cstk" in %if*ctx and %elif*ctx directives -
%    this allows to use \code{%ifctx} without excessive warnings. If there is
%    no active context, \code{%ifctx} goes through "false" branch.
%
%\item{Removed "user error: " prefix with \code{%error} directive: it just clobbers the
%    output and has absolutely no functionality. Besides, this allows to write
%    macros that does not differ from built-in functions in any way.
%
%\item{Added expansion of string that is output by \code{%error} directive. Now you
%    can do things like:
%
%\c      %define hello(x) Hello, x!
%\c
%\c      %define %$name andy
%\c      %error "hello(%$name)"
%
%\> Same happened with \code{%include} directive.
%
%\item{Now all directives that expect an identifier will try to expand and
%    concatenate everything without whitespaces in between before usage.
%    For example, with "unfixed" nasm the commands
%
%\c      %define %$abc hello
%\c      %define __%$abc goodbye
%\c      __%$abc
%
%\>    would produce "incorrect" output: last line will expand to
%
%\c      hello goodbyehello
%
%\>    Not quite what you expected, eh? :-) The answer is that preprocessor
%    treats the \code{%define} construct as if it would be
%
%\c      %define __ %$abc goodbye
%
%\>    (note the white space between __ and %$abc). After my "fix" it
%    will "correctly" expand into
%
%\c      goodbye
%
%\>    as expected. Note that I use quotes around words "correct", "incorrect"
%    etc because this is rather a feature not a bug; however current behaviour
%    is more logical (and allows more advanced macro usage :-).
%
%    Same change was applied to:
%        \code{%push},\code{%macro},\code{%imacro},\code{%define},\code{%idefine},\code{%xdefine},\code{%ixdefine},
%        \code{%assign},\code{%iassign},\code{%undef}
%
%\item{A new directive [WARNING {+|-}warning-id] have been added. It works only
%    if the assembly phase is enabled (i.e. it doesn't work with nasm -e).
%
%\item{A new warning type: macro-selfref. By default this warning is disabled;
%    when enabled NASM warns when a macro self-references itself; for example
%    the following source:
%
%\c        [WARNING macro-selfref]
%\c
%\c        %macro          push    1-*
%\c                %rep    %0
%\c                        push    %1
%\c                        %rotate 1
%\c                %endrep
%\c        %endmacro
%\c
%\c                        push    eax,ebx,ecx
%
%\>  will produce a warning, but if we remove the first line we won't see it
%    anymore (which is The Right Thing To Do {tm} IMHO since C preprocessor
%    eats such constructs without warnings at all).
%
%\item{Added a "error" routine to preprocessor which always will set ERR_PASS1
%    bit in severity_code. This removes annoying repeated errors on first
%    and second passes from preprocessor.
%
%\item{Added the %+ operator in single-line macros for concatenating two
%    identifiers. Usage example:
%
%\c        %define _myfunc _otherfunc
%\c        %define cextern(x) _ %+ x
%\c        cextern (myfunc)
%
%\>    After first expansion, third line will become "_myfunc". After this
%    expansion is performed again so it becomes "_otherunc".
%
%\item{Now if preprocessor is in a non-emitting state, no warning or error
%    will be emitted. Example:
%
%\c        %if 1
%\c                mov     eax,ebx
%\c        %else
%\c                put anything you want between these two brackets,
%\c                even macro-parameter references %1 or local
%\c                labels %$zz or macro-local labels %%zz - no
%\c                warning will be emitted.
%\c        %endif
%
%\item{Context-local variables on expansion as a last resort are looked up
%    in outer contexts. For example, the following piece:
%
%\c        %push   outer
%\c        %define %$a [esp]
%\c
%\c                %push   inner
%\c                %$a
%\c                %pop
%\c        %pop
%
%\>    will expand correctly the fourth line to [esp]; if we'll define another
%    %$a inside the "inner" context, it will take precedence over outer
%    definition. However, this modification has been applied only to
%    expand_smacro and not to smacro_define: as a consequence expansion
%    looks in outer contexts, but \code{%ifdef} won't look in outer contexts.
%
%\>    This behaviour is needed because we don't want nested contexts to
%    act on already defined local macros. Example:
%
%\c        %define %$arg1  [esp+4]
%\c        test    eax,eax
%\c        if      nz
%\c                mov     eax,%$arg1
%\c        endif
%
%\>    In this example the "if" mmacro enters into the "if" context, so %$arg1
%    is not valid anymore inside "if". Of course it could be worked around
%    by using explicitely %$$arg1 but this is ugly IMHO.
%
%\item{Fixed memory leak in \code{%undef}. The origline wasn't freed before
%    exiting on success.
%
%\item{Fixed trap in preprocessor when line expanded to empty set of tokens.
%    This happens, for example, in the following case:
%
%\c        #define SOMETHING
%\c        SOMETHING
%
%
%\xsubsection{cl-0.98} Version 0.98
%
%All changes since NASM 0.98p3 have been produced by H. Peter Anvin <hpa@zytor.com>.
%
%\item{The documentation comment delimiter is \# not #.
%
%\item{Allow EQU definitions to refer to external labels; reported by
%  Pedro Gimeno.
%
%\item{Re-enable support for RDOFF v1; reported by Pedro Gimeno.
%
%\item{Updated License file per OK from Simon and Julian.
%
%
%\xsubsection{cl-0.98p9} Version 0.98p9
%
%\item{Update documentation (although the instruction set reference will
%  have to wait; I don't want to hold up the 0.98 release for it.)
%
%\item{Verified that the NASM implementation of the PEXTRW and PMOVMSKB
%  instructions is correct.  The encoding differs from what the Intel
%  manuals document, but the Pentium III behaviour matches NASM, not
%  the Intel manuals.
%
%\item{Fix handling of implicit sizes in PSHUFW and PINSRW, reported by
%  Stefan Hoffmeister.
%
%\item{Resurrect the -s option, which was removed when changing the
%  diagnostic output to stdout.
%
%
%\xsubsection{cl-0.98p8} Version 0.98p8
%
%\item{Fix for "DB" when NASM is running on a bigendian machine.
%
%\item{Invoke insns.pl once for each output script, making Makefile.in
%  legal for "make -j".
%
%\item{Improve the Unix configure-based makefiles to make package
%  creation easier.
%
%\item{Included an RPM .spec file for building RPM (RedHat Package Manager)
%  packages on Linux or Unix systems.
%
%\item{Fix Makefile dependency problems.
%
%\item{Change src/rdsrc.pl to include sectioning information in info
%  output; required for install-info to work.
%
%\item{Updated the RDOFF distribution to version 2 from Jules; minor
%  massaging to make it compile in my environment.
%
%\item{Split doc files that can be built by anyone with a Perl interpreter off
%  into a separate archive.
%
%\item{"Dress rehearsal" release!
%
%
%\xsubsection{cl-0.98p7} Version 0.98p7
%
%\item{Fixed opcodes with a third byte-sized immediate argument to not
%  complain if given "byte" on the immediate.
%
%\item{Allow \code{%undef} to remove single-line macros with arguments.  This
%  matches the behaviour of #undef in the C preprocessor.
%
%\item{Allow -d, -u, -i and -p to be specified as -D, -U, -I and -P for
%  compatibility with most C compilers and preprocessors.  This allows
%  Makefile options to be shared between cc and nasm, for example.
%
%\item{Minor cleanups.
%
%\item{Went through the list of Katmai instructions and hopefully fixed the
%  (rather few) mistakes in it.
%
%\item{(Hopefully) fixed a number of disassembler bugs related to ambiguous
%  instructions (disambiguated by -p) and SSE instructions with REP.
%
%\item{Fix for bug reported by Mark Junger: "call dword 0x12345678" should
%  work and may add an OSP (affected CALL, JMP, Jcc).
%
%\item{Fix for environments when "stderr" isn't a compile-time constant.
%
%
%\xsubsection{cl-0.98p6} Version 0.98p6
%
%
%\item{Took officially over coordination of the 0.98 release; so drop
%  the p3.x notation. Skipped p4 and p5 to avoid confusion with John
%  Fine's J4 and J5 releases.
%
%\item{Update the documentation; however, it still doesn't include
%  documentation for the various new instructions.  I somehow wonder if
%  it makes sense to have an instruction set reference in the assembler
%  manual when Intel et al have PDF versions of their manuals online.
%
%\item{Recognize "idt" or "centaur" for the -p option to ndisasm.
%
%\item{Changed error messages back to stderr where they belong, but add an
%  -E option to redirect them elsewhere (the DOS shell cannot redirect
%  stderr.)
%
%\item{-M option to generate Makefile dependencies (based on code from Alex
%  Verstak.)
%
%\item{\code{%undef} preprocessor directive, and -u option, that undefines a
%  single-line macro.
%
%\item{OS/2 Makefile (Mkfiles/Makefile.os2) for Borland under OS/2; from
%  Chuck Crayne.
%
%\item{Various minor bugfixes (reported by):
%  - Dangling \code{%s} in preproc.c (Martin Junker)
%
%\item{THERE ARE KNOWN BUGS IN SSE AND THE OTHER KATMAI INSTRUCTIONS.  I am
%  on a trip and didn't bring the Katmai instruction reference, so I
%  can't work on them right now.
%
%\item{Updated the License file per agreement with Simon and Jules to
%  include a GPL distribution clause.
%
%
%\xsubsection{cl-0.98p3.7} Version 0.98p3.7
%
%\item{(Hopefully) fixed the canned Makefiles to include the outrdf2 and
%  zoutieee modules.
%
%\item{Renamed changes.asm to changed.asm.
%
%
%\xsubsection{cl-0.98p3.6} Version 0.98p3.6
%
%\item{Fixed a bunch of instructions that were added in 0.98p3.5 which had
%  memory operands, and the address-size prefix was missing from the
%  instruction pattern.
%
%
%\xsubsection{cl-0.98p3.5} Version 0.98p3.5
%
%\item{Merged in changes from John S. Fine's 0.98-J5 release.  John's based
%  0.98-J5 on my 0.98p3.3 release; this merges the changes.
%
%\item{Expanded the instructions flag field to a long so we can fit more
%  flags; mark SSE (KNI) and AMD or Katmai-specific instructions as
%  such.
%
%\item{Fix the "PRIV" flag on a bunch of instructions, and create new
%  "PROT" flag for protected-mode-only instructions (orthogonal to if
%  the instruction is privileged!) and new "SMM" flag for SMM-only
%  instructions.
%
%\item{Added AMD-only SYSCALL and SYSRET instructions.
%
%\item{Make SSE actually work, and add new Katmai MMX instructions.
%
%\item{Added a -p (preferred vendor) option to ndisasm so that it can
%  distinguish e.g. Cyrix opcodes also used in SSE.  For example:
%
%\c      ndisasm -p cyrix aliased.bin
%\c      00000000  670F514310        paddsiw mm0,[ebx+0x10]
%\c      00000005  670F514320        paddsiw mm0,[ebx+0x20]
%\c      ndisasm -p intel aliased.bin
%\c      00000000  670F514310        sqrtps xmm0,[ebx+0x10]
%\c      00000005  670F514320        sqrtps xmm0,[ebx+0x20]
%
%\item{Added a bunch of Cyrix-specific instructions.
%
%
%\xsubsection{cl-0.98p3.4} Version 0.98p3.4
%
%\item{Made at least an attempt to modify all the additional Makefiles (in
%  the Mkfiles directory).  I can't test it, but this was the best I
%  could do.
%
%\item{DOS DJGPP+"Opus Make" Makefile from John S. Fine.
%
%\item{changes.asm changes from John S. Fine.
%
%
%\xsubsection{cl-0.98p3.3} Version 0.98p3.3
%
%\item{Patch from Conan Brink to allow nesting of \code{%rep} directives.
%
%\item{If we're going to allow INT01 as an alias for INT1/ICEBP (one of
%  Jules 0.98p3 changes), then we should allow INT03 as an alias for INT3
%  as well.
%
%\item{Updated changes.asm to include the latest changes.
%
%\item{Tried to clean up the <CR>s that had snuck in from a DOS/Windows
%  environment into my Unix environment, and try to make sure than
%  DOS/Windows users get them back.
%
%\item{We would silently generate broken tools if insns.dat wasn't sorted
%  properly.  Change insns.pl so that the order doesn't matter.
%
%\item{Fix bug in insns.pl (introduced by me) which would cause conditional
%  instructions to have an extra "cc" in disassembly, e.g. "jnz"
%  disassembled as "jccnz".
%
%
%\xsubsection{cl-0.98p3.2} Version 0.98p3.2
%
%\item{Merged in John S. Fine's changes from his 0.98-J4 prerelease; see
%  http://www.csoft.net/cz/johnfine/
%
%\item{Changed previous "spotless" Makefile target (appropriate for distribution)
%  to "distclean", and added "cleaner" target which is same as "clean"
%  except deletes files generated by Perl scripts; "spotless" is union.
%
%\item{Removed BASIC programs from distribution.  Get a Perl interpreter
%  instead (see below.)
%
%\item{Calling this "pre-release 3.2" rather than "p3-hpa2" because of
%  John's contributions.
%
%\item{Actually link in the IEEE output format (zoutieee.c); fix a bunch of
%  compiler warnings in that file.  Note I don't know what IEEE output
%  is supposed to look like, so these changes were made "blind".
%
%
%\xsubsection{cl-0.98p3-hpa} Version 0.98p3-hpa
%
%\item{Merged nasm098p3.zip with nasm-0.97.tar.gz to create a fully
%  buildable version for Unix systems (Makefile.in updates, etc.)
%
%\item{Changed insns.pl to create the instruction tables in nasm.h and
%  names.c, so that a new instruction can be added by adding it *only*
%  to insns.dat.
%
%\item{Added the following new instructions: SYSENTER, SYSEXIT, FXSAVE,
%  FXRSTOR, UD1, UD2 (the latter two are two opcodes that Intel
%  guarantee will never be used; one of them is documented as UD2 in
%  Intel documentation, the other one just as "Undefined Opcode" --
%  calling it UD1 seemed to make sense.)
%
%\item{MAX_SYMBOL was defined to be 9, but LOADALL286 and LOADALL386 are 10
%  characters long.  Now MAX_SYMBOL is derived from insns.dat.
%
%\item{A note on the BASIC programs included: forget them.  insns.bas is
%  already out of date.  Get yourself a Perl interpreter for your
%  platform of choice at
%  \W{http://www.cpan.org/ports/index.html}{http://www.cpan.org/ports/index.html}.
%
%
%\xsubsection{cl-0.98p3} Version 0.98 pre-release 3
%
%\item{added response file support, improved command line handling, new layout
%help screen
%
%\item{fixed limit checking bug, 'OUT byte nn, reg' bug, and a couple of rdoff
%related bugs, updated Wishlist; 0.98 Prerelease 3.
%
%
%\xsubsection{cl-0.98p2} Version 0.98 pre-release 2
%
%\item{fixed bug in outcoff.c to do with truncating section names longer
%than 8 characters, referencing beyond end of string; 0.98 pre-release 2
%
%
%\xsubsection{cl-0.98p1} Version 0.98 pre-release 1
%
%\item{Fixed a bug whereby STRUC didn't work at all in RDF.
%
%\item{Fixed a problem with group specification in PUBDEFs in OBJ.
%
%\item{Improved ease of adding new output formats. Contribution due to
%Fox Cutter.
%
%\item{Fixed a bug in relocations in the `bin' format: was showing up when
%a relocatable reference crossed an 8192-byte boundary in any output
%section.
%
%\item{Fixed a bug in local labels: local-label lookups were inconsistent
%between passes one and two if an EQU occurred between the definition
%of a global label and the subsequent use of a local label local to
%that global.
%
%\item{Fixed a seg-fault in the preprocessor (again) which happened when
%you use a blank line as the first line of a multi-line macro
%definition and then defined a label on the same line as a call to
%that macro.
%
%\item{Fixed a stale-pointer bug in the handling of the NASM environment
%variable. Thanks to Thomas McWilliams.
%
%\item{ELF had a hard limit on the number of sections which caused
%segfaults when transgressed. Fixed.
%
%\item{Added ability for ndisasm to read from stdin by using `-' as the
%filename.
%
%\item{ndisasm wasn't outputting the TO keyword. Fixed.
%
%\item{Fixed error cascade on bogus expression in \code{%if} - an error in
%evaluation was causing the entire \code{%if} to be discarded, thus creating
%trouble later when the \code{%else} or \code{%endif} was encountered.
%
%\item{Forward reference tracking was instruction-granular not operand-
%granular, which was causing 286-specific code to be generated
%needlessly on code of the form `shr word [forwardref],1'. Thanks to
%Jim Hague for sending a patch.
%
%\item{All messages now appear on stdout, as sending them to stderr serves
%no useful purpose other than to make redirection difficult.
%
%\item{Fixed the problem with EQUs pointing to an external symbol - this
%now generates an error message.
%
%\item{Allowed multiple size prefixes to an operand, of which only the first
%is taken into account.
%
%\item{Incorporated John Fine's changes, including fixes of a large number
%of preprocessor bugs, some small problems in OBJ, and a reworking of
%label handling to define labels before their line is assembled, rather
%than after.
%
%\item{Reformatted a lot of the source code to be more readable. Included
%'coding.txt' as a guideline for how to format code for contributors.
%
%\item{Stopped nested \code{%reps} causing a panic - they now cause a slightly more
%friendly error message instead.
%
%\item{Fixed floating point constant problems (patch by Pedro Gimeno)
%
%\item{Fixed the return value of insn_size() not being checked for -1, indicating
%an error.
%
%\item{Incorporated 3Dnow! instructions.
%
%\item{Fixed the 'mov eax, eax + ebx' bug.
%
%\item{Fixed the GLOBAL EQU bug in ELF. Released developers release 3.
%
%\item{Incorporated John Fine's command line parsing changes
%
%\item{Incorporated David Lindauer's OMF debug support
%
%\item{Made changes for LCC 4.0 support (\code{__NASM_CDecl__}, removed register size
%specification warning when sizes agree).
%
%
%\xchapter{cl-0.9x} NASM 0.9 Series
%
%Revisions before 0.98.
%
%\xsubsection{cl-0.97} Version 0.97 released December 1997
%
%\item{This was entirely a bug-fix release to 0.96, which seems to have got
%cursed. Silly me.
%
%\item{Fixed stupid mistake in OBJ which caused `MOV EAX,<constant>' to
%fail. Caused by an error in the `MOV EAX,<segment>' support.
%
%\item{ndisasm hung at EOF when compiled with lcc on Linux because lcc on
%Linux somehow breaks feof(). ndisasm now does not rely on feof().
%
%\item{A heading in the documentation was missing due to a markup error in
%the indexing. Fixed.
%
%\item{Fixed failure to update all pointers on realloc() within extended-
%operand code in parser.c. Was causing wrong behaviour and seg faults
%on lines such as `dd 0.0,0.0,0.0,0.0,...'
%
%\item{Fixed a subtle preprocessor bug whereby invoking one multi-line
%macro on the first line of the expansion of another, when the second
%had been invoked with a label defined before it, didn't expand the
%inner macro.
%
%\item{Added internal.doc back in to the distribution archives - it was
%missing in 0.96 *blush*
%
%\item{Fixed bug causing 0.96 to be unable to assemble its own test files,
%specifically objtest.asm. *blush again*
%
%\item{Fixed seg-faults and bogus error messages caused by mismatching
%\code{%rep} and \code{%endrep} within multi-line macro definitions.
%
%\item{Fixed a problem with buffer overrun in OBJ, which was causing
%corruption at ends of long PUBDEF records.
%
%\item{Separated DOS archives into main-program and documentation to reduce
%download size.
%
%
%\xsubsection{cl-0.96} Version 0.96 released November 1997
%
%\item{Fixed a bug whereby, if `nasm sourcefile' would cause a filename
%collision warning and put output into `nasm.out', then `nasm
%sourcefile -o outputfile' still gave the warning even though the
%`-o' was honoured.
%Fixed name pollution under Digital UNIX: one of its header files
%defined R_SP, which broke the enum in nasm.h.
%
%\item{Fixed minor instruction table problems: FUCOM and FUCOMP didn't have
%two-operand forms; NDISASM didn't recognise the longer register
%forms of PUSH and POP (eg FF F3 for PUSH BX); TEST mem,imm32 was
%flagged as undocumented; the 32-bit forms of CMOV had 16-bit operand
%size prefixes; `AAD imm' and `AAM imm' are no longer flagged as
%undocumented because the Intel Architecture reference documents
%them.
%
%\item{Fixed a problem with the local-label mechanism, whereby strange
%types of symbol (EQUs, auto-defined OBJ segment base symbols)
%interfered with the `previous global label' value and screwed up
%local labels.
%
%\item{Fixed a bug whereby the stub preprocessor didn't communicate with
%the listing file generator, so that the -a and -l options in
%conjunction would produce a useless listing file.
%
%\item{Merged `os2' object file format back into `obj', after discovering
%that `obj' _also_ shouldn't have a link pass separator in a module
%containing a non-trivial MODEND. Flat segments are now declared
%using the FLAT attribute. `os2' is no longer a valid object format
%name: use `obj'.
%
%\item{Removed the fixed-size temporary storage in the evaluator. Very very
%long expressions (like `mov ax,1+1+1+1+...' for two hundred 1s or
%so) should now no longer crash NASM.
%
%\item{Fixed a bug involving segfaults on disassembly of MMX instructions,
%by changing the meaning of one of the operand-type flags in nasm.h.
%This may cause other apparently unrelated MMX problems; it needs to
%be tested thoroughly.
%
%\item{Fixed some buffer overrun problems with large OBJ output files.
%Thanks to DJ Delorie for the bug report and fix.
%
%\item{Made preprocess-only mode actually listen to the \code{%line} markers as it
%prints them, so that it can report errors more sanely.
%
%\item{Re-designed the evaluator to keep more sensible track of expressions
%involving forward references: can now cope with previously-nightmare
%situations such as:
%
%\c   mov ax,foo | bar
%\c   foo equ 1
%\c   bar equ 2
%
%\item{Added the ALIGN and ALIGNB standard macros.
%
%\item{Added PIC support in ELF: use of WRT to obtain the four extra
%relocation types needed.
%
%\item{Added the ability for output file formats to define their own
%extensions to the GLOBAL, COMMON and EXTERN directives.
%
%\item{Implemented common-variable alignment, and global-symbol type and
%size declarations, in ELF.
%
%\item{Implemented NEAR and FAR keywords for common variables, plus
%far-common element size specification, in OBJ.
%
%\item{Added a feature whereby EXTERNs and COMMONs in OBJ can be given a
%default WRT specification (either a segment or a group).
%
%\item{Transformed the Unix NASM archive into an auto-configuring package.
%
%\item{Added a sanity-check for people applying SEG to things which are
%already segment bases: this previously went unnoticed by the SEG
%processing and caused OBJ-driver panics later.
%
%\item{Added the ability, in OBJ format, to deal with `MOV EAX,<segment>'
%type references: OBJ doesn't directly support dword-size segment
%base fixups, but as long as the low two bytes of the constant term
%are zero, a word-size fixup can be generated instead and it will
%work.
%
%\item{Added the ability to specify sections' alignment requirements in
%Win32 object files and pure binary files.
%
%\item{Added preprocess-time expression evaluation: the \code{%assign} (and
%\code{%iassign}) directive and the bare \code{%if} (and \code{%elif}) conditional. Added
%relational operators to the evaluator, for use only in \code{%if}
%constructs: the standard relationals = < > <= >= <> (and C-like
%synonyms == and !=) plus low-precedence logical operators &&, ^^ and
%||.
%
%\item{Added a preprocessor repeat construct: \code{%rep} / \code{%exitrep} / \code{%endrep}.
%
%\item{Added the __FILE__ and __LINE__ standard macros.
%
%\item{Added a sanity check for number constants being greater than
%0xFFFFFFFF. The warning can be disabled.
%
%\item{Added the %0 token whereby a variadic multi-line macro can tell how
%many parameters it's been given in a specific invocation.
%
%\item{Added \code{%rotate}, allowing multi-line macro parameters to be cycled.
%
%\item{Added the `*' option for the maximum parameter count on multi-line
%macros, allowing them to take arbitrarily many parameters.
%
%\item{Added the ability for the user-level forms of EXTERN, GLOBAL and
%COMMON to take more than one argument.
%
%\item{Added the IMPORT and EXPORT directives in OBJ format, to deal with
%Windows DLLs.
%
%\item{Added some more preprocessor \code{%if} constructs: \code{%ifidn} / \code{%ifidni} (exact
%textual identity), and \code{%ifid} / \code{%ifnum} / \code{%ifstr} (token type testing).
%
%\item{Added the ability to distinguish SHL AX,1 (the 8086 version) from
%SHL AX,BYTE 1 (the 286-and-upwards version whose constant happens to
%be 1).
%
%\item{Added NetBSD/FreeBSD/OpenBSD's variant of a.out format, complete
%with PIC shared library features.
%
%\item{Changed NASM's idiosyncratic handling of FCLEX, FDISI, FENI, FINIT,
%FSAVE, FSTCW, FSTENV, and FSTSW to bring it into line with the
%otherwise accepted standard. The previous behaviour, though it was a
%deliberate feature, was a deliberate feature based on a
%misunderstanding. Apologies for the inconvenience.
%
%\item{Improved the flexibility of ABSOLUTE: you can now give it an
%expression rather than being restricted to a constant, and it can
%take relocatable arguments as well.
%
%\item{Added the ability for a variable to be declared as EXTERN multiple
%times, and the subsequent definitions are just ignored.
%
%\item{We now allow instruction prefixes (CS, DS, LOCK, REPZ etc) to be
%alone on a line (without a following instruction).
%
%\item{Improved sanity checks on whether the arguments to EXTERN, GLOBAL
%and COMMON are valid identifiers.
%
%\item{Added misc/exebin.mac to allow direct generation of .EXE files by
%hacking up an EXE header using DB and DW; also added test/binexe.asm
%to demonstrate the use of this. Thanks to Yann Guidon for
%contributing the EXE header code.
%
%\item{ndisasm forgot to check whether the input file had been successfully
%opened. Now it does. Doh!
%
%\item{Added the Cyrix extensions to the MMX instruction set.
%
%\item{Added a hinting mechanism to allow [EAX+EBX] and [EBX+EAX] to be
%assembled differently. This is important since [ESI+EBP] and
%[EBP+ESI] have different default base segment registers.
%
%\item{Added support for the PharLap OMF extension for 4096-byte segment
%alignment.
%
%
%\xsubsection{cl-0.95 released July 1997} Version 0.95 released July 1997
%
%\item{Fixed yet another ELF bug. This one manifested if the user relied on
%the default segment, and attempted to define global symbols without
%first explicitly declaring the target segment.
%
%\item{Added makefiles (for NASM and the RDF tools) to build Win32 console
%apps under Symantec C++. Donated by Mark Junker.
%
%\item{Added `macros.bas' and `insns.bas', QBasic versions of the Perl
%scripts that convert `standard.mac' to `macros.c' and convert
%`insns.dat' to `insnsa.c' and `insnsd.c'. Also thanks to Mark
%Junker.
%
%\item{Changed the diassembled forms of the conditional instructions so
%that JB is now emitted as JC, and other similar changes. Suggested
%list by Ulrich Doewich.
%
%\item{Added `@' to the list of valid characters to begin an identifier
%with.
%
%\item{Documentary changes, notably the addition of the `Common Problems'
%section in nasm.doc.
%
%\item{Fixed a bug relating to 32-bit PC-relative fixups in OBJ.
%
%\item{Fixed a bug in perm_copy() in labels.c which was causing exceptions
%in cleanup_labels() on some systems.
%
%\item{Positivity sanity check in TIMES argument changed from a warning to
%an error following a further complaint.
%
%\item{Changed the acceptable limits on byte and word operands to allow
%things like `~10111001b' to work.
%
%\item{Fixed a major problem in the preprocessor which caused seg-faults if
%macro definitions contained blank lines or comment-only lines.
%
%\item{Fixed inadequate error checking on the commas separating the
%arguments to `db', `dw' etc.
%
%\item{Fixed a crippling bug in the handling of macros with operand counts
%defined with a `+' modifier.
%
%\item{Fixed a bug whereby object file formats which stored the input file
%name in the output file (such as OBJ and COFF) weren't doing so
%correctly when the output file name was specified on the command
%line.
%
%\item{Removed [INC] and [INCLUDE] support for good, since they were
%obsolete anyway.
%
%\item{Fixed a bug in OBJ which caused all fixups to be output in 16-bit
%(old-format) FIXUPP records, rather than putting the 32-bit ones in
%FIXUPP32 (new-format) records.
%
%\item{Added, tentatively, OS/2 object file support (as a minor variant on
%OBJ).
%
%\item{Updates to Fox Cutter's Borland C makefile, Makefile.bc2.
%
%\item{Removed a spurious second fclose() on the output file.
%
%\item{Added the `-s' command line option to redirect all messages which
%would go to stderr (errors, help text) to stdout instead.
%
%\item{Added the `-w' command line option to selectively suppress some
%classes of assembly warning messages.
%
%\item{Added the `-p' pre-include and `-d' pre-define command-line options.
%
%\item{Added an include file search path: the `-i' command line option.
%
%\item{Fixed a silly little preprocessor bug whereby starting a line with a
%`%!' environment-variable reference caused an `unknown directive'
%error.
%
%\item{Added the long-awaited listing file support: the `-l' command line
%option.
%
%\item{Fixed a problem with OBJ format whereby, in the absence of any
%explicit segment definition, non-global symbols declared in the
%implicit default segment generated spurious EXTDEF records in the
%output.
%
%\item{Added the NASM environment variable.
%
%\item{From this version forward, Win32 console-mode binaries will be
%included in the DOS distribution in addition to the 16-bit binaries.
%Added Makefile.vc for this purpose.
%
%\item{Added `return 0;' to test/objlink.c to prevent compiler warnings.
%
%\item{Added the __NASM_MAJOR__ and __NASM_MINOR__ standard defines.
%
%\item{Added an alternative memory-reference syntax in which prefixing an
%operand with `&' is equivalent to enclosing it in square brackets,
%at the request of Fox Cutter.
%
%\item{Errors in pass two now cause the program to return a non-zero error
%code, which they didn't before.
%
%\item{Fixed the single-line macro cycle detection, which didn't work at
%all on macros with no parameters (caused an infinite loop). Also
%changed the behaviour of single-line macro cycle detection to work
%like cpp, so that macros like `extrn' as given in the documentation
%can be implemented.
%
%\item{Fixed the implementation of WRT, which was too restrictive in that
%you couldn't do `mov ax,[di+abc wrt dgroup]' because (di+abc) wasn't
%a relocatable reference.
%
%
%\xsubsection{cl-0.94 released April 1997} Version 0.94 released April 1997
%
%
%\item{Major item: added the macro processor.
%
%\item{Added undocumented instructions SMI, IBTS, XBTS and LOADALL286. Also
%reorganised CMPXCHG instruction into early-486 and Pentium forms.
%Thanks to Thobias Jones for the information.
%
%\item{Fixed two more stupid bugs in ELF, which were causing `ld' to
%continue to seg-fault in a lot of non-trivial cases.
%
%\item{Fixed a seg-fault in the label manager.
%
%\item{Stopped FBLD and FBSTP from _requiring_ the TWORD keyword, which is
%the only option for BCD loads/stores in any case.
%
%\item{Ensured FLDCW, FSTCW and FSTSW can cope with the WORD keyword, if
%anyone bothers to provide it. Previously they complained unless no
%keyword at all was present.
%
%\item{Some forms of FDIV/FDIVR and FSUB/FSUBR were still inverted: a
%vestige of a bug that I thought had been fixed in 0.92. This was
%fixed, hopefully for good this time...
%
%\item{Another minor phase error (insofar as a phase error can _ever_ be
%minor) fixed, this one occurring in code of the form
%
%\c   rol ax,forward_reference
%\c   forward_reference equ 1
%
%\item{The number supplied to TIMES is now sanity-checked for positivity,
%and also may be greater than 64K (which previously didn't work on
%16-bit systems).
%
%\item{Added Watcom C makefiles, and misc/pmw.bat, donated by Dominik Behr.
%
%\item{Added the INCBIN pseudo-opcode.
%
%\item{Due to the advent of the preprocessor, the [INCLUDE] and [INC]
%directives have become obsolete. They are still supported in this
%version, with a warning, but won't be in the next.
%
%\item{Fixed a bug in OBJ format, which caused incorrect object records to
%be output when absolute labels were made global.
%
%\item{Updates to RDOFF subdirectory, and changes to outrdf.c.
%
%
%\xsubsection{cl-0.93 released January 1997} Version 0.93 released January 1997
%
%This release went out in a great hurry after semi-crippling bugs
%were found in 0.92.
%
%\item{Really \e{did} fix the stack overflows this time. *blush*
%
%\item{Had problems with EA instruction sizes changing between passes, when
%an offset contained a forward reference and so 4 bytes were
%allocated for the offset in pass one; by pass two the symbol had
%been defined and happened to be a small absolute value, so only 1
%byte got allocated, causing instruction size mismatch between passes
%and hence incorrect address calculations. Fixed.
%
%\item{Stupid bug in the revised ELF section generation fixed (associated
%string-table section for .symtab was hard-coded as 7, even when this
%didn't fit with the real section table). Was causing `ld' to
%seg-fault under Linux.
%
%\item{Included a new Borland C makefile, Makefile.bc2, donated by Fox
%Cutter <lmb@comtch.iea.com>.
%
%
%\xsubsection{cl-0.92 released January 1997} Version 0.92 released January 1997
%
%\item{The FDIVP/FDIVRP and FSUBP/FSUBRP pairs had been inverted: this was
%fixed. This also affected the LCC driver.
%
%\item{Fixed a bug regarding 32-bit effective addresses of the form
%\code{[other_register+ESP]}.
%
%\item{Documentary changes, notably documentation of the fact that Borland
%Win32 compilers use `obj' rather than `win32' object format.
%
%\item{Fixed the COMENT record in OBJ files, which was formatted
%incorrectly.
%
%\item{Fixed a bug causing segfaults in large RDF files.
%
%\item{OBJ format now strips initial periods from segment and group
%definitions, in order to avoid complications with the local label
%syntax.
%
%\item{Fixed a bug in disassembling far calls and jumps in NDISASM.
%
%\item{Added support for user-defined sections in COFF and ELF files.
%
%\item{Compiled the DOS binaries with a sensible amount of stack, to
%prevent stack overflows on any arithmetic expression containing
%parentheses.
%
%\item{Fixed a bug in handling of files that do not terminate in a newline.
%
%
%\xsubsection{cl-0.91 released November 1996} Version 0.91 released November 1996
%
%\item{Loads of bug fixes.
%
%\item{Support for RDF added.
%
%\item{Support for DBG debugging format added.
%
%\item{Support for 32-bit extensions to Microsoft OBJ format added.
%
%\item{Revised for Borland C: some variable names changed, makefile added.
%
%\item{LCC support revised to actually work.
%
%\item{JMP/CALL NEAR/FAR notation added.
%
%\item{`a16', `o16', `a32' and `o32' prefixes added.
%
%\item{Range checking on short jumps implemented.
%
%\item{MMX instruction support added.
%
%\item{Negative floating point constant support added.
%
%\item{Memory handling improved to bypass 64K barrier under DOS.
%
%\item{\code{$} prefix to force treatment of reserved words as identifiers added.
%
%\item{Default-size mechanism for object formats added.
%
%\item{Compile-time configurability added.
%
%\item{\code{#}, \code{@}, \code{~} and c\{?} are now valid characters in labels.
%
%\item{\code{-e} and \code{-k} options in NDISASM added.
%
%\xsubsection{cl-0.90 released October 1996} Version 0.90 released October 1996
%
%First release version. First support for object file output. Other
%changes from previous version (0.3x) too numerous to document.
