%
% vim: ts=4 sw=4 et
%
\xchapter{source}{Building NASM from Source}

The source code for NASM is available from our website,
\href{http://www.nasm.us/}{http://wwww.nasm.us/},
see \fullref{sec:website}.

\xsection{tarball}{Building from a Source Archive}

The source archives available on the web site should be capable of
building on a number of platforms.  This is the recommended method for
building NASM to support platforms for which executables are not
available.

On a system which has Unix shell (\code{sh}), run:

\begin{lstlisting}
sh configure
make everything
\end{lstlisting}

A number of options can be passed to \code{configure}; see
\code{sh configure --help}.

A set of Makefiles for some other environments are also available;
please see the file \code{Mkfiles/README}.

To build the installer for the Windows platform, you will need the
Nullsoft Scriptable Installer, \textindex{NSIS}, installed.

To build the documentation, you will need a set of additional tools.
The documentation is not likely to be able to build on non-Unix
systems.

\xsection{git}{Building from the \codeindex{git} Repository}

The NASM development tree is kept in a source code repository using
the \code{git} distributed source control system.  The link is available
on the website.  This is recommended only to participate in the
development of NASM or to assist with testing the development code.

To build NASM from the \code{git} repository you will need a Perl and, if
building on a Unix system, GNU autoconf.

To build on a Unix system, run:

\begin{lstlisting}
sh autogen.sh
\end{lstlisting}

to create the \code{configure} script and then build as listed above.
