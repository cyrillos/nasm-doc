%
% vim: ts=4 sw=4 et
%
\xchapter{contact}{Contact Information}

\xsection{website}{Website}

NASM has a \textindex{website} at \href{http://www.nasm.us/}{http://www.nasm.us/}.

\textindexlc{New releases}, \textindex{release candidates}, and
\index{snapshots!daily development}\textindex{daily development snapshots}
of NASM are available from the official web site in source form as well
as binaries for a number of common platforms.

\xsubsection{forums}{User Forums}

Users of NASM may find the Forums on the website useful. These are,
however, not frequented much by the developers of NASM, so they are
not suitable for reporting bugs.

\xsubsection{develcom}{Development Community}

The development of NASM is coordinated primarily though the
\codeindex{nasm-devel} mailing list. If you wish to participate in
development of NASM, please join this mailing list. Subscription
links and archives of past posts are available on the website.

\xsection{bugs}{Reporting Bugs}
\index{bugs}

To report bugs in NASM, please use the \textindex{bug tracker} at
\href{http://www.nasm.us/}{http://www.nasm.us/} (click on "Bug Tracker"),
or if that fails then through one of the contacts in \nref{website}.

Please read \nref{qstart} first, and don't report the bug if it's
listed in there as a deliberate feature. (If you think the feature
is badly thought out, feel free to send us reasons why you think it
should be changed, but don't just send us mail saying `This is a
bug' if the documentation says we did it on purpose.) Then read
\nref{problems}, and don't bother reporting the bug if it's
listed there.

If you do report a bug, \emph{please} make sure your bug report includes
the following information:

\begin{itemize}
    \item{What operating system you're running NASM under. Linux,
        FreeBSD, NetBSD, MacOS X, Win16, Win32, Win64, MS-DOS, OS/2, VMS,
        whatever.}

    \item{If you compiled your own executable from a source archive, compiled
        your own executable from \code{git}, used the standard distribution
        binaries from the website, or got an executable from somewhere else
        (e.g. a Linux distribution.) If you were using a locally built
        executable, try to reproduce the problem using one of the standard
        binaries, as this will make it easier for us to reproduce your problem
        prior to fixing it.}

    \item{Which version of NASM you're using, and exactly how you invoked
        it. Give us the precise command line, and the contents of the
        \code{NASMENV} environment variable if any.}

    \item{Which versions of any supplementary programs you're using, and
        how you invoked them. If the problem only becomes visible at link
        time, tell us what linker you're using, what version of it you've
        got, and the exact linker command line. If the problem involves
        linking against object files generated by a compiler, tell us what
        compiler, what version, and what command line or options you used.
        (If you're compiling in an IDE, please try to reproduce the problem
        with the command-line version of the compiler.)}

    \item{If at all possible, send us a NASM source file which exhibits the
        problem. If this causes copyright problems (e.g. you can only
        reproduce the bug in restricted-distribution code) then bear in mind
        the following two points: firstly, we guarantee that any source code
        sent to us for the purposes of debugging NASM will be used \emph{only}
        for the purposes of debugging NASM, and that we will delete all our
        copies of it as soon as we have found and fixed the bug or bugs in
        question; and secondly, we would prefer \emph{not} to be mailed large
        chunks of code anyway. The smaller the file, the better. A
        three-line sample file that does nothing useful \emph{except}
        demonstrate the problem is much easier to work with than a
        fully fledged ten-thousand-line program. (Of course, some errors
        \emph{do} only crop up in large files, so this may not be possible.)}

    \item{A description of what the problem actually \emph{is}. `It doesn't
        work' is \emph{not} a helpful description! Please describe exactly what
        is happening that shouldn't be, or what isn't happening that should.
        Examples might be: `NASM generates an error message saying Line 3
        for an error that's actually on Line 5'; `NASM generates an error
        message that I believe it shouldn't be generating at all'; `NASM
        fails to generate an error message that I believe it \emph{should} be
        generating'; `the object file produced from this source code crashes
        my linker'; `the ninth byte of the output file is 66 and I think it
        should be 77 instead'.}

    \item{If you believe the output file from NASM to be faulty, send it to
        us. That allows us to determine whether our own copy of NASM
        generates the same file, or whether the problem is related to
        portability issues between our development platforms and yours. We
        can handle binary files mailed to us as MIME attachments, uuencoded,
        and even BinHex. Alternatively, we may be able to provide an FTP
        site you can upload the suspect files to; but mailing them is easier
        for us.}

    \item{Any other information or data files that might be helpful. If,
        for example, the problem involves NASM failing to generate an object
        file while TASM can generate an equivalent file without trouble,
        then send us \emph{both} object files, so we can see what TASM is doing
        differently from us.}
\end{itemize}
