\chapter{\textindexlc{Standard Macros}}
\label{ch:stdmac}

NASM defines a set of standard macros, which are already defined
when it starts to process any source file. If you really need a
program to be assembled with no pre-defined macros, you can use the
\codeindex{\%clear} directive to empty the preprocessor of everything
but context-local preprocessor variables and single-line macros.

Most \textindex{user-level assembler directives} are implemented as macros
which invoke primitive directives; these are described in \fullref{sec:directive}.
The rest of the standard macro set is described here.

\section{\textindex{NASM Version} Macros}
\label{sec:stdmacver}

The single-line macros \codeindex{\_\_NASM\_MAJOR\_\_}, \codeindex{\_\_NASM\_MINOR\_\_},
\codeindex{\_\_NASM\_SUBMINOR\_\_} and \codeindex{\_\_NASM\_PATCHLEVEL\_\_} expand to
the major, minor, subminor and patch level parts of the \textindex{version number of NASM} being used. So, under NASM 0.98.32p1 for example, \code{\_\_NASM\_MAJOR\_\_}
would be defined to be 0, \code{\_\_NASM\_MINOR\_\_} would be defined as 98,
\code{\_\_NASM\_SUBMINOR\_\_} would be defined to 32, and \code{\_\_NASM\_PATCHLEVEL\_\_}
would be defined as 1.

Additionally, the macro \codeindex{\_\_NASM\_SNAPSHOT\_\_} is defined for
automatically generated snapshot releases \emph{only}.

\section{\codeindex{\_\_NASM\_VERSION\_ID\_\_}: \textindex{NASM Version ID}}
\label{sec:stdmacverid}

The single-line macro \code{\_\_NASM\_VERSION\_ID\_\_} expands to a dword integer
representing the full version number of the version of nasm being used.
The value is the equivalent to \code{\_\_NASM\_MAJOR\_\_}, \code{\_\_NASM\_MINOR\_\_},
\code{\_\_NASM\_SUBMINOR\_\_} and \code{\_\_NASM\_PATCHLEVEL\_\_} concatenated to
produce a single doubleword. Hence, for 0.98.32p1, the returned number
would be equivalent to:

\begin{lstlisting}
dd      0x00622001
\end{lstlisting}

or

\begin{lstlisting}
db      1,32,98,0
\end{lstlisting}

Note that the above lines are generate exactly the same code, the second
line is used just to give an indication of the order that the separate
values will be present in memory.


\section{\codeindex{\_\_NASM\_VER\_\_}: \textindex{NASM Version string}}
\label{sec:stdmacverstr}

The single-line macro \code{\_\_NASM\_VER\_\_} expands to a string which defines
the version number of nasm being used. So, under NASM 0.98.32 for example,

\begin{lstlisting}
db      __NASM_VER__
\end{lstlisting}

would expand to

\begin{lstlisting}
db      "0.98.32"
\end{lstlisting}
%
%
%\S{fileline} \i\c{__FILE__} and \i\c{__LINE__}: File Name and Line Number
%
%Like the C preprocessor, NASM allows the user to find out the file
%name and line number containing the current instruction. The macro
%\c{__FILE__} expands to a string constant giving the name of the
%current input file (which may change through the course of assembly
%if \c{%include} directives are used), and \c{__LINE__} expands to a
%numeric constant giving the current line number in the input file.
%
%These macros could be used, for example, to communicate debugging
%information to a macro, since invoking \c{__LINE__} inside a macro
%definition (either single-line or multi-line) will return the line
%number of the macro \e{call}, rather than \e{definition}. So to
%determine where in a piece of code a crash is occurring, for
%example, one could write a routine \c{stillhere}, which is passed a
%line number in \c{EAX} and outputs something like `line 155: still
%here'. You could then write a macro
%
%\c %macro  notdeadyet 0
%\c
%\c         push    eax
%\c         mov     eax,__LINE__
%\c         call    stillhere
%\c         pop     eax
%\c
%\c %endmacro
%
%and then pepper your code with calls to \c{notdeadyet} until you
%find the crash point.
%
%
%\S{bitsm} \i\c{__BITS__}: Current BITS Mode
%
%The \c{__BITS__} standard macro is updated every time that the BITS mode is
%set using the \c{BITS XX} or \c{[BITS XX]} directive, where XX is a valid mode
%number of 16, 32 or 64. \c{__BITS__} receives the specified mode number and
%makes it globally available. This can be very useful for those who utilize
%mode-dependent macros.
%
%\S{ofmtm} \i\c{__OUTPUT_FORMAT__}: Current Output Format
%
%The \c{__OUTPUT_FORMAT__} standard macro holds the current Output Format,
%as given by the \c{-f} option or NASM's default. Type \c{nasm -hf} for a
%list.
%
%\c %ifidn __OUTPUT_FORMAT__, win32
%\c  %define NEWLINE 13, 10
%\c %elifidn __OUTPUT_FORMAT__, elf32
%\c  %define NEWLINE 10
%\c %endif
%
%
%\S{datetime} Assembly Date and Time Macros
%
%NASM provides a variety of macros that represent the timestamp of the
%assembly session.
%
%\b The \i\c{__DATE__} and \i\c{__TIME__} macros give the assembly date and
%time as strings, in ISO 8601 format (\c{"YYYY-MM-DD"} and \c{"HH:MM:SS"},
%respectively.)
%
%\b The \i\c{__DATE_NUM__} and \i\c{__TIME_NUM__} macros give the assembly
%date and time in numeric form; in the format \c{YYYYMMDD} and
%\c{HHMMSS} respectively.
%
%\b The \i\c{__UTC_DATE__} and \i\c{__UTC_TIME__} macros give the assembly
%date and time in universal time (UTC) as strings, in ISO 8601 format
%(\c{"YYYY-MM-DD"} and \c{"HH:MM:SS"}, respectively.)  If the host
%platform doesn't provide UTC time, these macros are undefined.
%
%\b The \i\c{__UTC_DATE_NUM__} and \i\c{__UTC_TIME_NUM__} macros give the
%assembly date and time universal time (UTC) in numeric form; in the
%format \c{YYYYMMDD} and \c{HHMMSS} respectively.  If the
%host platform doesn't provide UTC time, these macros are
%undefined.
%
%\b The \c{__POSIX_TIME__} macro is defined as a number containing the
%number of seconds since the POSIX epoch, 1 January 1970 00:00:00 UTC;
%excluding any leap seconds.  This is computed using UTC time if
%available on the host platform, otherwise it is computed using the
%local time as if it was UTC.
%
%All instances of time and date macros in the same assembly session
%produce consistent output.  For example, in an assembly session
%started at 42 seconds after midnight on January 1, 2010 in Moscow
%(timezone UTC+3) these macros would have the following values,
%assuming, of course, a properly configured environment with a correct
%clock:
%
%\c       __DATE__             "2010-01-01"
%\c       __TIME__             "00:00:42"
%\c       __DATE_NUM__         20100101
%\c       __TIME_NUM__         000042
%\c       __UTC_DATE__         "2009-12-31"
%\c       __UTC_TIME__         "21:00:42"
%\c       __UTC_DATE_NUM__     20091231
%\c       __UTC_TIME_NUM__     210042
%\c       __POSIX_TIME__       1262293242
%
%
%\S{use_def} \I\c{__USE_*__}\c{__USE_}\e{package}\c{__}: Package
%Include Test
%
%When a standard macro package (see \k{macropkg}) is included with the
%\c{%use} directive (see \k{use}), a single-line macro of the form
%\c{__USE_}\e{package}\c{__} is automatically defined.  This allows
%testing if a particular package is invoked or not.
%
%For example, if the \c{altreg} package is included (see
%\k{pkg_altreg}), then the macro \c{__USE_ALTREG__} is defined.
%
%
%\S{pass_macro} \i\c{__PASS__}: Assembly Pass
%
%The macro \c{__PASS__} is defined to be \c{1} on preparatory passes,
%and \c{2} on the final pass.  In preprocess-only mode, it is set to
%\c{3}, and when running only to generate dependencies (due to the
%\c{-M} or \c{-MG} option, see \k{opt-M}) it is set to \c{0}.
%
%\e{Avoid using this macro if at all possible.  It is tremendously easy
%to generate very strange errors by misusing it, and the semantics may
%change in future versions of NASM.}
%
%
%\S{struc} \i\c{STRUC} and \i\c{ENDSTRUC}: \i{Declaring Structure} Data Types
%
%The core of NASM contains no intrinsic means of defining data
%structures; instead, the preprocessor is sufficiently powerful that
%data structures can be implemented as a set of macros. The macros
%\c{STRUC} and \c{ENDSTRUC} are used to define a structure data type.
%
%\c{STRUC} takes one or two parameters. The first parameter is the name
%of the data type. The second, optional parameter is the base offset of
%the structure. The name of the data type is defined as a symbol with
%the value of the base offset, and the name of the data type with the
%suffix \c{_size} appended to it is defined as an \c{EQU} giving the
%size of the structure. Once \c{STRUC} has been issued, you are
%defining the structure, and should define fields using the \c{RESB}
%family of pseudo-instructions, and then invoke \c{ENDSTRUC} to finish
%the definition.
%
%For example, to define a structure called \c{mytype} containing a
%longword, a word, a byte and a string of bytes, you might code
%
%\c struc   mytype
%\c
%\c   mt_long:      resd    1
%\c   mt_word:      resw    1
%\c   mt_byte:      resb    1
%\c   mt_str:       resb    32
%\c
%\c endstruc
%
%The above code defines six symbols: \c{mt_long} as 0 (the offset
%from the beginning of a \c{mytype} structure to the longword field),
%\c{mt_word} as 4, \c{mt_byte} as 6, \c{mt_str} as 7, \c{mytype_size}
%as 39, and \c{mytype} itself as zero.
%
%The reason why the structure type name is defined at zero by default
%is a side effect of allowing structures to work with the local label
%mechanism: if your structure members tend to have the same names in
%more than one structure, you can define the above structure like this:
%
%\c struc mytype
%\c
%\c   .long:        resd    1
%\c   .word:        resw    1
%\c   .byte:        resb    1
%\c   .str:         resb    32
%\c
%\c endstruc
%
%This defines the offsets to the structure fields as \c{mytype.long},
%\c{mytype.word}, \c{mytype.byte} and \c{mytype.str}.
%
%NASM, since it has no \e{intrinsic} structure support, does not
%support any form of period notation to refer to the elements of a
%structure once you have one (except the above local-label notation),
%so code such as \c{mov ax,[mystruc.mt_word]} is not valid.
%\c{mt_word} is a constant just like any other constant, so the
%correct syntax is \c{mov ax,[mystruc+mt_word]} or \c{mov
%ax,[mystruc+mytype.word]}.
%
%Sometimes you only have the address of the structure displaced by an
%offset. For example, consider this standard stack frame setup:
%
%\c push ebp
%\c mov ebp, esp
%\c sub esp, 40
%
%In this case, you could access an element by subtracting the offset:
%
%\c mov [ebp - 40 + mytype.word], ax
%
%However, if you do not want to repeat this offset, you can use -40 as
%a base offset:
%
%\c struc mytype, -40
%
%And access an element this way:
%
%\c mov [ebp + mytype.word], ax
%
%
%\S{istruc} \i\c{ISTRUC}, \i\c{AT} and \i\c{IEND}: Declaring
%\i{Instances of Structures}
%
%Having defined a structure type, the next thing you typically want
%to do is to declare instances of that structure in your data
%segment. NASM provides an easy way to do this in the \c{ISTRUC}
%mechanism. To declare a structure of type \c{mytype} in a program,
%you code something like this:
%
%\c mystruc:
%\c     istruc mytype
%\c
%\c         at mt_long, dd      123456
%\c         at mt_word, dw      1024
%\c         at mt_byte, db      'x'
%\c         at mt_str,  db      'hello, world', 13, 10, 0
%\c
%\c     iend
%
%The function of the \c{AT} macro is to make use of the \c{TIMES}
%prefix to advance the assembly position to the correct point for the
%specified structure field, and then to declare the specified data.
%Therefore the structure fields must be declared in the same order as
%they were specified in the structure definition.
%
%If the data to go in a structure field requires more than one source
%line to specify, the remaining source lines can easily come after
%the \c{AT} line. For example:
%
%\c         at mt_str,  db      123,134,145,156,167,178,189
%\c                     db      190,100,0
%
%Depending on personal taste, you can also omit the code part of the
%\c{AT} line completely, and start the structure field on the next
%line:
%
%\c         at mt_str
%\c                 db      'hello, world'
%\c                 db      13,10,0
%
%
%\S{align} \i\c{ALIGN} and \i\c{ALIGNB}: Data Alignment
%
%The \c{ALIGN} and \c{ALIGNB} macros provides a convenient way to
%align code or data on a word, longword, paragraph or other boundary.
%(Some assemblers call this directive \i\c{EVEN}.) The syntax of the
%\c{ALIGN} and \c{ALIGNB} macros is
%
%\c         align   4               ; align on 4-byte boundary
%\c         align   16              ; align on 16-byte boundary
%\c         align   8,db 0          ; pad with 0s rather than NOPs
%\c         align   4,resb 1        ; align to 4 in the BSS
%\c         alignb  4               ; equivalent to previous line
%
%Both macros require their first argument to be a power of two; they
%both compute the number of additional bytes required to bring the
%length of the current section up to a multiple of that power of two,
%and then apply the \c{TIMES} prefix to their second argument to
%perform the alignment.
%
%If the second argument is not specified, the default for \c{ALIGN}
%is \c{NOP}, and the default for \c{ALIGNB} is \c{RESB 1}. So if the
%second argument is specified, the two macros are equivalent.
%Normally, you can just use \c{ALIGN} in code and data sections and
%\c{ALIGNB} in BSS sections, and never need the second argument
%except for special purposes.
%
%\c{ALIGN} and \c{ALIGNB}, being simple macros, perform no error
%checking: they cannot warn you if their first argument fails to be a
%power of two, or if their second argument generates more than one
%byte of code. In each of these cases they will silently do the wrong
%thing.
%
%\c{ALIGNB} (or \c{ALIGN} with a second argument of \c{RESB 1}) can
%be used within structure definitions:
%
%\c struc mytype2
%\c
%\c   mt_byte:
%\c         resb 1
%\c         alignb 2
%\c   mt_word:
%\c         resw 1
%\c         alignb 4
%\c   mt_long:
%\c         resd 1
%\c   mt_str:
%\c         resb 32
%\c
%\c endstruc
%
%This will ensure that the structure members are sensibly aligned
%relative to the base of the structure.
%
%A final caveat: \c{ALIGN} and \c{ALIGNB} work relative to the
%beginning of the \e{section}, not the beginning of the address space
%in the final executable. Aligning to a 16-byte boundary when the
%section you're in is only guaranteed to be aligned to a 4-byte
%boundary, for example, is a waste of effort. Again, NASM does not
%check that the section's alignment characteristics are sensible for
%the use of \c{ALIGN} or \c{ALIGNB}.
%
%Both \c{ALIGN} and \c{ALIGNB} do call \c{SECTALIGN} macro implicitly.
%See \k{sectalign} for details.
%
%See also the \c{smartalign} standard macro package, \k{pkg_smartalign}.
%
%
%\S{sectalign} \i\c{SECTALIGN}: Section Alignment
%
%The \c{SECTALIGN} macros provides a way to modify alignment attribute
%of output file section. Unlike the \c{align=} attribute (which is allowed
%at section definition only) the \c{SECTALIGN} macro may be used at any time.
%
%For example the directive
%
%\c SECTALIGN 16
%
%sets the section alignment requirements to 16 bytes. Once increased it can
%not be decreased, the magnitude may grow only.
%
%Note that \c{ALIGN} (see \k{align}) calls the \c{SECTALIGN} macro implicitly
%so the active section alignment requirements may be updated. This is by default
%behaviour, if for some reason you want the \c{ALIGN} do not call \c{SECTALIGN}
%at all use the directive
%
%\c SECTALIGN OFF
%
%It is still possible to turn in on again by
%
%\c SECTALIGN ON
%
%
%\C{macropkg} \i{Standard Macro Packages}
%
%The \i\c{%use} directive (see \k{use}) includes one of the standard
%macro packages included with the NASM distribution and compiled into
%the NASM binary.  It operates like the \c{%include} directive (see
%\k{include}), but the included contents is provided by NASM itself.
%
%The names of standard macro packages are case insensitive, and can be
%quoted or not.
