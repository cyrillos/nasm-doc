%
% vim: ts=4 sw=4 et
%
\xchapter{run}{Running NASM}

\xsection{cmdline}{NASM \textindexlc{Command-Line} Syntax}

To assemble a file, you issue a command of the form

\begin{lstlisting}
nasm -f <format> <filename> [-o <output>]
\end{lstlisting}

For example,

\begin{lstlisting}
nasm -f elf myfile.asm
\end{lstlisting}

will assemble \code{myfile.asm} into an ``ELF'' object
file \code{myfile.o}. And

\begin{lstlisting}
nasm -f bin myfile.asm -o myfile.com
\end{lstlisting}

will assemble ``myfile.asm'' into a raw binary file ``myfile.com''.

To produce a listing file, with the hex codes output from NASM
displayed on the left of the original sources, use the \code{-l}
option to give a listing file name, for example:

\begin{lstlisting}
nasm -f coff myfile.asm -l myfile.lst
\end{lstlisting}

To get further usage instructions from NASM, try typing

\begin{lstlisting}
nasm -h
\end{lstlisting}

As \code{-hf}, this will also list the available output file formats,
and what they are.

If you use Linux but aren't sure whether your system is
``a.out'' or ``ELF'', type

\begin{lstlisting}
file nasm
\end{lstlisting}

(in the directory in which you put the NASM binary when you
installed it). If it says something like

\begin{lstlisting}
nasm: ELF 32-bit LSB executable i386 (386 and up) Version 1
\end{lstlisting}

then your system is ``ELF'', and you should use the option \code{-f elf}
when you want NASM to produce Linux object files. If it says

\begin{lstlisting}
nasm: Linux/i386 demand-paged executable (QMAGIC)
\end{lstlisting}

or something similar, your system is ``a.out'', and you should use
\code{-f aout} instead (Linux ``a.out'' systems have long been
obsolete, and are rare these days.)

Like Unix compilers and assemblers, NASM is silent unless it
goes wrong: you won't see any output at all, unless it gives error
messages.

\xsubsection{opt-o}{The \codeindex{-o} Option: Specifying the
\textindexlc{Output File Name}}

NASM will normally choose the name of your output file for you;
precisely how it does this is dependent on the object file format.
For Microsoft object file formats (\code{obj}, \code{win32}
and \code{win64}), it will remove the ``.asm'' \textindex{extension}
(or whatever extension you like to use~-- NASM doesn't care) from your
source file name and substitute ``.obj''. For Unix object file formats
(\code{aout}, \code{as86}, \code{coff}, \code{elf32}, \code{elf64},
\code{elfx32}, \code{ieee}, \code{macho32} and \code{macho64})
it will substitute ``.o''.

For \code{dbg}, \code{rdf}, \code{ith} and \code{srec}, it will use
``.dbg'', ``.rdf'', ``.ith'' and ``.srec'', respectively, and for
the \code{bin} format it will simply remove the extension, so that
``myfile.asm'' produces the output file ``myfile''.

If the output file already exists, NASM will overwrite it, unless it
has the same name as the input file, in which case it will give a
warning and use ``\textindex{nasm.out}'' as the output
file name instead.

For situations in which this behaviour is unacceptable, NASM
provides the \code{-o} command-line option, which allows you to
specify your desired output file name. You invoke \code{-o} by
following it with the name you wish for the output file, either
with or without an intervening space. For example:

\begin{lstlisting}
nasm -f bin program.asm -o program.com
nasm -f bin driver.asm -o driver.sys
\end{lstlisting}

Note that this is a small \code{-o}, and is different from a capital
\code{-O}, which is used to specify the number of optimisation passes
required. See \fullref{subsec:opt-O-cap}.

\xsubsection{opt-f}{The \codeindex{-f} Option: Specifying the
\textindexlc{Output File Format}}

If you do not supply the \code{-f} option to NASM, it will choose an
output file format for you itself. In the distribution versions of
NASM, the default is always \codeindex{bin}; if you've compiled
your own copy of NASM, you can redefine \codeindex{OF\_DEFAULT}
at compile time and choose what you want the default to be.

Like \code{-o}, the intervening space between \code{-f} and the output
file format is optional; so \code{-f elf} and \code{-felf} are both valid.

A complete list of the available output file formats can be given by
issuing the command \codeindex{nasm -hf}.

\xsubsection{opt-l}{The \codeindex{-l} Option: Generating a \textindexlc{Listing File}}

If you supply the \code{-l} option to NASM, followed (with the usual
optional space) by a file name, NASM will generate a \textindex{source-listing file}
for you, in which addresses and generated code are listed on the left, and the
actual source code, with expansions of multi-line macros (except those which
specifically request no expansion in source listings: see \fullref{subsec:nolist})
on the right. For example:

\begin{lstlisting}
nasm -f elf myfile.asm -l myfile.lst
\end{lstlisting}

If a list file is selected, you may turn off listing for a section of your
source with \code{[list -]}, and turn it back on with \code{[list +]},
(the default, obviously). There is no ``user form'' (without the brackets).
This can be used to list only sections of interest, avoiding excessively
long listings.

\xsubsection{opt-M-cap}{The \codeindex{-M} Option: Generate
\textindexlc{Makefile Dependencies}}

This option can be used to generate makefile dependencies on stdout.
This can be redirected to a file for further processing. For example:

\begin{lstlisting}
nasm -M myfile.asm > myfile.dep
\end{lstlisting}

\xsubsection{opt-MG-cap}{The \codeindex{-MG} Option: Generate
\textindexlc{Makefile Dependencies}}

This option can be used to generate makefile dependencies on stdout.
This differs from the \code{-M} option in that if a nonexisting file is
encountered, it is assumed to be a generated file and is added to the
dependency list without a prefix.

\xsubsection{opt-MF-cap}{The \codeindex{-MF} Option: Set Makefile Dependency File}

This option can be used with the \code{-M} or \code{-MG} options
to send the output to a file, rather than to stdout. For example:

\begin{lstlisting}
nasm -M -MF myfile.dep myfile.asm
\end{lstlisting}

\xsubsection{opt-MD-cap}{The \codeindex{-MD} Option: Assemble
and Generate Dependencies}

The \code{-MD} option acts as the combination of the \code{-M}
and \code{-MF} options (i.e. a filename has to be specified).
However, unlike the \code{-M} or \code{-MG} options, \code{-MD}
does \emph{not} inhibit the normal operation of the assembler.
Use this to automatically generate updated dependencies with
every assembly session. For example:

\begin{lstlisting}
nasm -f elf -o myfile.o -MD myfile.dep myfile.asm
\end{lstlisting}

\xsubsection{opt-MT-cap}{The \codeindex{-MT} Option:
Dependency Target Name}

The \code{-MT} option can be used to override the default name of the
dependency target. This is normally the same as the output filename,
specified by the \code{-o} option.

\xsubsection{opt-MQ-cap}{The \codeindex{-MQ} Option:
Dependency Target Name (Quoted)}

The \code{-MQ} option acts as the \code{-MT} option, except
it tries to quote characters that have special meaning in Makefile
syntax. This is not foolproof, as not all characters with special
meaning are quotable in Make. The default output (if no \code{-MT} or
\code{-MQ} option is specified) is automatically quoted.

\xsubsection{opt-MP-cap}{The \codeindex{-MP} Option:
Emit phony targets}

When used with any of the dependency generation options, the
\code{-MP} option causes NASM to emit a phony target without
dependencies for each header file. This prevents Make from
complaining if a header file has been removed.

\xsubsection{opt-F-cap}{The \codeindex{-F} Option:
Selecting a \textindexlc{Debug Information Format}}

This option is used to select the format of the debug information
emitted into the output file, to be used by a debugger (or \emph{will}
be). Prior to version 2.03.01, the use of this switch did \emph{not}
enable output of the selected debug info format. Use \codeindex{-g},
see \fullref{opt-g}, to enable output. Versions 2.03.01 and later
automatically enable \code{-g} if \code{-F} is specified.

A complete list of the available debug file formats for an output
format can be seen by issuing the command \code{nasm -f <format> -y}.
Not all output formats currently support debugging output.
See \fullref{opt-y}.

This should not be confused with the \code{-f dbg} output format
option which is not built into NASM by default. For information on how
to enable it when building from the sources, see \fullref{sec:dbgfmt}.

\xsubsection{opt-g}{The \codeindex{-g} Option:
Enabling \textindexlc{Debug Information}}

This option can be used to generate debugging information in the specified
format. See \fullref{opt-F-cap}. Using \code{-g} without \code{-F}
results in emitting debug info in the default format, if any, for the
selected output format. If no debug information is currently implemented
in the selected output format, \code{-g} is \emph{silently ignored}.

\xsubsection{opt-X-cap}{The \codeindex{-X} Option:
Selecting an \textindexlc{Error Reporting Format}}

This option can be used to select an error reporting format for any
error messages that might be produced by NASM.

Currently, two error reporting formats may be selected. They are
the \code{-Xvc} option and the \code{-Xgnu} option.
The GNU format is the default and looks like this:

\begin{lstlisting}
filename.asm:65: error: specific error message
\end{lstlisting}

where \code{filename.asm} is the name of the source file in
which the error was detected, \code{65} is the source file
line number on which the error was detected, \code{error}
is the severity of the error (this could be \code{warning}),
and \code{specific error message} is a more detailed text message
which should help pinpoint the exact problem.

The other format, specified by \code{-Xvc} is the style used by
Microsoft Visual C++ and some other programs. It looks like this:

\begin{lstlisting}
filename.asm(65) : error: specific error message
\end{lstlisting}

where the only difference is that the line number is in parentheses
instead of being delimited by colons.

See also the \code{Visual C++} output format, \fullref{sec:win32fmt}.

\xsubsection{opt-Z-cap}{The \codeindex{-Z} Option:
Send Errors to a File}

Under ``MS-\textindex{DOS}'' it can be difficult (though there are
ways) to redirect the standard-error output of a program to a file.
Since NASM usually produces its warning and \textindex{error messages}
on \codeindex{stderr}, this can make it hard to capture the
errors if (for example) you want to load them into an editor.

NASM therefore provides the \code{-Z} option, taking a filename argument
which causes errors to be sent to the specified files rather than standard
error. Therefore you can \index{redirecting errors}redirect the errors
into a file by typing

\begin{lstlisting}
nasm -Z myfile.err -f obj myfile.asm
\end{lstlisting}

In earlier versions of NASM, this option was called \code{-E},
but it was changed since \code{-E} is an option conventionally
used for preprocessing only, with disastrous results.
See \fullref{subsec:opt-E-cap}.

\xsubsection{opt-s}{The \codeindex{-s} Option:
Send Errors to \codeindex{stdout}}

The \code{-s} option redirects \textindexlc{error messages} to
\code{stdout} rather than \code{stderr}, so it can be redirected
under ``MS-\textindex{DOS}''. To assemble the file \code{myfile.asm}
and pipe its output to the \code{more} program, you can type:

\begin{lstlisting}
nasm -s -f obj myfile.asm | more
\end{lstlisting}

See also the \code{-Z} option, \fullref{subsec:opt-Z-cap}.

\xsubsection{opt-i}{The \codeindex{-i}\indexcode{-I} Option:
Include File Search Directories}

When NASM sees the \codeindex{\%include} or \codeindex{\%pathsearch} directive
in a source file (see \fullref{sec:include}, \fullref{sec:pathsearch} or
\fullref{sec:incbin}), it will search for the given file not only in the
current directory, but also in any directories specified on the command
line by the use of the \code{-i} option. Therefore you can include files
from a \textindex{macro library}, for example, by typing

\begin{lstlisting}
nasm -ic:\macrolib\ -f obj myfile.asm
\end{lstlisting}

(As usual, a space between \code{-i} and the path name is allowed, and
optional).

NASM, in the interests of complete source-code portability, does not
understand the file naming conventions of the OS it is running on;
the string you provide as an argument to the \code{-i} option will be
prepended exactly as written to the name of the include file.
Therefore the trailing backslash in the above example is necessary.
Under Unix, a trailing forward slash is similarly necessary.

(You can use this to your advantage, if you're really \textindex{perverse},
by noting that the option \code{-ifoo} will cause \code{\%include "bar.i"}
to search for the file \code{foobar.i}...)

If you want to define a \emph{standard} \textindex{include search path},
similar to \code{/usr/include} on Unix systems, you should place one or
more \code{-i} directives in the \code{NASMENV} environment variable (see
\fullref{sec:nasmenv}).

For Makefile compatibility with many C compilers, this option can also
be specified as \code{-I}.

\xsubsection{opt-p}{The \codeindex{-p}\indexcode{-P} Option:
Pre-Include a File}
\index{pre-including files}

\indexcode{\%include}NASM allows you to specify files to be \emph{pre-included} into
your source file, by the use of the \code{-p} option. So running

\begin{lstlisting}
nasm myfile.asm -p myinc.inc
\end{lstlisting}

is equivalent to running \code{nasm myfile.asm} and placing the
directive \code{\%include "myinc.inc"} at the start of the file.

For consistency with the \code{-I}, \code{-D} and \code{-U} options,
this option can also be specified as \code{-P}.

\xsubsection{opt-d}{The \codeindex{-d}\indexcode{-D} Option:
Pre-Define a Macro}
\index{pre-defining macros}

\indexcode{\%define}Just as the \code{-p} option gives an alternative to placing
\code{\%include} directives at the start of a source file, the \code{-d}
option gives an alternative to placing a \code{\%define} directive. You
could code

\begin{lstlisting}
nasm myfile.asm -dFOO=100
\end{lstlisting}

as an alternative to placing the directive

\begin{lstlisting}
%define FOO 100
\end{lstlisting}

at the start of the file. You can miss off the macro value, as well:
the option \code{-dFOO} is equivalent to coding \code{\%define FOO}.
This form of the directive may be useful for selecting \textindex{assembly-time
options} which are then tested using \code{\%ifdef}, for example \code{-dDEBUG}.

For Makefile compatibility with many C compilers, this option can also
be specified as \code{-D}.

\xsubsection{opt-u}{The \codeindex{-u}\indexcode{-U} Option:
Undefine a Macro}
\index{undefining macros}

\indexcode{\%undef}The \code{-u} option undefines a macro that would otherwise
have been pre-defined, either automatically or by a \code{-p} or \code{-d}
option specified earlier on the command lines.

For example, the following command line:

\begin{lstlisting}
nasm myfile.asm -dFOO=100 -uFOO
\end{lstlisting}

would result in \code{FOO} \emph{not} being a predefined macro in the
program. This is useful to override options specified at a different
point in a Makefile.

For Makefile compatibility with many C compilers, this option can also
be specified as \code{-U}.

\xsubsection{opt-E-cap}{The \codeindex{-E}\indexcode{-e} Option: Preprocess Only}

NASM allows the \textindex{preprocessor} to be run on its own, up to a
point. Using the \code{-E} option (which requires no arguments) will
cause NASM to preprocess its input file, expand all the macro references,
remove all the comments and preprocessor directives, and print the resulting
file on standard output (or save it to a file, if the \code{-o} option
is also used).

This option cannot be applied to programs which require the
preprocessor to evaluate \index{preprocessor expressions}
\textindex{expressions} which depend on the values of symbols:
so code such as

\begin{lstlisting}
%assign tablesize ($-tablestart)
\end{lstlisting}

will cause an error in \textindex{preprocess-only mode}.

For compatiblity with older version of NASM, this option can also be
written \code{-e}. \code{-E} in older versions of NASM was the equivalent
of the current \code{-Z} option, \fullref{subsec:opt-Z-cap}.

\xsubsection{opt-a}{The \codeindex{-a} Option: Don't Preprocess At All}

If NASM is being used as the back end to a compiler, it might be
desirable to \index{suppressing preprocessing}suppress preprocessing
completely and assume the compiler has already done it, to save time
and increase compilation speeds. The \code{-a} option, requiring no
argument, instructs NASM to replace its powerful \textindex{preprocessor}
with a \textindex{stub preprocessor} which does nothing.

\subsection{opt-O-cap}{The \codeindex{-O} Option: Specifying
\textindexlc{Multipass Optimization}}

Using the \code{-O} option, you can tell NASM to carry out different
levels of optimization. The syntax is:

\begin{itemize}
	\item \code{-O0}: No optimization. All operands take their
	long forms, if a short form is not specified, except conditional
	jumps. This is intended to match NASM 0.98 behavior.

	\item \code{-O1}: Minimal optimization. As above, but immediate
	operands which will fit in a signed byte are optimized,
	unless the long form is specified. Conditional jumps default
	to the long form unless otherwise specified.

	\item \code{-Ox} (where \code{x} is the actual letter \code{x}):
	Multipass optimization. Minimize branch offsets and signed immediate
	bytes, overriding size specification unless the \code{strict} keyword
	has been used (see \fullref{sec:strict}). For compatibility with earlier
	releases, the letter \code{x} may also be any number greater than
	one. This number has no effect on the actual number of passes.
\end{itemize}

The \code{-Ox} mode is recommended for most uses, and is the default
since NASM 2.09.

Note that this is a capital \code{O}, and is different from a small \code{o},
which is used to specify the output file name. See \fullref{subsec:opt-o}.

\xsubsection{opt-t}{The \codeindex{-t} Option: Enable TASM Compatibility Mode}

NASM includes a limited form of compatibility with Borland's \textindex{TASM}.
When NASM's \code{-t} option is used, the following changes are made:

\begin{itemize}
    \item{local labels may be prefixed with \code{@@} instead of \code{.};}

    \item{size override is supported within brackets. In TASM compatible mode,
        a size override inside square brackets changes the size of the operand,
        and not the address type of the operand as it does in NASM syntax. E.g.
        \code{mov eax,[DWORD val]} is valid syntax in TASM compatibility mode.
        Note that you lose the ability to override the default address type for
        the instruction;}

    \item{unprefixed forms of some directives supported (\code{arg}, \code{elif},
        \code{else}, \code{endif}, \code{if}, \code{ifdef}, \code{ifdifi},
        \code{ifndef}, \code{include}, \code{local}).}
\end{itemize}

\xsubsection{opt-w}{The \codeindex{-w} and \codeindex{-W} Options:
Enable or Disable Assembly \textindexlc{Warnings}}

NASM can observe many conditions during the course of assembly which
are worth mentioning to the user, but not a sufficiently severe
error to justify NASM refusing to generate an output file. These
conditions are reported like errors, but come up with the word
``warning'' before the message. Warnings do not prevent NASM from
generating an output file and returning a success status to the
operating system.

Some conditions are even less severe than that: they are only
sometimes worth mentioning to the user. Therefore NASM supports the
\code{-w} command-line option, which enables or disables certain
classes of assembly warning. Such warning classes are described by a
name, for example \code{orphan-labels}; you can enable warnings of
this class by the command-line option \code{-w+orphan-labels} and
disable it by \code{-w-orphan-labels}.

The current \textindex{warning classes} are:
\begin{itemize}

	\item \codeindex{other} specifies any warning not otherwise
	specified in any class. Enabled by default.

	\item \codeindex{macro-params} covers warnings about
	\textindex{multi-line macros} being invoked with the wrong number
	of parameters. Enabled by default, see \fullref{sec:mlmacover}
	for an example of why you might want to disable it.

	\item \codeindex{macro-selfref} warns if a macro references itself.
	Disabled by default.

	\item \codeindex{macro-defaults} warns when a macro has more
	default parameters than optional parameters. Enabled by default,
	see \fullref{sec:mlmacdef} for why you might want to disable it.

	\item \codeindex{orphan-labels} covers warnings about source lines
	which contain no instruction but define a label without a trailing colon.
	NASM warns about this somewhat obscure condition by default,
	see \fullref{sec:syntax} for more information.

	\item \codeindex{number-overflow} covers warnings about numeric
	constants which don't fit in 64 bits. Enabled by default.

	\item \codeindex{gnu-elf-extensions} warns if 8-bit or 16-bit
	relocations are used in \code{-f elf} format. The GNU extensions
	allow this. Disabled by default.

	\item \codeindex{float-overflow} warns about floating point overflow.
	Enabled by default.

	\item \codeindex{float-denorm} warns about floating point denormals.
	Disabled by default.

	\item \codeindex{float-underflow} warns about floating point underflow.
	Disabled by default.

	\item \codeindex{float-toolong} warns about too many digits in
	floating-point numbers. Enabled by default.

	\item \codeindex{user} controls \code{\%warning} directives (see
	\fullref{sec:pperror}). Enabled by default.

	\item \codeindex{lock} warns about \code{LOCK} prefixes on unlockable
	instructions. Enabled by default.

	\item \codeindex{hle} warns about invalid use of the HLE \code{XACQUIRE}
	or \code{XRELEASE} prefixes. Enabled by default.

	\item \codeindex{bnd} warns about ineffective use of the \code{BND}
	prefix when a relaxed form of jmp instruction becomes jmp short form.
	Enabled by default.

	\item \codeindex{zext-reloc} warns that a relocation has been
	zero-extended due to limitations in the output format. Enabled by default.

	\item \codeindex{ptr} warns about keywords used in other assemblers that might
	indicate a mistake in the source code. Currently only the MASM
	\code{PTR} keyword is recognized. Enabled by default.

	\item \codeindex{bad-pragma} warns about a malformed or otherwise unparsable
	\code{\%pragma} directive. Disabled by default.

	\item \codeindex{unknown-pragma} warns about an unknown \code{\%pragma} directive.
	This is not yet implemented. Disabled by default.

	\item \codeindex{not-my-pragma} warns about a \code{\%pragma} directive which is
	not applicable to this particular assembly session. This is not yet
	implemented. Disabled by default.

	\item \codeindex{unknown-warning} warns about a \code{-w} or \code{-W} option or a
	\code{[WARNING]} directive that contains an unknown warning name or is
	otherwise not possible to process. Disabled by default.

	\item \codeindex{all} is an alias for \emph{all} suppressible warning classes.
	Thus, \code{-w+all} enables all available warnings, and \code{-w-all}
	disables warnings entirely (since NASM 2.13).
\end{itemize}

Since version 2.00, NASM has also supported the \code{gcc}-like syntax
\code{-Wwarning-class} and \code{-Wno-warning-class} instead of
\code{-w+warning-class} and \code{-w-warning-class}, respectively; both
syntaxes work identically.

The option \code{-w+error} or \codeindex{-Werror} can be used to treat warnings
as errors. This can be controlled on a per warning class basis
(\code{-w+error=}\emph{warning-class} or \code{-Werror=}\emph{warning-class});
if no \emph{warning-class} is specified NASM treats it as
\code{-w+error=all}; the same applies to \code{-w-error} or
\codeindex{-Wno-error}, of course.

In addition, you can control warnings in the source code itself, using
the \codeindex{[WARNING]} directive. See \fullref{sec:asmdir-warning}.

\xsubsection{opt-v}{The \codeindex{-v} Option: Display \textindexlc{Version} Info}

Typing \code{NASM -v} will display the version of NASM which you are using,
and the date on which it was compiled.

You will need the version number if you report a bug.

For command-line compatibility with Yasm, the form \codeindex{--v} is also
accepted for this option starting in NASM version 2.11.05.

\xsubsection{opt-y}{The \codeindex{-y} Option: Display Available Debug Info Formats}

Typing \code{nasm -f <option> -y} will display a list of the available
debug info formats for the given output format. The default format
is indicated by an asterisk. For example:

\begin{lstlisting}
nasm -f elf -y

valid debug formats for 'elf32' output format are
('*' denotes default):
* stabs     ELF32 (i386) stabs debug format for Linux
  dwarf     elf32 (i386) dwarf debug format for Linux
\end{lstlisting}

\xsubsection{opt-pfix}{The \codeindex{--prefix} and \codeindex{--postfix} Options}

The \code{--prefix} and \code{--postfix} options prepend or append
(respectively) the given argument to all \code{global} or
\code{extern} variables. E.g. \code{--prefix \_} will prepend the
underscore to all global and external variables, as C requires it in
some, but not all, system calling conventions.

\xsubsection{nasmenv}{The \codeindex{NASMENV} \textindex{Environment} Variable}

If you define an environment variable called \code{NASMENV}, the program
will interpret it as a list of extra command-line options, which are
processed before the real command line. You can use this to define
standard search directories for include files, by putting \code{-i}
options in the \code{NASMENV} variable.

The value of the variable is split up at white space, so that the
value \code{-s -ic:\textbackslash nasmlib\textbackslash} will be
treated as two separate options. However, that means that the value
\code{-dNAME="my name"} won't do what you might want, because it
will be split at the space and the NASM command-line processing
will get confused by the two nonsensical words \code{-dNAME="my}
and \code{name"}.

To get round this, NASM provides a feature whereby, if you begin the
\code{NASMENV} environment variable with some character that isn't
a minus sign, then NASM will treat this character as the
\textindex{separator character} for options. So setting the \code{NASMENV}
variable to the value \code{!-s!-ic:\textbackslash nasmlib\textbackslash}
is equivalent to setting it to \code{-s -ic:\textbackslash nasmlib\textbackslash},
but \code{!-dNAME="my name"} will work.

This environment variable was previously called \code{NASM}. This was
changed with version 0.98.31.

\xsection{qstart}{\textindex{Quick Start} for \textindex{MASM} Users}

If you're used to writing programs with MASM, or with \textindex{TASM} in
MASM-compatible (non-Ideal) mode, or with \textindex{a86}, this section
attempts to outline the major differences between MASM's syntax and
NASM's. If you're not already used to MASM, it's probably worth
skipping this section.

\xsubsection{qscs}{NASM Is \index{case sensitivity}Case-Sensitive}

One simple difference is that NASM is case-sensitive. It makes a
difference whether you call your label \code{foo}, \code{Foo} or
\code{FOO}. If you're assembling to DOS or OS/2 ``.OBJ'' files,
you can invoke the \codeindex{UPPERCASE} directive (documented in
\fullref{sec:objfmt}) to ensure that all symbols exported to other
code modules are forced to be upper case; but even then, \emph{within}
a single module, NASM will distinguish between labels differing only
in case.

\xsubsection{qsbrackets}{NASM Requires \textindexlc{Square Brackets}
For \textindexlc{Memory References}}

NASM was designed with simplicity of syntax in mind. One of the
\textindex{design goals} of NASM is that it should be possible,
as far as is practical, for the user to look at a single line of
NASM code and tell what opcode is generated by it. You can't do
this in MASM: if you declare, for example,

\begin{lstlisting}
foo     equ     1
bar     dw      2
\end{lstlisting}

then the two lines of code

\begin{lstlisting}
mov     ax,foo
mov     ax,bar
\end{lstlisting}

generate completely different opcodes, despite having
identical-looking syntaxes.

NASM avoids this undesirable situation by having a much simpler
syntax for memory references. The rule is simply that any access to
the \emph{contents} of a memory location requires square brackets
around the address, and any access to the \emph{address} of a variable
doesn't. So an instruction of the form \code{mov ax,foo} will
\emph{always} refer to a compile-time constant, whether it's an \code{EQU}
or the address of a variable; and to access the \emph{contents} of the
variable \code{bar}, you must code \code{mov ax,[bar]}.

This also means that NASM has no need for MASM's \codeindex{OFFSET}
keyword, since the MASM code \code{mov ax,offset bar} means exactly the
same thing as NASM's \code{mov ax,bar}. If you're trying to get
large amounts of MASM code to assemble sensibly under NASM, you
can always code \code{\%idefine offset} to make the preprocessor
treat the \code{OFFSET} keyword as a no-op.

This issue is even more confusing in \textindex{a86}, where declaring a
label with a trailing colon defines it to be a `label' as opposed to
a `variable' and causes a86 to adopt NASM-style semantics; so in
a86, \code{mov ax,var} has different behaviour depending on whether
\code{var} was declared as \code{var: dw 0} (a label) or
\code{var dw 0} (a word-size variable). NASM is very simple by
comparison: \emph{everything} is a label.

NASM, in the interests of simplicity, also does not support the
\textindex{hybrid syntaxes} supported by MASM and its clones, such as
\code{mov ax,table[bx]}, where a memory reference is denoted by one
portion outside square brackets and another portion inside. The
correct syntax for the above is \code{mov ax,[table+bx]}. Likewise,
\code{mov ax,es:[di]} is wrong and \code{mov ax,[es:di]} is right.

\xsubsection{qstypes}{NASM Doesn't Store \textindexlc{Variable Types}}

NASM, by design, chooses not to remember the types of variables you
declare. Whereas MASM will remember, on seeing \code{var dw 0}, that
you declared \code{var} as a word-size variable, and will then be able
to fill in the \textindex{ambiguity} in the size of the instruction
\code{mov var,2}, NASM will deliberately remember nothing about
the symbol \code{var} except where it begins, and so you must
explicitly code \code{mov word [var],2}.

For this reason, NASM doesn't support the \code{LODS}, \code{MOVS},
\code{STOS}, \code{SCAS}, \code{CMPS}, \code{INS}, or \code{OUTS}
instructions, but only supports the forms such as \code{LODSB},
\code{MOVSW}, and \code{SCASD}, which explicitly specify the size
of the components of the strings being manipulated.

\xsubsection{qsassume}{NASM Doesn't \codeindex{ASSUME}}

As part of NASM's drive for simplicity, it also does not support the
\code{ASSUME} directive. NASM will not keep track of what values you
choose to put in your segment registers, and will never \emph{automatically}
generate a \textindex{segment override} prefix.

\xsubsection{qsmodel}{NASM Doesn't Support \textindexlc{Memory Models}}

NASM also does not have any directives to support different 16-bit
memory models. The programmer has to keep track of which functions
are supposed to be called with a \textindex{far call} and which with a
\textindex{near call}, and is responsible for putting the correct form of
\code{RET} instruction (\code{RETN} or \code{RETF}; NASM accepts
\code{RET} itself as an alternate form for \code{RETN}); in addition,
the programmer is responsible for coding CALL FAR instructions where
necessary when calling \emph{external} functions, and must also keep
track of which external variable definitions are far and which are
near.

\xsubsection{qsfpu}{\textindexlc{Floating-Point} Differences}

NASM uses different names to refer to floating-point registers from
MASM: where MASM would call them \code{ST(0)}, \code{ST(1)} and
so on, and \textindex{a86} would call them simply \code{0}, \code{1}
and so on, NASM chooses to call them \code{st0}, \code{st1} etc.

As of version 0.96, NASM now treats the instructions with
`\textindex{nowait}' forms in the same way as MASM-compatible assemblers.
The idiosyncratic treatment employed by 0.95 and earlier was based
on a misunderstanding by the authors.

\xsubsection{qsother}{Other Differences}

For historical reasons, NASM uses the keyword \codeindex{TWORD} where
MASM and compatible assemblers use \codeindex{TBYTE}.

NASM does not declare \textindex{uninitialized storage} in the same way
as MASM: where a MASM programmer might use \code{stack db 64 dup (?)},
NASM requires \code{stack resb 64}, intended to be read as \emph{reserve 64
bytes}. For a limited amount of compatibility, since NASM treats
\code{?} as a valid character in symbol names, you can code \code{? equ 0}
and then writing \code{dw ?} will at least do something vaguely useful.
\index{RESB}\codeindex{DUP} is still not a supported syntax, however.

In addition to all of this, macros and directives work completely
differently to MASM. See \fullref{sec:preproc} and \fullref{sec:directive}
for further details.
