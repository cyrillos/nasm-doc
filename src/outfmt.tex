%
% vim: ts=4 sw=4 et
%
\xchapter{outfmt}{\textindexlc{Output Formats}}

NASM is a portable assembler, designed to be able to compile on any
ANSI C-supporting platform and produce output to run on a variety of
Intel x86 operating systems. For this reason, it has a large number
of available output formats, selected using the \codeindex{-f} option
on the NASM \textindex{command line}. Each of these formats, along with
its extensions to the base NASM syntax, is detailed in this chapter.

\xsection{binfmt}{\codeindex{bin}: \textindexlc{Flat-Form Binary}\index{pure binary} Output}
\index{file extension!bin}

The \code{bin} format does not produce object files: it generates
nothing in the output file except the code you wrote. Such ``pure
binary'' files are used by \textindex{MS-DOS}: \codeindex{.COM}
executables and \codeindex{.SYS} device drivers are pure binary
files. Pure binary output is also useful for \textindex{operating system}
and \textindex{boot loader} development.

The \code{bin} format supports \textindex{multiple section names}.
For details of how NASM handles sections in the \code{bin} format,
see \nref{multisec}.

Using the \code{bin} format puts NASM by default into 16-bit mode
(see \nref{bits}). In order to use \code{bin} to write 32-bit
or 64-bit code, such as an OS kernel, you need to explicitly issue
the \indexcode{BITS}\code{BITS 32} or \indexcode{BITS}\code{BITS 64}
directive.

\code{bin} has no default output file name extension: instead, it
leaves your file name as it is once the original extension has been
removed. Thus, the default is for NASM to assemble \code{binprog.asm}
into a binary file called \code{binprog}.

\xsubsection{binfmt}{\codeindex{ORG}: Binary File \textindexlc{Program Origin}}

The \code{bin} format provides an additional directive to the list
given in \nref{directive}: \code{ORG}. The function of the
\code{ORG} directive is to specify the origin address which NASM
will assume the program begins at when it is loaded into memory.

For example, the following code will generate the longword
\code{0x00000104}:

\begin{lstlisting}
org     0x100
dd      label
label:
\end{lstlisting}

Unlike the \code{ORG} directive provided by MASM-compatible assemblers,
which allows you to jump around in the object file and overwrite
code you have already generated, NASM's \code{ORG} does exactly what
the directive says: \emph{origin}. Its sole function is to specify one
offset which is added to all internal address references within the
section; it does not permit any of the trickery that MASM's version
does. See \nref{proborg} for further comments.

\xsubsection{binseg}{\code{bin} Extensions to the \code{SECTION} Directive}
\index{section!bin extensions to}

The \code{bin} output format extends the \code{SECTION} (or \code{SEGMENT})
directive to allow you to specify the alignment requirements of segments.
This is done by appending the \codeindex{ALIGN} qualifier to the end of
the section-definition line. For example,

\begin{lstlisting}
section .data   align=16
\end{lstlisting}

switches to the section \code{.data} and also specifies that it must be
aligned on a 16-byte boundary.

The parameter to \code{ALIGN} specifies how many low bits of the
section start address must be forced to zero. The alignment value
given may be any power of two.
\index{section alignment!in bin}
\index{segment alignment!in bin}
\index{alignment!in bin sections}

\xsubsection{multisec}{\textindexlc{Multisection} Support for the \code{bin} Format}
\index{bin!multisection}

The \code{bin} format allows the use of multiple sections, of arbitrary names,
besides the ``known'' \code{.text}, \code{.data}, and \code{.bss} names.

\begin{itemize}
    \item{Sections may be designated \codeindex{progbits} or \codeindex{nobits}.
        Default is \code{progbits} (except \code{.bss}, which defaults to
        \code{nobits}, of course).}

    \item{Sections can be aligned at a specified boundary following the previous
        section with \code{align=}, or at an arbitrary byte-granular position with
        \codeindex{start=}.}

    \item{Sections can be given a virtual start address, which will be used
        for the calculation of all memory references within that section
        with \codeindex{vstart=}.}

    \item{Sections can be ordered using \codeindex{follows=}\code{<section>} or
        \codeindex{vfollows=}\code{<section>} as an alternative to specifying
        an explicit start address.}

    \item{Arguments to \code{org}, \code{start}, \code{vstart}, and \code{align=}
        are critical expressions. See \nref{crit}. E.g.
        \code{align=(1 << ALIGN\_SHIFT)} - \code{ALIGN\_SHIFT} must be defined
        before it is used here.}

    \item{Any code which comes before an explicit \code{SECTION} directive
        is directed by default into the \code{.text} section.}

    \item{If an \code{ORG} statement is not given, \code{ORG 0} is used by default.}

    \item{The \code{.bss} section will be placed after the last \code{progbits}
        section, unless \code{start=}, \code{vstart=}, \code{follows=}, or
        \code{vfollows=} has been specified.}

    \item{All sections are aligned on dword boundaries, unless a different
        alignment has been specified.}

    \item{Sections may not overlap.}

    \item{NASM creates the \code{section.<secname>.start} for each section,
        which may be used in your code.}
\end{itemize}

\xsubsection{map}{\textindexlc{Map Files}}
\index{file extension!map}

Map files can be generated in \code{-f bin} format by means of the \code{[map]}
option. Map types of \code{all} (default), \code{brief}, \code{sections},
\code{segments}, or \code{symbols} may be specified. Output may be directed
to \code{stdout} (default), \code{stderr}, or a specified file. E.g.
\code{[map symbols myfile.map]}. No ``user form'' exists, the square
brackets must be used.

\xsection{ithfmt}{\codeindex{ith}: \textindexlc{Intel Hex} Output}
\index{file extension!ith}

The \code{ith} file format produces Intel hex-format files. Just as the
\code{bin} format, this is a flat memory image format with no support for
relocation or linking. It is usually used with ROM programmers and
similar utilities.

All extensions supported by the \code{bin} file format is also supported by
the \code{ith} file format.

\code{ith} provides a default output file-name extension of \code{.ith}.

\xsection{srecfmt}{\codeindex{srec}: \textindexlc{Motorola S-Records} Output}
\index{file extension!srec}

The \code{srec} file format produces Motorola S-records files. Just as the
\code{bin} format, this is a flat memory image format with no support for
relocation or linking. It is usually used with ROM programmers and similar
utilities.

All extensions supported by the \code{bin} file format is also supported by
the \code{srec} file format.

\code{srec} provides a default output file-name extension of \code{.srec}.

\xsection{objfmt}{\codeindex{obj}: \textindexlc{Microsoft OMF}\index{OMF} Object Files}
\index{file extension!obj}

The \code{obj} file format (NASM calls it \code{obj} rather than
\code{omf} for historical reasons) is the one produced by \textindex{MASM}
and \textindex{TASM}, which is typically fed to 16-bit DOS linkers
to produce \codeindex{.EXE} files. It is also the format used by
\textindex{OS/2}.

\code{obj} provides a default output file-name extension of \code{.obj}.

\code{obj} is not exclusively a 16-bit format, though: NASM has full
support for the 32-bit extensions to the format. In particular,
32-bit \code{obj} format files are used by \textindex{Borland's Win32
compilers}, instead of using Microsoft's newer \codeindex{win32} object
file format.

The \code{obj} format does not define any special segment names: you
can call your segments anything you like. Typical names for segments
in \code{obj} format files are \code{CODE}, \code{DATA} and \code{BSS}.

If your source file contains code before specifying an explicit
\code{SEGMENT} directive, then NASM will invent its own segment called
\codeindex{\_\_NASMDEFSEG} for you.

When you define a segment in an \code{obj} file, NASM defines the
segment name as a symbol as well, so that you can access the segment
address of the segment. So, for example:

\begin{lstlisting}
segment data

dvar:   dw      1234

segment code

function:
    mov     ax,data     ; get segment address of data
    mov     ds,ax       ; and move it into DS
    inc     word [dvar] ; now this reference will work
    ret
\end{lstlisting}

The \code{obj} format also enables the use of the \codeindex{SEG}
and \codeindex{WRT} operators, so that you can write code which
does things like

\begin{lstlisting}
extern  foo

    mov     ax,seg foo              ; get preferred segment of foo
    mov     ds,ax
    mov     ax,data                 ; a different segment
    mov     es,ax
    mov     ax,[ds:foo]             ; this accesses `foo'
    mov     [es:foo wrt data],bx    ; so does this
\end{lstlisting}

\xsubsection{objseg}{\code{obj} Extensions to the \code{SEGMENT} Directive}
\index{SEGMENT!obj extensions to}

The \code{obj} output format extends the \code{SEGMENT} (or \code{SECTION})
directive to allow you to specify various properties of the segment
you are defining. This is done by appending extra qualifiers to the
end of the segment-definition line. For example,

\begin{lstlisting}
segment code private align=16
\end{lstlisting}

defines the segment \code{code}, but also declares it to be a private
segment, and requires that the portion of it described in this code
module must be aligned on a 16-byte boundary.

The available qualifiers are:

%\begin{tabular}{ l l }
%\codeindex{CLASS} &
%\begin{minipage}[t]{0.8\columnwidth}
%can be used to specify the segment class; this feature indicates to
%the linker that segments of the same class should be placed near each
%other in the output file.  The class name can be any word, e.g.
%\code{CLASS=CODE}.
%\end{minipage} \\
%
%\codeindex{OVERLAY} &
%\begin{minipage}[t]{0.8\columnwidth}
%like \code{CLASS}, is specified with an arbitrary word as an argument,
%and provides overlay information to an overlay-capable linker.
%\end{minipage}
%\end{tabular}

\begin{itemize}
    \item{\codeindex{PRIVATE}, \codeindex{PUBLIC}, \codeindex{COMMON}
        and \codeindex{STACK} specify the combination characteristics
        of the segment. \code{PRIVATE} segments do not get combined
        with any others by the linker; \code{PUBLIC} and \code{STACK}
        segments get concatenated together at link time; and \code{COMMON}
        segments all get overlaid on top of each other rather than stuck
        end-to-end.}

    \item{\codeindex{ALIGN} is used, as shown above, to specify how many
        low bits of the segment start address must be forced to zero.
        The alignment value given may be any power of two from 1 to 4096;
        in reality, the only values supported are 1, 2, 4, 16, 256 and 4096,
        so if 8 is specified it will be rounded up to 16, and 32, 64 and 128
        will all be rounded up to 256, and so on. Note that alignment to
        4096-byte boundaries is a \textindex{PharLap} extension to the
        format and may not be supported by all linkers.
        \index{section alignment!in OBJ}
        \index{segment alignment!in OBJ}
        \index{alignment!in OBJ sections}}

    \item{\codeindex{CLASS} can be used to specify the segment class;
        this feature indicates to the linker that segments of the same
        class should be placed near each other in the output file.
        The class name can be any word, e.g. \code{CLASS=CODE}.}

    \item{\codeindex{OVERLAY}, like \code{CLASS}, is specified with
        an arbitrary word as an argument, and provides overlay information
        to an overlay-capable linker.}

    \item{Segments can be declared as \codeindex{USE16} or \codeindex{USE32},
        which has the effect of recording the choice in the object file
        and also ensuring that NASM's default assembly mode when assembling
        in that segment is 16-bit or 32-bit respectively.}

    \item{When writing \textindex{OS/2} object files, you should declare
        32-bit segments as \codeindex{FLAT}, which causes the default
        segment base for anything in the segment to be the special group
        \code{FLAT}, and also defines the group if it is not already defined.}

    \item{The \code{obj} file format also allows segments to be declared as
        having a pre-defined absolute segment address, although no linkers
        are currently known to make sensible use of this feature;
        nevertheless, NASM allows you to declare a segment such as
        \code{SEGMENT SCREEN ABSOLUTE=0xB800} if you need to. The
        \codeindex{ABSOLUTE} and \code{ALIGN} keywords are mutually
        exclusive.}
\end{itemize}

NASM's default segment attributes are \code{PUBLIC}, \code{ALIGN=1}, no
class, no overlay, and \code{USE16}.

\xsubsection{group}{\codeindex{GROUP}: Defining Groups of Segments}
\index{segments!groups of}

The \code{obj} format also allows segments to be grouped, so that a
single segment register can be used to refer to all the segments in
a group. NASM therefore supplies the \code{GROUP} directive, whereby
you can code

\begin{lstlisting}
segment data
    ; some data
segment bss
    ; some uninitialized data
group dgroup data bss
\end{lstlisting}

which will define a group called \code{dgroup} to contain the segments
\code{data} and \code{bss}. Like \code{SEGMENT}, \code{GROUP} causes
the group name to be defined as a symbol, so that you can refer to
a variable \code{var} in the \code{data} segment as \code{var wrt data}
or as \code{var wrt dgroup}, depending on which segment value is
currently in your segment register.

If you just refer to \code{var}, however, and \code{var} is declared
in a segment which is part of a group, then NASM will default to giving
you the offset of \code{var} from the beginning of the \emph{group},
not the \emph{segment}. Therefore \code{SEG var}, also, will return
the group base rather than the segment base.

NASM will allow a segment to be part of more than one group, but
will generate a warning if you do this. Variables declared in a
segment which is part of more than one group will default to being
relative to the first group that was defined to contain the segment.

A group does not have to contain any segments; you can still make
\code{WRT} references to a group which does not contain the variable
you are referring to. OS/2, for example, defines the special group
\code{FLAT} with no segments in it.

\xsubsection{uppercase}{\codeindex{UPPERCASE}: Disabling Case Sensitivity in Output}

Although NASM itself is \textindex{case sensitive}, some OMF linkers are
not; therefore it can be useful for NASM to output single-case
object files. The \code{UPPERCASE} format-specific directive causes all
segment, group and symbol names that are written to the object file
to be forced to upper case just before being written. Within a
source file, NASM is still case-sensitive; but the object file can
be written entirely in upper case if desired.

\code{UPPERCASE} is used alone on a line; it requires no parameters.

\xsubsection{import}{\codeindex{IMPORT}: Importing DLL Symbols}
\index{DLL symbols!importing}
\index{symbols!importing from DLLs}

The \code{IMPORT} format-specific directive defines a symbol to be
imported from a DLL, for use if you are writing a DLL's
\textindex{import library} in NASM. You still need to declare the
symbol as \code{EXTERN} as well as using the \code{IMPORT}
directive.

The \code{IMPORT} directive takes two required parameters, separated
by white space, which are (respectively) the name of the symbol you
wish to import and the name of the library you wish to import it
from. For example:

\begin{lstlisting}
import  WSAStartup wsock32.dll
\end{lstlisting}

A third optional parameter gives the name by which the symbol is
known in the library you are importing it from, in case this is not
the same as the name you wish the symbol to be known by to your code
once you have imported it. For example:

\begin{lstlisting}
import  asyncsel wsock32.dll WSAAsyncSelect
\end{lstlisting}

\xsubsection{export}{\codeindex{EXPORT}: Exporting DLL Symbols}
\index{DLL symbols!exporting}
\index{symbols!exporting from DLLs}

The \code{EXPORT} format-specific directive defines a global
symbol to be exported as a DLL symbol, for use if you are
writing a DLL in NASM. You still need to declare the symbol
as \code{GLOBAL} as well as using the \code{EXPORT} directive.

\code{EXPORT} takes one required parameter, which is the name of the
symbol you wish to export, as it was defined in your source file. An
optional second parameter (separated by white space from the first)
gives the \emph{external} name of the symbol: the name by which you
wish the symbol to be known to programs using the DLL. If this name
is the same as the internal name, you may leave the second parameter
off.

Further parameters can be given to define attributes of the exported
symbol. These parameters, like the second, are separated by white
space. If further parameters are given, the external name must also
be specified, even if it is the same as the internal name. The
available attributes are:

\begin{itemize}
    \item{\code{resident} indicates that the exported name is
        to be kept resident by the system loader. This is
        an optimisation for frequently used symbols imported
        by name.}

    \item{\code{nodata} indicates that the exported symbol
        is a function which does not make use of any initialized
        data.}

    \item{\code{parm=NNN}, where \code{NNN} is an integer, sets
        the number of parameter words for the case in which
        the symbol is a call gate between 32-bit and 16-bit
        segments.}

    \item{An attribute which is just a number indicates that
        the symbol should be exported with an identifying
        number (ordinal), and gives the desired number.}
\end{itemize}

For example:

\begin{lstlisting}
export  myfunc
export  myfunc TheRealMoreFormalLookingFunctionName
export  myfunc myfunc 1234  ; export by ordinal
export  myfunc myfunc resident parm=23 nodata
\end{lstlisting}

\xsubsection{dotdotstart}{\codeindex{..start}: Defining the \textindexlc{Program Entry Point}}

\code{OMF} linkers require exactly one of the object files being linked to
define the program entry point, where execution will begin when the
program is run. If the object file that defines the entry point is
assembled using NASM, you specify the entry point by declaring the
special symbol \code{..start} at the point where you wish execution to
begin.

\xsubsection{objextern}{\code{obj} Extensions to the \code{EXTERN} Directive}
\index{EXTERN!obj extensions to}

If you declare an external symbol with the directive

\begin{lstlisting}
extern  foo
\end{lstlisting}

then references such as \code{mov ax,foo} will give you the offset of
\code{foo} from its preferred segment base (as specified in whichever
module \code{foo} is actually defined in). So to access the contents of
\code{foo} you will usually need to do something like

\begin{lstlisting}
mov     ax,seg foo      ; get preferred segment base
mov     es,ax           ; move it into ES
mov     ax,[es:foo]     ; and use offset `foo' from it
\end{lstlisting}

This is a little unwieldy, particularly if you know that an external
is going to be accessible from a given segment or group, say
\code{dgroup}. So if \code{DS} already contained \code{dgroup},
you could simply code

\begin{lstlisting}
mov     ax,[foo wrt dgroup]
\end{lstlisting}

However, having to type this every time you want to access \code{foo}
can be a pain; so NASM allows you to declare \code{foo} in the
alternative form

\begin{lstlisting}
extern  foo:wrt dgroup
\end{lstlisting}

This form causes NASM to pretend that the preferred segment base of
\code{foo} is in fact \code{dgroup}; so the expression \code{seg foo}
will now return \code{dgroup}, and the expression \code{foo} is
equivalent to \code{foo wrt dgroup}.

This \index{default-WRT mechanism}default-\code{WRT} mechanism can be used
to make externals appear to be relative to any group or segment in
your program. It can also be applied to common variables: see
\nref{objcommon}.

\xsubsection{objcommon}{\code{obj} Extensions to the \code{COMMON} Directive}
\index{COMMON!obj extensions to}

The \code{obj} format allows common variables to be either near
\index{common variables!near} or far\index{common variables!far};
NASM allows you to specify which your variables should be by the
use of the syntax

\begin{lstlisting}
common  nearvar 2:near  ; nearvar is a near common
common  farvar  10:far  ; and farvar is far
\end{lstlisting}

Far common variables may be greater in size than 64Kb, and so the
OMF specification says that they are declared as a number of
\emph{elements} of a given size. So a 10-byte far common variable could
be declared as ten one-byte elements, five two-byte elements, two
five-byte elements or one ten-byte element.

Some \code{OMF} linkers require the \index{element size!in common
variables}\index{common variables!element size}element size, as well as
the variable size, to match when resolving common variables declared
in more than one module. Therefore NASM must allow you to specify
the element size on your far common variables. This is done by the
following syntax:

\begin{lstlisting}
common  c_5by2  10:far  5   ; two five-byte elements
common  c_2by5  10:far  2   ; five two-byte elements
\end{lstlisting}

If no element size is specified, the default is 1. Also, the \code{FAR}
keyword is not required when an element size is specified, since
only far commons may have element sizes at all. So the above
declarations could equivalently be

\begin{lstlisting}
common  c_5by2  10:5        ; two five-byte elements
common  c_2by5  10:2        ; five two-byte elements
\end{lstlisting}

In addition to these extensions, the \code{COMMON} directive
in \code{obj} also supports default-\code{WRT} specification
like \code{EXTERN} does (explained in \nref{objextern}).
So you can also declare things like

\begin{lstlisting}
common  foo     10:wrt dgroup
common  bar     16:far 2:wrt data
common  baz     24:wrt data:6
\end{lstlisting}

\xsubsection{objdepend}{Embedded File Dependency Information}

Since NASM 2.13.02, \code{obj} files contain embedded dependency file
information. To suppress the generation of dependencies, use

\begin{lstlisting}
%pragma obj nodepend
\end{lstlisting}

\xsection{win32fmt}{\codeindex{win32}: Microsoft Win32 Object Files}

The \code{win32} output format generates Microsoft Win32 object files,
suitable for passing to Microsoft linkers such as \emph{Visual C++}.
Note that Borland Win32 compilers do not use this format, but use
\code{obj} instead (see \nref{objfmt}).

\code{win32} provides a default output file-name extension of \code{.obj}.

Note that although Microsoft say that Win32 object files follow the
COFF (Common Object File Format) standard, the object files produced
by Microsoft Win32 compilers are not compatible with COFF linkers such
as DJGPP's, and vice versa. This is due to a difference of opinion over
the precise semantics of PC-relative relocations. To produce COFF files
suitable for DJGPP, use NASM's \code{coff} output format; conversely,
the \code{coff} format does not produce object files that Win32 linkers
can generate correct output from.

\xsubsection{win32sect}{\code{win32} Extensions to the \code{SECTION} Directive}
\index{SECTION!win32 extensions to}

Like the \code{obj} format, \code{win32} allows you to specify additional
information on the \code{SECTION} directive line, to control the type
and properties of sections you declare. Section types and properties
are generated automatically by NASM for the \textindex{standard section names}
\code{.text}, \code{.data} and \code{.bss}, but may still be overridden by
these qualifiers.

The available qualifiers are:

\begin{itemize}
    \item{\code{code}, or equivalently \code{text}, defines the section
        to be a code section. This marks the section as readable and
        executable, but not writable, and also indicates to the linker
        that the type of the section is code.}

    \item{\code{data} and \code{bss} define the section to be a data
        section, analogously to \code{code}. Data sections are marked
        as readable and writable, but not executable. \code{data}
        declares an initialized data section, whereas \code{bss} declares
        an uninitialized data section.}

    \item{\code{rdata} declares an initialized data section that is
        readable but not writable. Microsoft compilers use this section
        to place constants in it.}

    \item{\code{info} defines the section to be an \textindex{informational section},
        which is not included in the executable file by the linker, but may
        (for example) pass information \emph{to} the linker. For example,
        declaring an \code{info}-type section called \codeindex{.drectve} causes
        the linker to interpret the contents of the section as command-line
        options.}

    \item{\code{align=}, used with a trailing number as in \code{obj}, gives the
        \index{section alignment!in win32} \index{alignment!in win32 sections}
        alignment requirements of the section. The maximum you may
        specify is 64: the Win32 object file format contains no means to
        request a greater section alignment than this. If alignment is not
        explicitly specified, the defaults are 16-byte alignment for code
        sections, 8-byte alignment for rdata sections and 4-byte alignment
        for data (and BSS) sections.
        Informational sections get a default alignment of 1 byte (no
        alignment), though the value does not matter.}
\end{itemize}

The defaults assumed by NASM if you do not specify the above
qualifiers are:

\begin{lstlisting}
section .text   code    align=16
section .data   data    align=4
section .rdata  rdata   align=8
section .bss    bss     align=4
\end{lstlisting}

Any other section name is treated by default like \code{.text}.

\xsubsection{win32safeseh}{\code{win32} Safe Structured Exception Handling}

Among other improvements in Windows XP SP2 and Windows Server 2003
Microsoft has introduced concept of "safe structured exception
handling." General idea is to collect handlers' entry points in
designated read-only table and have alleged entry point verified
against this table prior exception control is passed to the handler. In
order for an executable module to be equipped with such "safe exception
handler table," all object modules on linker command line has to comply
with certain criteria. If one single module among them does not, then
the table in question is omitted and above mentioned run-time checks
will not be performed for application in question. Table omission is by
default silent and therefore can be easily overlooked. One can instruct
linker to refuse to produce binary without such table by passing
\code{/safeseh} command line option.

Without regard to this run-time check merits it's natural to expect
NASM to be capable of generating modules suitable for \code{/safeseh}
linking. From developer's viewpoint the problem is two-fold:

\begin{itemize}
    \item{how to adapt modules not deploying exception handlers of their own;}
    \item{how to adapt/develop modules utilizing custom exception handling.}
\end{itemize}

Former can be easily achieved with any NASM version by adding following
line to source code:

\begin{lstlisting}
$@feat.00 equ 1
\end{lstlisting}

As of version 2.03 NASM adds this absolute symbol automatically. If
it's not already present to be precise. I.e. if for whatever reason
developer would choose to assign another value in source file, it would
still be perfectly possible.

Registering custom exception handler on the other hand requires certain
"magic." As of version 2.03 additional directive is implemented,
\code{safeseh}, which instructs the assembler to produce appropriately
formatted input data for above mentioned "safe exception handler
table." Its typical use would be:

\begin{lstlisting}
section .text
extern  _MessageBoxA@16
%if     __NASM_VERSION_ID__ >= 0x02030000
safeseh handler                 ; register handler as "safe handler"
%endif
handler:
        push    DWORD 1         ; MB_OKCANCEL
        push    DWORD caption
        push    DWORD text
        push    DWORD 0
        call    _MessageBoxA@16
        sub     eax,1           ; incidentally suits as return value
                                ; for exception handler
        ret
global  _main
_main:
        push    DWORD handler
        push    DWORD [fs:0]
        mov     DWORD [fs:0],esp ; engage exception handler
        xor     eax,eax
        mov     eax,DWORD[eax]   ; cause exception
        pop     DWORD [fs:0]     ; disengage exception handler
        add     esp,4
        ret
text:   db      'OK to rethrow, CANCEL to generate core dump',0
caption:db      'SEGV',0

section .drectve info
        db      '/defaultlib:user32.lib /defaultlib:msvcrt.lib '
\end{lstlisting}

As you might imagine, it's perfectly possible to produce .exe binary
with "safe exception handler table" and yet engage unregistered
exception handler. Indeed, handler is engaged by simply manipulating
\code{[fs:0]} location at run-time, something linker has no power over,
run-time that is. It should be explicitly mentioned that such failure
to register handler's entry point with \code{safeseh} directive has
undesired side effect at run-time. If exception is raised and
unregistered handler is to be executed, the application is abruptly
terminated without any notification whatsoever. One can argue that
system could  at least have logged some kind "non-safe exception
handler in x.exe at address n" message in event log, but no, literally
no notification is provided and user is left with no clue on what
caused application failure.

Finally, all mentions of linker in this paragraph refer to Microsoft
linker version 7.x and later. Presence of \code{@feat.00} symbol and input
data for "safe exception handler table" causes no backward
incompatibilities and "safeseh" modules generated by NASM 2.03 and
later can still be linked by earlier versions or non-Microsoft linkers.

\xsubsection{codeview}{Debugging formats for Windows}
\index{Windows debugging formats}

The \code{win32} and \code{win64} formats support the Microsoft CodeView
debugging format. Currently CodeView version 8 format is supported
(\i\c{cv8}), but newer versions of the CodeView debugger should be
able to handle this format as well.

\xsection{win64fmt}{\codeindex{win64}: Microsoft Win64 Object Files}

The \code{win64} output format generates Microsoft Win64 object files,
which is nearly 100\% identical to the \code{win32} object format
(\nref{win32fmt}) with the exception that it is meant to target
64-bit code and the x86-64 platform altogether. This object file is used
exactly the same as the \code{win32} object format, in NASM, with regard to this exception.

\xsubsection{win64pic}{\code{win64}: Writing Position-Independent Code}

While \code{REL} takes good care of RIP-relative addressing, there is one
aspect that is easy to overlook for a Win64 programmer: indirect
references. Consider a switch dispatch table:

\begin{lstlisting}
        jmp     qword [dsptch+rax*8]
        ...
dsptch: dq      case0
        dq      case1
        ...
\end{lstlisting}

Even a novice Win64 assembler programmer will soon realize that the code
is not 64-bit savvy. Most notably linker will refuse to link it with

\begin{lstlisting}
'ADDR32' relocation to '.text' invalid without /LARGEADDRESSAWARE:NO
\end{lstlisting}

So [s]he will have to split jmp instruction as following:

\begin{lstlisting}
        lea     rbx,[rel dsptch]
        jmp     qword [rbx+rax*8]
\end{lstlisting}

What happens behind the scene is that effective address in \code{lea} is
encoded relative to instruction pointer, or in perfectly position-independent
manner. But this is only part of the problem! Trouble is that in .dll context
\code{caseN} relocations will make their way to the final module and might
have to be adjusted at .dll load time. To be specific when it can't be loaded
at preferred address. And when this occurs, pages with such relocations will
be rendered private to current process, which kind of undermines the idea
of sharing .dll. But no worry, it's trivial to fix:

\begin{lstlisting}
        lea     rbx,[rel dsptch]
        add     rbx,[rbx+rax*8]
        jmp     rbx
        ...
dsptch: dq      case0-dsptch
        dq      case1-dsptch
        ...
\end{lstlisting}

NASM version 2.03 and later provides another alternative, \code{wrt
..imagebase} operator, which returns offset from base address of the
current image, be it .exe or .dll module, therefore the name. For those
acquainted with PE-COFF format base address denotes start of
\code{IMAGE\_DOS\_HEADER} structure. Here is how to implement switch with
these image-relative references:

\begin{lstlisting}
        lea     rbx,[rel dsptch]
        mov     eax,[rbx+rax*4]
        sub     rbx,dsptch wrt ..imagebase
        add     rbx,rax
        jmp     rbx
        ...
dsptch: dd      case0 wrt ..imagebase
        dd      case1 wrt ..imagebase
\end{lstlisting}

One can argue that the operator is redundant. Indeed, snippet before
last works just fine with any NASM version and is not even Windows
specific... The real reason for implementing \code{wrt ..imagebase} will
become apparent in next paragraph.

It should be noted that \code{wrt ..imagebase} is defined as 32-bit
operand only:

\begin{lstlisting}
dd      label wrt ..imagebase           ; ok
dq      label wrt ..imagebase           ; bad
mov     eax,label wrt ..imagebase       ; ok
mov     rax,label wrt ..imagebase       ; bad
\end{lstlisting}

\xsubsection{win64seh}{\code{win64}: Structured Exception Handling}

Structured exception handing in Win64 is completely different matter
from Win32. Upon exception program counter value is noted, and
linker-generated table comprising start and end addresses of all the
functions [in given executable module] is traversed and compared to the
saved program counter. Thus so called \code{UNWIND\_INFO} structure is
identified. If it's not found, then offending subroutine is assumed to
be "leaf" and just mentioned lookup procedure is attempted for its
caller. In Win64 leaf function is such function that does not call any
other function \emph{nor} modifies any Win64 non-volatile registers,
including stack pointer. The latter ensures that it's possible to
identify leaf function's caller by simply pulling the value from the
top of the stack.

While majority of subroutines written in assembler are not calling any
other function, requirement for non-volatile registers' immutability
leaves developer with not more than 7 registers and no stack frame,
which is not necessarily what [s]he counted with. Customarily one would
meet the requirement by saving non-volatile registers on stack and
restoring them upon return, so what can go wrong? If [and only if] an
exception is raised at run-time and no \code{UNWIND\_INFO} structure is
associated with such "leaf" function, the stack unwind procedure will
expect to find caller's return address on the top of stack immediately
followed by its frame. Given that developer pushed caller's
non-volatile registers on stack, would the value on top point at some
code segment or even addressable space? Well, developer can attempt
copying caller's return address to the top of stack and this would
actually work in some very specific circumstances. But unless developer
can guarantee that these circumstances are always met, it's more
appropriate to assume worst case scenario, i.e. stack unwind procedure
going berserk. Relevant question is what happens then? Application is
abruptly terminated without any notification whatsoever. Just like in
Win32 case, one can argue that system could at least have logged
"unwind procedure went berserk in x.exe at address n" in event log, but
no, no trace of failure is left.

Now, when we understand significance of the \code{UNWIND\_INFO} structure,
let's discuss what's in it and/or how it's processed. First of all it
is checked for presence of reference to custom language-specific
exception handler. If there is one, then it's invoked. Depending on the
return value, execution flow is resumed (exception is said to be
"handled"), \emph{or} rest of \code{UNWIND\_INFO} structure is processed as
following. Beside optional reference to custom handler, it carries
information about current callee's stack frame and where non-volatile
registers are saved. Information is detailed enough to be able to
reconstruct contents of caller's non-volatile registers upon call to
current callee. And so caller's context is reconstructed, and then
unwind procedure is repeated, i.e. another \code{UNWIND\_INFO} structure is
associated, this time, with caller's instruction pointer, which is then
checked for presence of reference to language-specific handler, etc.
The procedure is recursively repeated till exception is handled. As
last resort system "handles" it by generating memory core dump and
terminating the application.

As for the moment of this writing NASM unfortunately does not
facilitate generation of above mentioned detailed information about
stack frame layout. But as of version 2.03 it implements building
blocks for generating structures involved in stack unwinding. As
simplest example, here is how to deploy custom exception handler for
leaf function:

\begin{lstlisting}
default rel
section .text
extern  MessageBoxA
handler:
        sub     rsp,40
        mov     rcx,0
        lea     rdx,[text]
        lea     r8,[caption]
        mov     r9,1            ; MB_OKCANCEL
        call    MessageBoxA
        sub     eax,1           ; incidentally suits as return value
                                ; for exception handler
        add     rsp,40
        ret
global  main
main:
        xor     rax,rax
        mov     rax,QWORD[rax]  ; cause exception
        ret
main_end:
text:   db      'OK to rethrow, CANCEL to generate core dump',0
caption:db      'SEGV',0

section .pdata  rdata align=4
        dd      main wrt ..imagebase
        dd      main_end wrt ..imagebase
        dd      xmain wrt ..imagebase
section .xdata  rdata align=8
xmain:  db      9,0,0,0
        dd      handler wrt ..imagebase
section .drectve info
        db      '/defaultlib:user32.lib /defaultlib:msvcrt.lib '
\end{lstlisting}

What you see in \code{.pdata} section is element of the "table comprising
start and end addresses of function" along with reference to associated
\code{UNWIND\_INFO} structure. And what you see in \code{.xdata} section is
\code{UNWIND\_INFO} structure describing function with no frame, but with
designated exception handler. References are \emph{required} to be
image-relative (which is the real reason for implementing \code{wrt
..imagebase} operator). It should be noted that \code{rdata align=n}, as
well as \code{wrt ..imagebase}, are optional in these two segments'
contexts, i.e. can be omitted. Latter means that \emph{all} 32-bit
references, not only above listed required ones, placed into these two
segments turn out image-relative. Why is it important to understand?
Developer is allowed to append handler-specific data to \code{UNWIND\_INFO}
structure, and if [s]he adds a 32-bit reference, then [s]he will have
to remember to adjust its value to obtain the real pointer.

As already mentioned, in Win64 terms leaf function is one that does not
call any other function \emph{nor} modifies any non-volatile register,
including stack pointer. But it's not uncommon that assembler
programmer plans to utilize every single register and sometimes even
have variable stack frame. Is there anything one can do with bare
building blocks? I.e. besides manually composing fully-fledged
\code{UNWIND\_INFO} structure, which would surely be considered
error-prone? Yes, there is. Recall that exception handler is called
first, before stack layout is analyzed. As it turned out, it's
perfectly possible to manipulate current callee's context in custom
handler in manner that permits further stack unwinding. General idea is
that handler would not actually "handle" the exception, but instead
restore callee's context, as it was at its entry point and thus mimic
leaf function. In other words, handler would simply undertake part of
unwinding procedure. Consider following example:

\begin{lstlisting}
function:
        mov     rax,rsp         ; copy rsp to volatile register
        push    r15             ; save non-volatile registers
        push    rbx
        push    rbp
        mov     r11,rsp         ; prepare variable stack frame
        sub     r11,rcx
        and     r11,-64
        mov     QWORD[r11],rax  ; check for exceptions
        mov     rsp,r11         ; allocate stack frame
        mov     QWORD[rsp],rax  ; save original rsp value
magic_point:
        ...
        mov     r11,QWORD[rsp]  ; pull original rsp value
        mov     rbp,QWORD[r11-24]
        mov     rbx,QWORD[r11-16]
        mov     r15,QWORD[r11-8]
        mov     rsp,r11         ; destroy frame
        ret
\end{lstlisting}

The keyword is that up to \code{magic\_point} original \code{rsp} value
remains in chosen volatile register and no non-volatile register,
except for \code{rsp}, is modified. While past \code{magic\_point}
\code{rsp} remains constant till the very end of the \code{function}.
In this case custom language-specific exception handler would look like this:

\begin{lstlisting}
EXCEPTION_DISPOSITION
handler(EXCEPTION_RECORD *rec, ULONG64 frame,
        CONTEXT *context, DISPATCHER_CONTEXT *disp)
{
    ULONG64 *rsp;

    if (context->Rip < (ULONG64)magic_point)
        rsp = (ULONG64 *)context->Rax;
    else {
        rsp = ((ULONG64 **)context->Rsp)[0];
        context->Rbp = rsp[-3];
        context->Rbx = rsp[-2];
        context->R15 = rsp[-1];
    }
    context->Rsp = (ULONG64)rsp;

    memcpy(disp->ContextRecord, context, sizeof(CONTEXT));
    RtlVirtualUnwind(UNW_FLAG_NHANDLER, disp->ImageBase,
                     dips->ControlPc, disp->FunctionEntry,
                     disp->ContextRecord,
                     &disp->HandlerData,
                     &disp->EstablisherFrame,
                     NULL);

    return ExceptionContinueSearch;
}
\end{lstlisting}

As custom handler mimics leaf function, corresponding \code{UNWIND\_INFO}
structure does not have to contain any information about stack frame
and its layout.

\xsection{cofffmt}{\codeindex{coff}: \textindexlc{Common Object File Format}}

The \code{coff} output type produces \code{COFF} object files suitable for
linking with the \textindex{DJGPP} linker.

\code{coff} provides a default output file-name extension of \code{.o}.

The \code{coff} format supports the same extensions to the \code{SECTION}
directive as \code{win32} does, except that the \code{align} qualifier and
the \code{info} section type are not supported.

\xsection{machofmt}{\codeindex{macho32} and \codeindex{macho64}:
\textindexlc{Mach Object File Format}}
\index{Mach-O}

The \code{macho32}, \code{macho64} output formts produces Mach-O
object files suitable for linking with the \textindex{MacOS X} linker.
\codeindex{macho} is a synonym for \code{macho32}.

\code{macho} provides a default output file-name extension of \code{.o}.

\xsubsection{machosect}{\code{macho} extensions to the \code{SECTION} Directive}
\index{SECTION!macho extensions to}

The \code{macho} output format specifies section names in the format
"\emph{segment}\code{,}\emph{section}". No spaces are allowed around the
comma. The following flags can also be specified:

\begin{itemize}
    \item{\code{data} - this section contains initialized data items}
    \item{\code{text} - this section contains code exclusively}
    \item{\code{mixed} - this section contains both code and data}
    \item{\code{bss} - this section is uninitialized and filled with zero}
    \item{\code{zerofill} - same as \code{bss}}
    \item{\code{no\_dead\_strip} - inhibit dead code stripping for this section}
    \item{\code{live\_support} - set the live support flag for this section}
    \item{\code{strip\_static\_syms} - strip static symbols for this section}
    \item{\code{align=}\emph{alignment} - specify section alignment}
\end{itemize}

The default is \code{data}, unless the section name is \code{\_\_text} or
\code{\_\_bss} in which case the default is \code{text} or \code{bss},
respectively.

For compatibility with other Unix platforms, the following standard
names are also supported:

\begin{lstlisting}
.text   = __TEXT,__text     text
.rodata = __DATA,__const    data
.data   = __DATA,__data     data
.bss    = __DATA,__bss      bss
\end{lstlisting}

If the \code{.rodata} section contains no relocations, it is instead put
into the \code{\_\_TEXT,\_\_const} section unless this section has already
been specified explicitly. However, it is probably better to specify
\code{\_\_TEXT,\_\_const} and \code{\_\_DATA,\_\_const} explicitly as appropriate.

\xsubsection{machotls}{\textindexlc{Thread Local Storage in Mach-O}\index{TLS}:
\code{macho} special symbols and \codeindex{WRT}}

Mach-O defines the following special symbols that can be used on the
right-hand side of the \code{WRT} operator:

\begin{itemize}
    \item{\code{..tlvp} is used to specify access to thread-local storage.}
    \item{\code{..gotpcrel} is used to specify references to the Global Offset Table.
        The GOT is supported in the \code{macho64} format only.}
\end{itemize}

\xsubsection{macho-ssvs}{\code{macho} specfic directive \codeindex{subsections\_via\_symbols}}

The directive \code{subsections\_via\_symbols} sets the
\code{MH\_SUBSECTIONS\_VIA\_SYMBOLS} flag in the Mach-O header, which tells
the linker that the symbols in the file matches the conventions
required to allow for link-time dead code elimination.

This directive takes no arguments.

This is a macro implemented as a \code{\%pragma}. It can also be
specified in its \code{\%pragma} form, in which case it will not affect
non-Mach-O builds of the same source code:

\begin{lstlisting}
%pragma macho subsections_via_symbols
\end{lstlisting}

\xsubsection{macho-ssvs}{\code{macho} specfic directive \codeindex{no\_dead\_strip}}

The directive \code{no\_dead\_strip} sets the Mach-O \code{SH\_NO\_DEAD\_STRIP}
section flag on the section containing a a specific symbol. This directive takes
a list of symbols as its arguments.

This is a macro implemented as a \code{\%pragma}.  It can also be
specified in its \code{\%pragma} form, in which case it will not affect
non-Mach-O builds of the same source code:

\begin{lstlisting}
%pragma macho no_dead_strip symbol...
\end{lstlisting}

\xsection{elffmt}{\codeindex{elf32}, \codeindex{elf64}, \codeindex{elfx32}:
\textindexlc{Executable and Linkable Format} Object Files}
\index{ELF}\index{linux!elf}

The \code{elf32}, \code{elf64} and \code{elfx32} output formats generate
\code{ELF32} and \code{ELF64} (Executable and Linkable Format) object files,
as used by Linux as well as \textindex{Unix System V}, including
\textindex{Solaris x86}, \textindex{UnixWare} and \textindex{SCO Unix}.
\code{elf} provides a default output file-name extension of \code{.o}.
\code{elf} is a synonym for \code{elf32}.

The \code{elfx32} format is used for the \textindex{x32} ABI, which is
a 32-bit ABI with the CPU in 64-bit mode.

\xsubsection{abisect}{ELF specific directive \codeindex{osabi}}

The ELF header specifies the application binary interface for the
target operating system (OSABI). This field can be set by using the
\code{osabi} directive with the numeric value (0-255) of the target
system. If this directive is not used, the default value will be "UNIX
System V ABI" (0) which will work on most systems which support ELF.

\xsubsection{elfsect}{\code{elf} extensions to the \code{SECTION} Directive}
\index{SECTION!elf extensions to}

Like the \code{obj} format, \code{elf} allows you to specify additional
information on the \code{SECTION} directive line, to control the type
and properties of sections you declare. Section types and properties
are generated automatically by NASM for the \textindexlc{standard section
names}, but may still be overridden by these qualifiers.

The available qualifiers are:

\begin{itemize}
    \item{\codeindex{alloc} defines the section to be one which is loaded into
        memory when the program is run. \codeindex{noalloc} defines it to be one
        which is not, such as an informational or comment section.}

    \item{\codeindex{exec} defines the section to be one which should have execute
        permission when the program is run. \codeindex{noexec} defines it as one
        which should not.}

    \item{\codeindex{write} defines the section to be one which should be writable
        when the program is run. \codeindex{nowrite} defines it as one which should
        not.}

    \item{\codeindex{progbits} defines the section to be one with explicit contents
        stored in the object file: an ordinary code or data section, for
        example, \codeindex{nobits} defines the section to be one with no explicit
        contents given, such as a BSS section.}

    \item{\code{align=}, used with a trailing number as in \code{obj}, gives the
        \index{section alignment!in elf}\index{alignment!in elf sections}alignment
        requirements of the section.}

    \item{\codeindex{tls} defines the section to be one which contains
        thread local variables.}
\end{itemize}

The defaults assumed by NASM if you do not specify the above
qualifiers are:
\indexcode{.text} \indexcode{.rodata} \indexcode{.lrodata}
\indexcode{.data} \indexcode{.ldata} \indexcode{.bss}
\indexcode{.lbss} \indexcode{.tdata} \indexcode{.tbss}
\indexcode{.comment}

\begin{lstlisting}
section .text    progbits  alloc   exec    nowrite  align=16
section .rodata  progbits  alloc   noexec  nowrite  align=4
section .lrodata progbits  alloc   noexec  nowrite  align=4
section .data    progbits  alloc   noexec  write    align=4
section .ldata   progbits  alloc   noexec  write    align=4
section .bss     nobits    alloc   noexec  write    align=4
section .lbss    nobits    alloc   noexec  write    align=4
section .tdata   progbits  alloc   noexec  write    align=4    tls
section .tbss    nobits    alloc   noexec  write    align=4    tls
section .comment progbits  noalloc noexec  nowrite  align=1
section other    progbits  alloc   noexec  nowrite  align=1
\end{lstlisting}

(Any section name other than those in the above table is treated by
default like \code{other} in the above. Please note that section
names are case sensitive.)

\xsubsection{elfwrt}{\textindexlc{Position-Independent Code}: \code{elf}
Special Symbols and \codeindex{WRT}}
\index{PIC}

Since \code{ELF} does not support segment-base references, the \code{WRT}
operator is not used for its normal purpose; therefore NASM's \code{elf}
output format makes use of \code{WRT} for a different purpose, namely the
PIC-specific \index{relocations!PIC-specific}relocation types.

\code{elf} defines five special symbols which you can use as the
right-hand side of the \code{WRT} operator to obtain PIC relocation
types. They are \codeindex{..gotpc}, \codeindex{..gotoff}, \codeindex{..got},
\codeindex{..plt} and \codeindex{..sym}. Their functions are summarized here:

\begin{itemize}
    \item{Referring to the symbol marking the global offset table base
        using \code{wrt ..gotpc} will end up giving the distance from the
        beginning of the current section to the global offset table.
        (\codeindex{\_GLOBAL\_OFFSET\_TABLE\_} is the standard symbol name
        used to refer to the \textindex{GOT}.) So you would then need to add
        \codeindex{\$\$} to the result to get the real address of the GOT.}

    \item{Referring to a location in one of your own sections using
        \code{wrt ..gotoff} will give the distance from the beginning of
        the GOT to the specified location, so that adding on the address
        of the GOT would give the real address of the location you wanted.}

    \item{Referring to an external or global symbol using \code{wrt ..got}
        causes the linker to build an entry \emph{in} the GOT containing the
        address of the symbol, and the reference gives the distance from the
        beginning of the GOT to the entry; so you can add on the address of
        the GOT, load from the resulting address, and end up with the
        address of the symbol.}

    \item{Referring to a procedure name using \code{wrt ..plt} causes the
        linker to build a \textindex{procedure linkage table} entry for the symbol,
        and the reference gives the address of the \textindex{PLT} entry. You can
        only use this in contexts which would generate a PC-relative
        relocation normally (i.e. as the destination for \code{CALL} or
        \code{JMP}), since ELF contains no relocation type to refer to PLT
        entries absolutely.}

    \item{Referring to a symbol name using \code{wrt ..sym} causes NASM to
        write an ordinary relocation, but instead of making the relocation
        relative to the start of the section and then adding on the offset
        to the symbol, it will write a relocation record aimed directly at
        the symbol in question. The distinction is a necessary one due to a
        peculiarity of the dynamic linker.}
\end{itemize}

A fuller explanation of how to use these relocation types to write
shared libraries entirely in NASM is given in \nref{picdll}.

\xsubsection{elftls}{\textindexlc{Thread Local Storage in ELF}:
\code{elf} Special Symbols and \codeindex{WRT}}
\index{TLS}

In ELF32 mode, referring to an external or global symbol using
\code{wrt ..tlsie}\indexcode{..tlsie} causes the linker to build
an entry \emph{in} the GOT containing the
offset of the symbol within the TLS block, so you can access the value
of the symbol with code such as:

\begin{lstlisting}
mov eax,[tid wrt ..tlsie]
mov [gs:eax],ebx
\end{lstlisting}

In ELF64 or ELFx32 mode, referring to an external or global symbol using
\code{wrt ..gottpoff}\indexcode{..gottpoff} causes the linker to build an
entry \emph{in} the GOT containing the offset of the symbol within the TLS
block, so you can access the value of the symbol with code such as:

\begin{lstlisting}
mov rax,[rel tid wrt ..gottpoff]
mov rcx,[fs:rax]
\end{lstlisting}

\xsubsection{elfglob}{\code{elf} Extensions to the \code{GLOBAL} Directive}
\index{GLOBAL!elf extensions to}

\code{ELF} object files can contain more information about a global symbol
than just its address: they can contain the \index{symbol sizes!specifying}
\index{size!of symbols}size of the symbol and its \index{symbol types!specifying}
\index{type!of symbols}type as well. These are not merely debugger conveniences,
but are actually necessary when the program being written is a
\textindexlc{shared library}. NASM therefore supports some extensions to the
\code{GLOBAL} directive, allowing you to specify these features.

You can specify whether a global variable is a function or a data
object by suffixing the name with a colon and the word
\codeindex{function} or \codeindex{data}. (\codeindex{object} is
a synonym for \code{data}.) For example:

\begin{lstlisting}
global  hashlookup:function, hashtable:data
\end{lstlisting}

exports the global symbol \code{hashlookup} as a function and
\code{hashtable} as a data object.

Optionally, you can control the ELF visibility of the symbol. Just
add one of the visibility keywords: \codeindex{default},
\codeindex{internal}, \codeindex{hidden}, or \codeindex{protected}.
The default is \code{default} of course. For example, to make
\code{hashlookup} hidden:

\begin{lstlisting}
global  hashlookup:function hidden
\end{lstlisting}

You can also specify the size of the data associated with the
symbol, as a numeric expression (which may involve labels, and even
forward references) after the type specifier. Like this:

\begin{lstlisting}
global  hashtable:data (hashtable.end - hashtable)

hashtable:
        db this,that,theother   ; some data here
.end:
\end{lstlisting}

This makes NASM automatically calculate the length of the table and
place that information into the \code{ELF} symbol table.

Declaring the type and size of global symbols is necessary when
writing shared library code. For more information, see
\nref{picglobal}.

\xsubsection{elfcomm}{\code{elf} Extensions to the \code{COMMON} Directive}
\index{COMMON!elf extensions to}

\code{ELF} also allows you to specify alignment requirements
\index{common variables!alignment in elf}
\index{alignment!of elf common variables} on common variables.
This is done by putting a number (which must be a power of two)
after the name and size of the common variable, separated (as usual)
by a colon. For example, an array of doublewords would benefit from
4-byte alignment:

\begin{lstlisting}
common dwordarray 128:4
\end{lstlisting}

This declares the total size of the array to be 128 bytes, and
requires that it be aligned on a 4-byte boundary.

\xsubsection{elf16}{16-bit code and ELF}
\index{ELF!16-bit code and}

The \code{ELF32} specification doesn't provide relocations for 8- and
16-bit values, but the GNU \code{ld} linker adds these as an extension.
NASM can generate GNU-compatible relocations, to allow 16-bit code to
be linked as ELF using GNU \code{ld}. If NASM is used with the
\code{-w+gnu-elf-extensions} option, a warning is issued when one of
these relocations is generated.

\xsubsection{elfdbg}{Debug formats and ELF}
\index{ELF!Debug formats}

ELF provides debug information in \code{STABS} and \code{DWARF} formats.
Line number information is generated for all executable sections, but please
note that only the ".text" section is executable by default.

\xsection{aoutfmt}{\codeindex{aout}: Linux \code{a.out} Object Files}
\index{a.out!Linux version}
\index{linux!a.out}

The \code{aout} format generates \code{a.out} object files, in the
form used by early Linux systems (current Linux systems use ELF, see
\nref{elffmt}.) These differ from other \code{a.out} object
files in that the magic number in the first four bytes of the file is
different; also, some implementations of \code{a.out}, for example
NetBSD's, support position-independent code, which Linux's
implementation does not.

\code{a.out} provides a default output file-name extension of \code{.o}.

\code{a.out} is a very simple object format. It supports no special
directives, no special symbols, no use of \code{SEG} or \code{WRT}, and no
extensions to any standard directives. It supports only the three
\textindexlc{standard section names} \codeindex{.text}, \codeindex{.data}
and \codeindex{.bss}.

\xsection{aoutfmt}{\codeindex{aoutb}: \textindex{NetBSD}/\textindex{FreeBSD}/\textindex{OpenBSD}
\code{a.out} Object Files}
\index{a.out!BSD version}

The \code{aoutb} format generates \code{a.out} object files, in the form
used by the various free \code{BSD Unix} clones, \code{NetBSD}, \code{FreeBSD}
and \code{OpenBSD}. For simple object files, this object format is exactly
the same as \code{aout} except for the magic number in the first four bytes
of the file. However, the \code{aoutb} format supports
\index{PIC}\textindexlc{position-independent code} in the same way as the
\code{elf} format, so you can use it to write \code{BSD}
\textindexlc{shared libraries}.

\code{aoutb} provides a default output file-name extension of \code{.o}.

\code{aoutb} supports no special directives, no special symbols, and
only the three \textindexlc{standard section names} \codeindex{.text},
\codeindex{.data} and \codeindex{.bss}. However, it also supports the same
use of \codeindex{WRT} as \code{elf} does, to provide position-independent
code relocation types. See \nref{elfwrt} for full documentation
of this feature.

\code{aoutb} also supports the same extensions to the \code{GLOBAL}
directive as \code{elf} does: see \nref{elfglob} for
documentation of this.

\xsection{as86fmt}{\code{as86}: \textindex{Minix}/Linux \codeindex{as86} Object Files}
\index{linux!as86}

The Minix/Linux 16-bit assembler \code{as86} has its own non-standard
object file format. Although its companion linker \codeindex{ld86}
produces something close to ordinary \code{a.out} binaries as output,
the object file format used to communicate between \code{as86} and
\code{ld86} is not itself \code{a.out}.

NASM supports this format, just in case it is useful, as \code{as86}.
\code{as86} provides a default output file-name extension of \code{.o}.

\code{as86} is a very simple object format (from the NASM user's point
of view). It supports no special directives, no use of \code{SEG} or
\code{WRT}, and no extensions to any standard directives. It supports
only the three \textindexlc{standard section names} \codeindex{.text},
\codeindex{.data} and \codeindex{.bss}. The only special symbol supported
is \code{..start}.

\xsection{rdffmt}{\index{RDOFF}\codeindex{rdf}: \textindexlc{Relocatable Dynamic
Object File Format}}

The \code{rdf} output format produces \code{RDOFF} object files.
\code{RDOFF} (Relocatable Dynamic Object File Format) is a home-grown
object-file format, designed alongside NASM itself and reflecting in
its file format the internal structure of the assembler.

\code{RDOFF} is not used by any well-known operating systems. Those
writing their own systems, however, may well wish to use \code{RDOFF}
as their object format, on the grounds that it is designed primarily
for simplicity and contains very little file-header bureaucracy.

The Unix NASM archive, and the DOS archive which includes sources,
both contain an \index{rdoff subdirectory}\code{rdoff} subdirectory
holding a set of RDOFF utilities: an RDF linker, an \code{RDF}
static-library manager, an RDF file dump utility, and a program
which will load and execute an RDF executable under Linux.

\code{rdf} supports only the \index{standard section names}
\codeindex{.text}, \codeindex{.data} and \codeindex{.bss}.

\xsubsection{rdflib}{Requiring a Library: The \codeindex{LIBRARY} Directive}

\code{RDOFF} contains a mechanism for an object file to demand a given
library to be linked to the module, either at load time or run time.
This is done by the \code{LIBRARY} directive, which takes one argument
which is the name of the module:

\begin{lstlisting}
library mylib.rdl
\end{lstlisting}

\xsubsection{rdfmod}{Specifying a Module Name: The \codeindex{MODULE} Directive}

Special \code{RDOFF} header record is used to store the name of the module.
It can be used, for example, by run-time loader to perform dynamic
linking. \code{MODULE} directive takes one argument which is the name
of current module:

\begin{lstlisting}
module  mymodname
\end{lstlisting}

Note that when you statically link modules and tell linker to strip
the symbols from output file, all module names will be stripped too.
To avoid it, you should start module names with \index{\$!prefix}\code{\$},
like:

\begin{lstlisting}
module $kernel.core
\end{lstlisting}

\xsubsection{rdfglob}{\code{rdf} Extensions to the \code{GLOBAL} Directive}
\index{GLOBAL!rdf extensions to}

\code{RDOFF} global symbols can contain additional information needed by
the static linker. You can mark a global symbol as exported, thus
telling the linker do not strip it from target executable or library
file. Like in \code{ELF}, you can also specify whether an exported symbol
is a procedure (function) or data object.

Suffixing the name with a colon and the word \codeindex{export} you make the
symbol exported:

\begin{lstlisting}
global  sys_open:export
\end{lstlisting}

To specify that exported symbol is a procedure (function), you add the
word \codeindex{proc} or \codeindex{function} after declaration:

\begin{lstlisting}
global  sys_open:export proc
\end{lstlisting}

Similarly, to specify exported data object, add the word \codeindex{data}
or \codeindex{object} to the directive:

\begin{lstlisting}
global  kernel_ticks:export data
\end{lstlisting}

\xsubsection{rdfimpt}{\code{rdf} Extensions to the \code{EXTERN} Directive}
\index{EXTERN!rdf extensions to}

By default the \code{EXTERN} directive in \code{RDOFF} declares a "pure external"
symbol (i.e. the static linker will complain if such a symbol is not resolved).
To declare an "imported" symbol, which must be resolved later during a dynamic
linking phase, \code{RDOFF} offers an additional \code{import} modifier. As in
\code{GLOBAL}, you can also specify whether an imported symbol is a procedure
(function) or data object. For example:

\begin{lstlisting}
library $libc
extern  _open:import
extern  _printf:import proc
extern  _errno:import data
\end{lstlisting}

Here the directive \code{LIBRARY} is also included, which gives the dynamic linker
a hint as to where to find requested symbols.

\xsection{dbgfmt}{\codeindex{dbg}: Debugging Format}

The \code{dbg} format does not output an object file as such; instead,
it outputs a text file which contains a complete list of all the
transactions between the main body of NASM and the output-format
back end module. It is primarily intended to aid people who want to
write their own output drivers, so that they can get a clearer idea
of the various requests the main program makes of the output driver,
and in what order they happen.

For simple files, one can easily use the \code{dbg} format like this:

\begin{lstlisting}
nasm -f dbg filename.asm
\end{lstlisting}

which will generate a diagnostic file called \code{filename.dbg}.
However, this will not work well on files which were designed for a
different object format, because each object format defines its own
macros (usually user-level forms of directives), and those macros
will not be defined in the \code{dbg} format. Therefore it can be
useful to run NASM twice, in order to do the preprocessing with the
native object format selected:

\begin{lstlisting}
nasm -e -f rdf -o rdfprog.i rdfprog.asm
nasm -a -f dbg rdfprog.i
\end{lstlisting}

This preprocesses \code{rdfprog.asm} into \code{rdfprog.i}, keeping the
\code{rdf} object format selected in order to make sure RDF special
directives are converted into primitive form correctly. Then the
preprocessed source is fed through the \code{dbg} format to generate
the final diagnostic output.

This workaround will still typically not work for programs intended
for \code{obj} format, because the \code{obj}- \code{SEGMENT} and \code{GROUP}
directives have side effects of defining the segment and group names
as symbols; \code{dbg} will not do this, so the program will not
assemble. You will have to work around that by defining the symbols
yourself (using \code{EXTERN}, for example) if you really need to get a
\code{dbg} trace of an \code{obj}-specific source file.

\code{dbg} accepts any section name and any directives at all, and logs
them all to its output file.

\code{dbg} accepts and logs any \c{\%pragma}, but the specific
\c{\%pragma}:

\begin{lstlisting}
%pragma dbg maxdump <size>
\end{lstlisting}

where \code{<size>} is either a number or \code{unlimited}, can be
used to control the maximum size for dumping the full contents of a
\code{rawdata} output object.
