\chapter{The NASM Language}
\label{ch:lang}

\section{Layout of a NASM Source Line}
\label{sec:syntax}

Like most assemblers, each NASM source line contains (unless it
is a macro, a preprocessor directive or an assembler directive: see
\fullref{sec:preproc} and \fullref{sec:directive}) some combination
of the four fields

\begin{lstlisting}
	label:    instruction operands        ; comment
\end{lstlisting}

As usual, most of these fields are optional; the presence or absence
of any combination of a label, an instruction and a comment is allowed.
Of course, the operand field is either required or forbidden by the
presence and nature of the instruction field.

NASM uses backslash (\code{\\}) as the line continuation character;
if a line ends with backslash, the next line is considered to be
a part of the backslash-ended line.

NASM places no restrictions on white space within a line: labels may
have white space before them, or instructions may have no space
before them, or anything. The \textindex{colon} after a label is also
optional. (Note that this means that if you intend to code \code{lodsb}
alone on a line, and type \code{lodab} by accident, then that's still a
valid source line which does nothing but define a label. Running
NASM with the command-line option \index{orphan-labels}\code{-w+orphan-labels}
will cause it to warn you if you define a label alone on a line without
a \textindex{trailing colon}.)

\textindex{Valid characters} in labels are letters, numbers, \code{\_},
\code{\$}, \code{\#}, \code{\@}, \code{~}, \code{.}, and \code{?}.
The only characters which may be used as the \emph{first} character of
an identifier are letters, \code{\.} (with special meaning: see
\fullref{sec:locallab}), \code{\_} and \code{?}.
An identifier may also be prefixed with a \index{\$}\index{prefix}
\code{\$} to indicate that it is intended to be read as an identifier
and not a reserved word; thus, if some other module you are linking with
defines a symbol called \code{eax}, you can refer to \code{\$eax} in NASM
code to distinguish the symbol from the register. Maximum length of
an identifier is 4095 characters.

The instruction field may contain any machine instruction: Pentium
and P6 instructions, FPU instructions, MMX instructions and even
undocumented instructions are all supported. The instruction may be
prefixed by \code{LOCK}, \code{REP}, \code{REPE}/\code{REPZ},
\code{REPNE}/\code{REPNZ}, \code{XACQUIRE}/\code{XRELEASE} or
\code{BND}/\code{NOBND}, in the usual way. Explicit
\index{address-size prefixes}address-size and \textindex{operand-size prefixes}
\codeindex{A16}, \codeindex{A32}, \codeindex{A64}, \codeindex{O16}
and \codeindex{O32}, \codeindex{O64} are provided~-- one example of their
use is given in \fullref{sec:mixsize}. You can also use the name of a
\index{segment override}segment register as an instruction prefix: coding
\code{es mov [bx],ax} is equivalent to coding \code{mov [es:bx],ax}.
We recommend the latter syntax, since it is consistent with other syntactic
features of the language, but for instructions such as \code{LODSB}, which
has no operands and yet can require a segment override, there is no clean
syntactic way to proceed apart from \code{es lodsb}.

An instruction is not required to use a prefix: prefixes such as
\code{CS}, \code{A32}, \code{LOCK} or \code{REPE} can appear on
a line by themselves, and NASM will just generate the prefix bytes.

In addition to actual machine instructions, NASM also supports a
number of pseudo-instructions, described in \k{pseudop}.

Instruction \textindex{operands} may take a number of forms: they can be
registers, described simply by the register name (e.g. \code{ax},
\code{bp}, \code{ebx}, \code{cr0}: NASM does not use the \code{gas}-style
syntax in which register names must be prefixed by a \code{\%} sign),
or they can be \textindex{effective addresses} (see \fullref{sec:effaddr}),
constants (\fullref{sec:const}) or expressions (\fullref{sec:expr}).

For x87 \textindex{floating-point} instructions, NASM accepts a wide
range of syntaxes: you can use two-operand forms like MASM supports,
or you can use NASM's native single-operand forms in most cases.
% Details of
% all forms of each supported instruction are given in
% \fullref{sec:iref}.
For example, you can code:

\begin{lstlisting}
	fadd    st1             ; this sets st0 := st0 + st1
	fadd    st0,st1         ; so does this

	fadd    st1,st0         ; this sets st1 := st1 + st0
	fadd    to st1          ; so does this
\end{lstlisting}

Almost any x87 floating-point instruction that references memory must
use one of the prefixes \codeindex{DWORD}, \codeindex{QWORD} or
\codeindex{TWORD} to indicate what size of \textindex{memory operand}
it refers to.

\section{\htextindex{Pseudo-Instructions}}
\label{sec:pseudop}

Pseudo-instructions are things which, though not real x86 machine
instructions, are used in the instruction field anyway because that's
the most convenient place to put them. The current pseudo-instructions
are \codeindex{DB}, \codeindex{DW}, \codeindex{DD}, \codeindex{DQ},
\codeindex{DT}, \codeindex{DO}, \codeindex{DY} and \codeindex{DZ};
their \textindex{uninitialized} counterparts \codeindex{RESB},
\codeindex{RESW}, \codeindex{RESD}, \codeindex{RESQ},
\codeindex{REST}, \codeindex{RESO}, \codeindex{RESY} and
\codeindex{RESZ}; the \codeindex{INCBIN} command, the \codeindex{EQU}
command, and the \codeindex{TIMES} prefix.

\subsection{DB and Friends: Declaring Initialized Data}
\label{subsec:db}

\codeindex{DB}, \codeindex{DW}, \codeindex{DD}, \codeindex{DQ},
\codeindex{DT}, \codeindex{DO}, \codeindex{DY} and \codeindex{DZ}
are used, much as in MASM, to declare initialized data in
the output file. They can be invoked in a wide range of ways:
\index{floating-point}\index{character constant}\index{string constant}
 
\begin{lstlisting}
	db    0x55                ; just the byte 0x55
	db    0x55,0x56,0x57      ; three bytes in succession
	db    'a',0x55            ; character constants are OK
	db    'hello',13,10,'$'   ; so are string constants
	dw    0x1234              ; 0x34 0x12
	dw    'a'                 ; 0x61 0x00 (it's just a number)
	dw    'ab'                ; 0x61 0x62 (character constant)
	dw    'abc'               ; 0x61 0x62 0x63 0x00 (string)
	dd    0x12345678          ; 0x78 0x56 0x34 0x12
	dd    1.234567e20         ; floating-point constant
	dq    0x123456789abcdef0  ; eight byte constant
	dq    1.234567e20         ; double-precision float
	dt    1.234567e20         ; extended-precision float
\end{lstlisting}

\code{DT}, \code{DO}, \code{DY} and \code{DZ} do not accept
\textindex{numeric constants} as operands.

\subsection{RESB and Friends: Declaring \htextindex{Uninitialized} Data}
\label{subsec:resb}

\codeindex{RESB}, \codeindex{RESW}, \codeindex{RESD}, \codeindex{RESQ},
\codeindex{REST}, \codeindex{RESO}, \codeindex{RESY} and \codeindex{RESZ}
are designed to be used in the BSS section of a module: they declare
\emph{uninitialized} storage space. Each takes a single operand, which is
the number of bytes, words, doublewords or whatever to reserve. As stated
in \fullref{sec:qsother}, NASM does not support the MASM/TASM syntax of
reserving uninitialized space by writing \index{?}\code{DW ?} or similar
things: this is what it does instead. The operand to a \code{RESB}-type
pseudo-instruction is a \textindex{critical expression}:
see \fullref{sec:crit}.

For example:

\begin{lstlisting}
	buffer:         resb    64              ; reserve 64 bytes
	wordvar:        resw    1               ; reserve a word
	realarray       resq    10              ; array of ten reals
	ymmval:         resy    1               ; one YMM register
	zmmvals:        resz    32              ; 32 ZMM registers
\end{lstlisting}

\subsection{\hcodeindex{INCBIN}: Including External \htextindex{Binary Files}}
\label{subsec:incbin}

\code{INCBIN} is borrowed from the old Amiga assembler \textindex{DevPac}:
it includes a binary file verbatim into the output file. This can be handy
for (for example) including \textindex{graphics} and \textindex{sound} data
directly into a game executable file. It can be called in one of these
three ways:

\begin{lstlisting}
	incbin  "file.dat"             ; include the whole file
	incbin  "file.dat",1024        ; skip the first 1024 bytes
	incbin  "file.dat",1024,512    ; skip the first 1024, and
\end{lstlisting}

\code{INCBIN} is both a directive and a standard macro; the standard
macro version searches for the file in the include file search path
and adds the file to the dependency lists. This macro can be
overridden if desired.

\subsection{\hcodeindex{EQU}: Defining Constants}
\label{subsec:equ}

\code{EQU} defines a symbol to a given constant value: when \code{EQU} is
used, the source line must contain a label. The action of \code{EQU} is
to define the given label name to the value of its (only) operand.
This definition is absolute, and cannot change later. So, for
example,

\begin{lstlisting}
	message         db      'hello, world'
	msglen          equ     $-message
\end{lstlisting}

defines \code{msglen} to be the constant 12. \code{msglen} may
not then be redefined later. This is not a \textindex{preprocessor}
definition either: the value of \code{msglen} is evaluated \code{once},
using the value of \code{\$} (see \fullref{sec:expr} for an explanation
of \code{\$}) at the point of definition, rather than being evaluated
wherever it is referenced and using the value of \code{\$} at
the point of reference.

\subsection{\hcodeindex{TIMES}: \htextindex{Repeating} Instructions or Data}
\label{subsec:times}

The \code{TIMES} prefix causes the instruction to be assembled multiple
times. This is partly present as NASM's equivalent of the \codeindex{DUP}
syntax supported by \textindex{MASM}-compatible assemblers, in that you can
code

\begin{lstlisting}
	zerobuf:        times 64 db 0
\end{lstlisting}

or similar things; but \code{TIMES} is more versatile than that. The
argument to \code{TIMES} is not just a numeric constant, but a numeric
\emph{expression}, so you can do things like

\begin{lstlisting}
	buffer: db      'hello, world'
	        times 64-$+buffer db ' '
\end{lstlisting}

which will store exactly enough spaces to make the total length of
\code{buffer} up to 64. Finally, \code{TIMES} can be applied to ordinary
instructions, so you can code trivial \textindex{unrolled loops} in it:

\begin{lstlisting}
	times 100 movsb
\end{lstlisting}

Note that there is no effective difference between \code{times 100 resb
1} and \code{resb 100}, except that the latter will be assembled about
100 times faster due to the internal structure of the assembler.

The operand to \code{TIMES} is a critical expression (\fullref{sec:crit}).

Note also that \code{TIMES} can't be applied to \textindex{macros}: the reason
for this is that \code{TIMES} is processed after the macro phase, which
allows the argument to \code{TIMES} to contain expressions such as
\code{64-\$+buffer} as above. To repeat more than one line of code,
or a complex macro, use the preprocessor \codeindex{\%rep} directive.

\section{Effective Addresses}
\label{sec:effaddr}

An \textindex{effective address} is any operand to an instruction which
\index{memory reference}references memory. Effective addresses, in NASM,
have a very simple syntax: they consist of an expression evaluating
to the desired address, enclosed in \textindex{square brackets}. For
example:

\begin{lstlisting}
	wordvar dw      123
	        mov     ax,[wordvar]
	        mov     ax,[wordvar+1]
	        mov     ax,[es:wordvar+bx]
\end{lstlisting}

Anything not conforming to this simple system is not a valid memory
reference in NASM, for example \code{es:wordvar[bx]}.

More complicated effective addresses, such as those involving more
than one register, work in exactly the same way:

\begin{lstlisting}
	mov     eax,[ebx*2+ecx+offset]
	mov     ax,[bp+di+8]
\end{lstlisting}

NASM is capable of doing \textindex{algebra} on these effective addresses,
so that things which don't necessarily \emph{look} legal are perfectly
all right:

\begin{lstlisting}
	mov     eax,[ebx*5]             ; assembles as [ebx*4+ebx]
	mov     eax,[label1*2-label2]   ; ie [label1+(label1-label2)]
\end{lstlisting}

Some forms of effective address have more than one assembled form;
in most such cases NASM will generate the smallest form it can. For
example, there are distinct assembled forms for the 32-bit effective
addresses \code{[eax*2+0]} and \code{[eax+eax]}, and NASM will
generally generate the latter on the grounds that the former requires
four bytes to store a zero offset.

NASM has a hinting mechanism which will cause \code{[eax+ebx]} and
\code{[ebx+eax]} to generate different opcodes; this is occasionally
useful because \code{[esi+ebp]} and \code{[ebp+esi]} have different
default segment registers.

However, you can force NASM to generate an effective address in a
particular form by the use of the keywords \code{BYTE}, \code{WORD},
\code{DWORD} and \code{NOSPLIT}. If you need \code{[eax+3]} to be
assembled using a double-word offset field instead of the one byte NASM
will normally generate, you can code \code{[dword eax+3]}. Similarly, you
can force NASM to use a byte offset for a small value which it hasn't seen
on the first pass (see \fullref{sec:crit} for an example of such a code
fragment) by using \code{[byte eax+offset]}. As special cases, \code{[byte eax]}
will code \code{[eax+0]} with a byte offset of zero, and \code{[dword eax]}
will code it with a double-word offset of zero. The normal form, \code{[eax]},
will be coded with no offset field.

The form described in the previous paragraph is also useful if you
are trying to access data in a 32-bit segment from within 16 bit code.
For more information on this see the section on mixed-size addressing
(\fullref{sec:mixaddr}). In particular, if you need to access data with
a known offset that is larger than will fit in a 16-bit value, if you don't
specify that it is a dword offset, nasm will cause the high word of
the offset to be lost.

Similarly, NASM will split \code{[eax*2]} into \code{[eax+eax]} because
that allows the offset field to be absent and space to be saved; in fact,
it will also split \code{[eax*2+offset]} into \code{[eax+eax+offset]}.
You can combat this behaviour by the use of the \code{NOSPLIT} keyword:
\code{[nosplit eax*2]} will force \code{[eax*2+0]} to be generated literally.
\code{[nosplit eax*1]} also has the same effect. In another way, a split EA
form \code{[0, eax*2]} can be used, too.  However, \code{NOSPLIT} in
\code{[nosplit eax+eax]} will be ignored because user's intention here
is considered as \code{[eax+eax]}.

In 64-bit mode, NASM will by default generate absolute addresses. The
\codeindex{REL} keyword makes it produce \code{RIP}-relative addresses.
Since this is frequently the normally desired behaviour, see the \code{DEFAULT}
directive (\fullref{sec:default}). The keyword \codeindex{ABS} overrides
\codeindex{REL}.

A new form of split effective addres syntax is also supported. This is
mainly intended for mib operands as used by MPX instructions, but can
be used for any memory reference. The basic concept of this form is
splitting base and index.

\begin{lstlisting}
	mov eax,[ebx+8,ecx*4]   ; ebx=base, ecx=index, 4=scale, 8=disp
\end{lstlisting}

For mib operands, there are several ways of writing effective address
depending on the tools. NASM supports all currently possible ways of
mib syntax:

\begin{lstlisting}
	; bndstx
	; next 5 lines are parsed same
	; base=rax, index=rbx, scale=1, displacement=3
	bndstx [rax+0x3,rbx], bnd0      ; NASM - split EA
	bndstx [rbx*1+rax+0x3], bnd0    ; GAS - '*1' indecates an index reg
	bndstx [rax+rbx+3], bnd0        ; GAS - without hints
	bndstx [rax+0x3], bnd0, rbx     ; ICC-1
	bndstx [rax+0x3], rbx, bnd0     ; ICC-2
\end{lstlisting}

When broadcasting decorator is used, the opsize keyword should match
the size of each element.

\begin{lstlisting}
	VDIVPS zmm4, zmm5, dword [rbx]{1to16}   ; single-precision float
	VDIVPS zmm4, zmm5, zword [rbx]          ; packed 512 bit memory
\end{lstlisting}

\section{\htextindex{Constants}}
\label{sec:const}

NASM understands four different types of constant: numeric,
character, string and floating-point.

\subsection{\htextindex{Numeric Constants}}
\label{subsec:numconst}

A numeric constant is simply a number. NASM allows you to specify
numbers in a variety of number bases, in a variety of ways: you can
suffix \code{H} or \code{X}, \code{D} or \code{T}, \code{Q} or
\code{O}, and \code{B} or \code{Y} for \textindex{hexadecimal},
\textindex{decimal}, \textindex{octal} and \textindex{binary} respectively,
or you can prefix \code{0x}, for hexadecimal in the style of C, or you
can prefix \code{\$} for hexadecimal in the style of Borland Pascal or
Motorola Assemblers. Note, though, that the \index{\$} \index{prefix}\code{\$}
prefix does double duty as a prefix on identifiers (see \fullref{sec:syntax}),
so a hex number prefixed with a \code{\$} sign must have a digit after the
\code{\$} rather than a letter. In addition, current versions of NASM accept
the prefix \code{0h} for hexadecimal, \code{0d} or \code{0t} for decimal,
\code{0o} or \code{0q} for octal, and \code{0b} or \code{0y} for binary.
Please note that unlike C, a \code{0} prefix by itself does \emph{not} imply
an octal constant!

Numeric constants can have underscores (\code{\_}) interspersed to break
up long strings.

Some examples (all producing exactly the same code):

\begin{lstlisting}
	mov     ax,200          ; decimal
	mov     ax,0200         ; still decimal
	mov     ax,0200d        ; explicitly decimal
	mov     ax,0d200        ; also decimal
	mov     ax,0c8h         ; hex
	mov     ax,$0c8         ; hex again: the 0 is required
	mov     ax,0xc8         ; hex yet again
	mov     ax,0hc8         ; still hex
	mov     ax,310q         ; octal
	mov     ax,310o         ; octal again
	mov     ax,0o310        ; octal yet again
	mov     ax,0q310        ; octal yet again
	mov     ax,11001000b    ; binary
	mov     ax,1100_1000b   ; same binary constant
	mov     ax,1100_1000y   ; same binary constant once more
	mov     ax,0b1100_1000  ; same binary constant yet again
	mov     ax,0y1100_1000  ; same binary constant yet again
\end{lstlisting}

\label{subsec:strings}
\subsection{\index{Strings}\htextindex{Character Strings}}

A character string consists of up to eight characters enclosed in
either single quotes (\code{'...'}), double quotes (\code{"..."}) or
backquotes (\code{`...`}).  Single or double quotes are equivalent to
NASM (except of course that surrounding the constant with single
quotes allows double quotes to appear within it and vice versa); the
contents of those are represented verbatim.  Strings enclosed in
backquotes support C-style \code{\\}-escapes for special characters.

The following \textindex{escape sequences} are recognized by backquoted strings:

\begin{lstlisting}
	\'          single quote (')
	\"          double quote (")
	\`          backquote (`)
	\\          backslash (\)
	\?          question mark (?)
	\a          BEL (ASCII 7)
	\b          BS  (ASCII 8)
	\t          TAB (ASCII 9)
	\n          LF  (ASCII 10)
	\v          VT  (ASCII 11)
	\f          FF  (ASCII 12)
	\r          CR  (ASCII 13)
	\e          ESC (ASCII 27)
	\377        Up to 3 octal digits - literal byte
	\xFF        Up to 2 hexadecimal digits - literal byte
	\u1234      4 hexadecimal digits - Unicode character
	\U12345678  8 hexadecimal digits - Unicode character
\end{lstlisting}

All other escape sequences are reserved. Note that \code{\\0},
meaning a \code{NUL} character (ASCII 0), is a special case of
the octal escape sequence.

\textindex{Unicode} characters specified with \code{\\u} or \code{\\U}
are converted to \textindex{UTF-8}. For example, the following lines
are all equivalent:

\begin{lstlisting}
	db `\u263a`            ; UTF-8 smiley face
	db `\xe2\x98\xba`      ; UTF-8 smiley face
	db 0E2h, 098h, 0BAh    ; UTF-8 smiley face
\end{lstlisting}
