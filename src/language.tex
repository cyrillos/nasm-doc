%
% vim: ts=4 sw=4 et
%
\xchapter{lang}{The NASM Language}

\xsection{syntax}{Layout of a NASM Source Line}

Like most assemblers, each NASM source line contains (unless it
is a macro, a preprocessor directive or an assembler directive: see
\nref{preproc} and \nref{directive}) some combination
of the four fields

\begin{lstlisting}
label:    instruction operands        ; comment
\end{lstlisting}

As usual, most of these fields are optional; the presence or absence
of any combination of a label, an instruction and a comment is allowed.
Of course, the operand field is either required or forbidden by the
presence and nature of the instruction field.

NASM uses backslash (\code{\textbackslash}) as the line continuation character;
if a line ends with backslash, the next line is considered to be
a part of the backslash-ended line.

NASM places no restrictions on white space within a line: labels may
have white space before them, or instructions may have no space
before them, or anything. The \textindex{colon} after a label is also
optional. (Note that this means that if you intend to code \code{lodsb}
alone on a line, and type \code{lodab} by accident, then that's still a
valid source line which does nothing but define a label. Running
NASM with the command-line option \index{orphan-labels}\code{-w+orphan-labels}
will cause it to warn you if you define a label alone on a line without
a \textindex{trailing colon}.)

\textindex{Valid characters} in labels are letters, numbers, \code{\_},
\code{\$}, \code{\#}, \code{\@}, \code{~}, \code{.}, and \code{?}.
The only characters which may be used as the \emph{first} character of
an identifier are letters, \code{\.} (with special meaning: see
\nref{locallab}), \code{\_} and \code{?}.
An identifier may also be prefixed with a \index{\$}\index{prefix}
\code{\$} to indicate that it is intended to be read as an identifier
and not a reserved word; thus, if some other module you are linking with
defines a symbol called \code{eax}, you can refer to \code{\$eax} in NASM
code to distinguish the symbol from the register. Maximum length of
an identifier is 4095 characters.

The instruction field may contain any machine instruction: Pentium
and P6 instructions, FPU instructions, MMX instructions and even
undocumented instructions are all supported. The instruction may be
prefixed by \code{LOCK}, \code{REP}, \code{REPE}/\code{REPZ},
\code{REPNE}/\code{REPNZ}, \code{XACQUIRE}/\code{XRELEASE} or
\code{BND}/\code{NOBND}, in the usual way. Explicit
\index{address-size!prefixes}address-size and \textindex{operand-size!prefixes}
\codeindex{A16}, \codeindex{A32}, \codeindex{A64}, \codeindex{O16}
and \codeindex{O32}, \codeindex{O64} are provided~-- one example of their
use is given in \nref{mixsize}. You can also use the name of a
\index{segment override}segment register as an instruction prefix: coding
\code{es mov [bx],ax} is equivalent to coding \code{mov [es:bx],ax}.
We recommend the latter syntax, since it is consistent with other syntactic
features of the language, but for instructions such as \code{LODSB}, which
has no operands and yet can require a segment override, there is no clean
syntactic way to proceed apart from \code{es lodsb}.

An instruction is not required to use a prefix: prefixes such as
\code{CS}, \code{A32}, \code{LOCK} or \code{REPE} can appear on
a line by themselves, and NASM will just generate the prefix bytes.

In addition to actual machine instructions, NASM also supports a
number of pseudo-instructions, described in \k{pseudop}.

Instruction \textindex{operands} may take a number of forms: they can be
registers, described simply by the register name (e.g. \code{ax},
\code{bp}, \code{ebx}, \code{cr0}: NASM does not use the \code{gas}-style
syntax in which register names must be prefixed by a \code{\%} sign),
or they can be \textindex{effective addresses} (see \nref{effaddr}),
constants (\nref{const}) or expressions (\nref{expr}).

For x87 \textindex{floating-point} instructions, NASM accepts a wide
range of syntaxes: you can use two-operand forms like MASM supports,
or you can use NASM's native single-operand forms in most cases.
% Details of all forms of each supported instruction are given in
% \nref{iref}.
For example, you can code:

\begin{lstlisting}
fadd    st1             ; this sets st0 := st0 + st1
fadd    st0,st1         ; so does this

fadd    st1,st0         ; this sets st1 := st1 + st0
fadd    to st1          ; so does this
\end{lstlisting}

Almost any x87 floating-point instruction that references memory must
use one of the prefixes \codeindex{DWORD}, \codeindex{QWORD} or
\codeindex{TWORD} to indicate what size of \textindex{memory operand}
it refers to.

\xsection{pseudop}{\textindexlc{Pseudo-Instructions}}

Pseudo-instructions are things which, though not real x86 machine
instructions, are used in the instruction field anyway because that's
the most convenient place to put them. The current pseudo-instructions
are \codeindex{DB}, \codeindex{DW}, \codeindex{DD}, \codeindex{DQ},
\codeindex{DT}, \codeindex{DO}, \codeindex{DY} and \codeindex{DZ};
their \textindex{uninitialized} counterparts \codeindex{RESB},
\codeindex{RESW}, \codeindex{RESD}, \codeindex{RESQ},
\codeindex{REST}, \codeindex{RESO}, \codeindex{RESY} and
\codeindex{RESZ}; the \codeindex{INCBIN} command, the \codeindex{EQU}
command, and the \codeindex{TIMES} prefix.

\xsubsection{db}{DB and Friends: Declaring Initialized Data}

\codeindex{DB}, \codeindex{DW}, \codeindex{DD}, \codeindex{DQ},
\codeindex{DT}, \codeindex{DO}, \codeindex{DY} and \codeindex{DZ}
are used, much as in MASM, to declare initialized data in
the output file. They can be invoked in a wide range of ways:
\index{floating-point!constants}\index{character!constants}\index{string!constants}

\begin{lstlisting}
db    0x55                ; just the byte 0x55
db    0x55,0x56,0x57      ; three bytes in succession
db    'a',0x55            ; character constants are OK
db    'hello',13,10,'$'   ; so are string constants
dw    0x1234              ; 0x34 0x12
dw    'a'                 ; 0x61 0x00 (it's just a number)
dw    'ab'                ; 0x61 0x62 (character constant)
dw    'abc'               ; 0x61 0x62 0x63 0x00 (string)
dd    0x12345678          ; 0x78 0x56 0x34 0x12
dd    1.234567e20         ; floating-point constant
dq    0x123456789abcdef0  ; eight byte constant
dq    1.234567e20         ; double-precision float
dt    1.234567e20         ; extended-precision float
\end{lstlisting}

\code{DT}, \code{DO}, \code{DY} and \code{DZ} do not accept
\textindex{numeric constants}\index{numeric!constants} as operands.

\xsubsection{resb}{RESB and Friends: Declaring \textindexlc{Uninitialized} Data}

\codeindex{RESB}, \codeindex{RESW}, \codeindex{RESD}, \codeindex{RESQ},
\codeindex{REST}, \codeindex{RESO}, \codeindex{RESY} and \codeindex{RESZ}
are designed to be used in the BSS section of a module: they declare
\emph{uninitialized} storage space. Each takes a single operand, which is
the number of bytes, words, doublewords or whatever to reserve. As stated
in \nref{qsother}, NASM does not support the MASM/TASM syntax of
reserving uninitialized space by writing \index{?}\code{DW ?} or similar
things: this is what it does instead. The operand to a \code{RESB}-type
pseudo-instruction is a \textindex{critical expression}:
see \nref{crit}.

For example:

\begin{lstlisting}
buffer:         resb    64              ; reserve 64 bytes
wordvar:        resw    1               ; reserve a word
realarray       resq    10              ; array of ten reals
ymmval:         resy    1               ; one YMM register
zmmvals:        resz    32              ; 32 ZMM registers
\end{lstlisting}

\xsubsection{incbin}{\codeindex{INCBIN}: Including External \textindexlc{Binary Files}}

\code{INCBIN} is borrowed from the old Amiga assembler \textindex{DevPac}:
it includes a binary file verbatim into the output file. This can be handy
for (for example) including \textindex{graphics} and \textindex{sound} data
directly into a game executable file. It can be called in one of these
three ways:

\begin{lstlisting}
incbin  "file.dat"             ; include the whole file
incbin  "file.dat",1024        ; skip the first 1024 bytes
incbin  "file.dat",1024,512    ; skip the first 1024, and
\end{lstlisting}

\code{INCBIN} is both a directive and a standard macro; the standard
macro version searches for the file in the include file search path
and adds the file to the dependency lists. This macro can be
overridden if desired.

\xsubsection{equ}{\codeindex{EQU}: Defining Constants}

\code{EQU} defines a symbol to a given constant value: when \code{EQU} is
used, the source line must contain a label. The action of \code{EQU} is
to define the given label name to the value of its (only) operand.
This definition is absolute, and cannot change later. So, for
example,

\begin{lstlisting}
message         db      'hello, world'
msglen          equ     $-message
\end{lstlisting}

defines \code{msglen} to be the constant 12. \code{msglen} may
not then be redefined later. This is not a \textindex{preprocessor}
definition either: the value of \code{msglen} is evaluated \code{once},
using the value of \code{\$} (see \nref{expr} for an explanation
of \code{\$}) at the point of definition, rather than being evaluated
wherever it is referenced and using the value of \code{\$} at
the point of reference.

\xsubsection{times}{\codeindex{TIMES}: \textindexlc{Repeating} Instructions or Data}

The \code{TIMES} prefix causes the instruction to be assembled multiple
times. This is partly present as NASM's equivalent of the \codeindex{DUP}
syntax supported by \textindex{MASM}-compatible assemblers, in that you can
code

\begin{lstlisting}
zerobuf:        times 64 db 0
\end{lstlisting}

or similar things; but \code{TIMES} is more versatile than that. The
argument to \code{TIMES} is not just a numeric constant, but a numeric
\emph{expression}, so you can do things like

\begin{lstlisting}
buffer: db      'hello, world'
        times 64-$+buffer db ' '
\end{lstlisting}

which will store exactly enough spaces to make the total length of
\code{buffer} up to 64. Finally, \code{TIMES} can be applied to ordinary
instructions, so you can code trivial \textindex{unrolled loops} in it:

\begin{lstlisting}
times 100 movsb
\end{lstlisting}

Note that there is no effective difference between \code{times 100 resb
1} and \code{resb 100}, except that the latter will be assembled about
100 times faster due to the internal structure of the assembler.

The operand to \code{TIMES} is a critical expression (\nref{crit}).

Note also that \code{TIMES} can't be applied to \textindex{macros}: the reason
for this is that \code{TIMES} is processed after the macro phase, which
allows the argument to \code{TIMES} to contain expressions such as
\code{64-\$+buffer} as above. To repeat more than one line of code,
or a complex macro, use the preprocessor \codeindex{\%rep} directive.

\xsection{effaddr}{Effective Addresses}

An \textindex{effective address} is any operand to an instruction which
\index{memory reference}references memory. Effective addresses, in NASM,
have a very simple syntax: they consist of an expression evaluating
to the desired address, enclosed in \textindex{square brackets}. For
example:

\begin{lstlisting}
wordvar dw      123
        mov     ax,[wordvar]
        mov     ax,[wordvar+1]
        mov     ax,[es:wordvar+bx]
\end{lstlisting}

Anything not conforming to this simple system is not a valid memory
reference in NASM, for example \code{es:wordvar[bx]}.

More complicated effective addresses, such as those involving more
than one register, work in exactly the same way:

\begin{lstlisting}
mov     eax,[ebx*2+ecx+offset]
mov     ax,[bp+di+8]
\end{lstlisting}

NASM is capable of doing \textindex{algebra} on these effective addresses,
so that things which don't necessarily \emph{look} legal are perfectly
all right:

\begin{lstlisting}
mov     eax,[ebx*5]             ; assembles as [ebx*4+ebx]
mov     eax,[label1*2-label2]   ; ie [label1+(label1-label2)]
\end{lstlisting}

Some forms of effective address have more than one assembled form;
in most such cases NASM will generate the smallest form it can. For
example, there are distinct assembled forms for the 32-bit effective
addresses \code{[eax*2+0]} and \code{[eax+eax]}, and NASM will
generally generate the latter on the grounds that the former requires
four bytes to store a zero offset.

NASM has a hinting mechanism which will cause \code{[eax+ebx]} and
\code{[ebx+eax]} to generate different opcodes; this is occasionally
useful because \code{[esi+ebp]} and \code{[ebp+esi]} have different
default segment registers.

However, you can force NASM to generate an effective address in a
particular form by the use of the keywords \code{BYTE}, \code{WORD},
\code{DWORD} and \code{NOSPLIT}. If you need \code{[eax+3]} to be
assembled using a double-word offset field instead of the one byte NASM
will normally generate, you can code \code{[dword eax+3]}. Similarly, you
can force NASM to use a byte offset for a small value which it hasn't seen
on the first pass (see \nref{crit} for an example of such a code
fragment) by using \code{[byte eax+offset]}. As special cases, \code{[byte eax]}
will code \code{[eax+0]} with a byte offset of zero, and \code{[dword eax]}
will code it with a double-word offset of zero. The normal form, \code{[eax]},
will be coded with no offset field.

The form described in the previous paragraph is also useful if you
are trying to access data in a 32-bit segment from within 16 bit code.
For more information on this see the section on mixed-size addressing
(\nref{mixaddr}). In particular, if you need to access data with
a known offset that is larger than will fit in a 16-bit value, if you don't
specify that it is a dword offset, nasm will cause the high word of
the offset to be lost.

Similarly, NASM will split \code{[eax*2]} into \code{[eax+eax]} because
that allows the offset field to be absent and space to be saved; in fact,
it will also split \code{[eax*2+offset]} into \code{[eax+eax+offset]}.
You can combat this behaviour by the use of the \code{NOSPLIT} keyword:
\code{[nosplit eax*2]} will force \code{[eax*2+0]} to be generated literally.
\code{[nosplit eax*1]} also has the same effect. In another way, a split EA
form \code{[0, eax*2]} can be used, too. However, \code{NOSPLIT} in
\code{[nosplit eax+eax]} will be ignored because user's intention here
is considered as \code{[eax+eax]}.

In 64-bit mode, NASM will by default generate absolute addresses. The
\codeindex{REL} keyword makes it produce \code{RIP}-relative addresses.
Since this is frequently the normally desired behaviour, see the \code{DEFAULT}
directive (\nref{default}). The keyword \codeindex{ABS} overrides
\codeindex{REL}.

A new form of split effective addres syntax is also supported. This is
mainly intended for mib operands as used by MPX instructions, but can
be used for any memory reference. The basic concept of this form is
splitting base and index.

\begin{lstlisting}
mov eax,[ebx+8,ecx*4]   ; ebx=base, ecx=index, 4=scale, 8=disp
\end{lstlisting}

For mib operands, there are several ways of writing effective address
depending on the tools. NASM supports all currently possible ways of
mib syntax:

\begin{lstlisting}
; bndstx
; next 5 lines are parsed same
; base=rax, index=rbx, scale=1, displacement=3
bndstx [rax+0x3,rbx], bnd0      ; NASM - split EA
bndstx [rbx*1+rax+0x3], bnd0    ; GAS - '*1' indecates an index reg
bndstx [rax+rbx+3], bnd0        ; GAS - without hints
bndstx [rax+0x3], bnd0, rbx     ; ICC-1
bndstx [rax+0x3], rbx, bnd0     ; ICC-2
\end{lstlisting}

When broadcasting decorator is used, the opsize keyword should match
the size of each element.

\begin{lstlisting}
vdivps zmm4, zmm5, dword [rbx]{1to16}   ; single-precision float
vdivps zmm4, zmm5, zword [rbx]          ; packed 512 bit memory
\end{lstlisting}

\xsection{const}{\textindexlc{Constants}}

NASM understands four different types of constant: numeric,
character, string and floating-point.

\xsubsection{numconst}{\textindexlc{Numeric Constants}}

A numeric constant is simply a number. NASM allows you to specify
numbers in a variety of number bases, in a variety of ways: you can
suffix \code{H} or \code{X}, \code{D} or \code{T}, \code{Q} or
\code{O}, and \code{B} or \code{Y} for \textindex{hexadecimal},
\textindex{decimal}, \textindex{octal} and \textindex{binary} respectively,
or you can prefix \code{0x}, for hexadecimal in the style of C, or you
can prefix \code{\$} for hexadecimal in the style of Borland Pascal or
Motorola Assemblers. Note, though, that the \index{\$} \index{prefix}\code{\$}
prefix does double duty as a prefix on identifiers (see \nref{syntax}),
so a hex number prefixed with a \code{\$} sign must have a digit after the
\code{\$} rather than a letter. In addition, current versions of NASM accept
the prefix \code{0h} for hexadecimal, \code{0d} or \code{0t} for decimal,
\code{0o} or \code{0q} for octal, and \code{0b} or \code{0y} for binary.
Please note that unlike C, a \code{0} prefix by itself does \emph{not} imply
an octal constant!

Numeric constants can have underscores (\code{\_}) interspersed to break
up long strings.

Some examples (all producing exactly the same code):

\begin{lstlisting}
mov     ax,200          ; decimal
mov     ax,0200         ; still decimal
mov     ax,0200d        ; explicitly decimal
mov     ax,0d200        ; also decimal
mov     ax,0c8h         ; hex
mov     ax,$0c8         ; hex again: the 0 is required
mov     ax,0xc8         ; hex yet again
mov     ax,0hc8         ; still hex
mov     ax,310q         ; octal
mov     ax,310o         ; octal again
mov     ax,0o310        ; octal yet again
mov     ax,0q310        ; octal yet again
mov     ax,11001000b    ; binary
mov     ax,1100_1000b   ; same binary constant
mov     ax,1100_1000y   ; same binary constant once more
mov     ax,0b1100_1000  ; same binary constant yet again
mov     ax,0y1100_1000  ; same binary constant yet again
\end{lstlisting}

\xsubsection{strings}{\index{Strings}\textindexlc{Character Strings}}

A character string consists of up to eight characters enclosed in
either single quotes (\code{'...'}), double quotes (\code{"..."}) or
backquotes (\code{`...`}). Single or double quotes are equivalent to
NASM (except of course that surrounding the constant with single
quotes allows double quotes to appear within it and vice versa); the
contents of those are represented verbatim. Strings enclosed in
backquotes support C-style \code{\textbackslash}-escapes for
special characters.

The following \textindex{escape sequences} are recognized by backquoted strings:

\begin{lstlisting}
\'          single quote (')
\"          double quote (")
\`          backquote (`)
\\          backslash (\)
\?          question mark (?)
\a          BEL (ASCII 7)
\b          BS  (ASCII 8)
\t          TAB (ASCII 9)
\n          LF  (ASCII 10)
\v          VT  (ASCII 11)
\f          FF  (ASCII 12)
\r          CR  (ASCII 13)
\e          ESC (ASCII 27)
\377        Up to 3 octal digits - literal byte
\xFF        Up to 2 hexadecimal digits - literal byte
\u1234      4 hexadecimal digits - Unicode character
\U12345678  8 hexadecimal digits - Unicode character
\end{lstlisting}

All other escape sequences are reserved. Note that \code{\textbackslash 0},
meaning a \code{NUL} character (ASCII 0), is a special case of
the octal escape sequence.

\textindex{Unicode} characters specified with \code{\textbackslash u}
or \code{\textbackslash U} are converted to \textindex{UTF-8}.
For example, the following lines are all equivalent:

\begin{lstlisting}
db `\u263a`            ; UTF-8 smiley face
db `\xe2\x98\xba`      ; UTF-8 smiley face
db 0E2h, 098h, 0BAh    ; UTF-8 smiley face
\end{lstlisting}

\xsubsection{chrconst}{\textindexlc{Character Constants}}

A character constant consists of a string up to eight bytes long, used
in an expression context. It is treated as if it was an integer.

A character constant with more than one byte will be arranged
with \i{little-endian} order in mind: if you code

\begin{lstlisting}
mov eax,'abcd'
\end{lstlisting}

then the constant generated is not \code{0x61626364}, but \code{0x64636261},
so that if you were then to store the value into memory, it would read
\code{abcd} rather than \code{dcba}. This is also the sense of character
constants understood by the Pentium's \codeindex{CPUID} instruction.

\xsubsection{strconst}{\textindexlc{String Constants}}

String constants are character strings used in the context of some
pseudo-instructions, namely the \indexcode{DW}\indexcode{DD}\indexcode{DQ}
\indexcode{DT}\indexcode{DO}\indexcode{DY}\codeindex{DB} family and
\codeindex{INCBIN} (where it represents a filename.) They are also used in
certain preprocessor directives.

A string constant looks like a character constant, only longer. It
is treated as a concatenation of maximum-size character constants
for the conditions. So the following are equivalent:

\begin{lstlisting}
db    'hello'               ; string constant
db    'h','e','l','l','o'   ; equivalent character constants
\end{lstlisting}

And the following are also equivalent:

\begin{lstlisting}
dd    'ninechars'           ; doubleword string constant
dd    'nine','char','s'     ; becomes three doublewords
db    'ninechars',0,0,0     ; and really looks like this
\end{lstlisting}

Note that when used in a string-supporting context, quoted strings are
treated as a string constants even if they are short enough to be a
character constant, because otherwise \code{db 'ab'} would have the same
effect as \code{db 'a'}, which would be silly. Similarly, three-character
or four-character constants are treated as strings when they are
operands to \code{DW}, and so forth.

\xsubsection{unicode}{\index{UTF-16}\index{UTF-32}\textindexlc{Unicode} Constants}

The special operators \codeindex{\_\_utf16\_\_}, \codeindex{\_\_utf16le\_\_},
\codeindex{\_\_utf16be\_\_}, \codeindex{\_\_utf32\_\_}, \codeindex{\_\_utf32le\_\_}
and \codeindex{\_\_utf32be\_\_} allows definition of Unicode strings.
They take a string in UTF-8 format and converts it to UTF-16 or UTF-32,
respectively. Unless the \code{be} forms are specified, the output is
littleendian.

For example:

\begin{lstlisting}
%define u(x) __utf16__(x)
%define w(x) __utf32__(x)

      dw u('C:\WINDOWS'), 0       ; Pathname in UTF-16
      dd w(`A + B = \u206a`), 0   ; String in UTF-32
\end{lstlisting}

The UTF operators can be applied either to strings passed to the
\code{DB} family instructions, or to character constants in an expression
context.

\xsubsection{fltconst}{\index{floating-point!constants}Floating-Point Constants}

\textindexlc{Floating-point} constants are acceptable only as arguments to
\codeindex{DB}, \codeindex{DW}, \codeindex{DD}, \codeindex{DQ}, \codeindex{DT},
and \codeindex{DO}, or as arguments to the special operators \codeindex{\_\_float8\_\_},
\codeindex{\_\_float16\_\_}, \codeindex{\_\_float32\_\_}, \codeindex{\_\_float64\_\_},
\codeindex{\_\_float80m\_\_}, \codeindex{\_\_float80e\_\_}, \codeindex{\_\_float128l\_\_},
and \codeindex{\_\_float128h\_\_}.

Floating-point constants are expressed in the traditional form:
digits, then a period, then optionally more digits, then optionally an
\code{E} followed by an exponent. The period is mandatory, so that NASM
can distinguish between \code{dd 1}, which declares an integer constant,
and \code{dd 1.0} which declares a floating-point constant.

NASM also support C99-style hexadecimal floating-point: \code{0x},
hexadecimal digits, period, optionally more hexadeximal digits, then
optionally a \code{P} followed by a \emph{binary} (not hexadecimal)
exponent in decimal notation. As an extension, NASM additionally
supports the \code{0h} and \code{\$} prefixes for hexadecimal,
as well binary and octal floating-point, using the \code{0b} or
\code{0y} and \code{0o} or \code{0q} prefixes, respectively.

Underscores to break up groups of digits are permitted in
floating-point constants as well.

Some examples:

\begin{lstlisting}
db    -0.2                    ; "Quarter precision"
dw    -0.5                    ; IEEE 754r/SSE5 half precision
dd    1.2                     ; an easy one
dd    1.222_222_222           ; underscores are permitted
dd    0x1p+2                  ; 1.0x2^2 = 4.0
dq    0x1p+32                 ; 1.0x2^32 = 4 294 967 296.0
dq    1.e10                   ; 10 000 000 000.0
dq    1.e+10                  ; synonymous with 1.e10
dq    1.e-10                  ; 0.000 000 000 1
dt    3.141592653589793238462 ; pi
do    1.e+4000                ; IEEE 754r quad precision
\end{lstlisting}

The 8-bit "quarter-precision" floating-point format is
sign:exponent:mantissa = 1:4:3 with an exponent bias of 7. This
appears to be the most frequently used 8-bit floating-point format,
although it is not covered by any formal standard. This is sometimes
called a ``\textindex{minifloat}''.

The special operators are used to produce floating-point numbers in
other contexts. They produce the binary representation of a specific
floating-point number as an integer, and can use anywhere integer
constants are used in an expression. \code{\_\_float80m\_\_} and
\code{\_\_float80e\_\_} produce the 64-bit mantissa and 16-bit
exponent of an 80-bit floating-point number, and \code{\_\_float128l\_\_}
and \code{\_\_float128h\_\_} produce the lower and upper 64-bit halves
of a 128-bit floating-point number, respectively.

For example:

\begin{lstlisting}
mov    rax,__float64__(3.141592653589793238462)
\end{lstlisting}

would assign the binary representation of pi as a 64-bit floating
point number into \code{RAX}. This is exactly equivalent to:

\begin{lstlisting}
mov    rax,0x400921fb54442d18
\end{lstlisting}

NASM cannot do compile-time arithmetic on floating-point constants.
This is because NASM is designed to be portable - although it always
generates code to run on x86 processors, the assembler itself can
run on any system with an ANSI C compiler. Therefore, the assembler
cannot guarantee the presence of a floating-point unit capable of
handling the \textindexlc{Intel number formats}, and so for NASM
to be able to do floating arithmetic it would have to include its
own complete set of floating-point routines, which would significantly
increase the size of the assembler for very little benefit.

The special tokens \codeindex{\_\_Infinity\_\_}, \codeindex{\_\_QNaN\_\_} (or
\codeindex{\_\_NaN\_\_}) and \codeindex{\_\_SNaN\_\_} can be used to generate
\index{infinity}infinities, quiet \textindex{NaN}s, and signalling NaNs,
respectively. These are normally used as macros:

\begin{lstlisting}
%define Inf __Infinity__
%define NaN __QNaN__

      dq    +1.5, -Inf, NaN         ; Double-precision constants
\end{lstlisting}

The \code{\%use fp} standard macro package contains a set of convenience
macros. See \nref{pkgfp}.

\xsubsection{bcdconst}{\index{floating-point!packed BCD constants}Packed BCD Constants}

x87-style packed BCD constants can be used in the same contexts as
80-bit floating-point numbers. They are suffixed with \code{p} or
prefixed with \code{0p}, and can include up to 18 decimal digits.

As with other numeric constants, underscores can be used
to separate digits.

For example:

\begin{lstlisting}
dt 12_345_678_901_245_678p
dt -12_345_678_901_245_678p
dt +0p33
dt 33p
\end{lstlisting}

\xsection{expr}{\textindex{Expressions}}

Expressions in NASM are similar in syntax to those in C. Expressions
are evaluated as 64-bit integers which are then adjusted to the
appropriate size.

NASM supports two special tokens in expressions, allowing
calculations to involve the current assembly position: the
\index{\$}\index{here}\code{\$} and \codeindex{\$\$} tokens.
\code{\$} evaluates to the assembly position at the beginning
of the line containing the expression; so you can code an
\textindex{infinite loop} using \code{JMP \$}. \code{\$\$}
evaluates to the beginning of the current section; so you can
tell how far into the section you are by using \code{(\$-\$\$)}.

The arithmetic \textindex{operators} provided by NASM are listed here,
in increasing order of \textindex{precedence}.

\xsubsection{expor}{\codeindex{|}: \textindexlc{Bitwise OR} Operator}

The \code{|} operator gives a bitwise OR, exactly as performed by the
\code{OR} machine instruction. Bitwise OR is the lowest-priority
arithmetic operator supported by NASM.

\xsubsection{expxor}{\codeindex{\textasciicircum}: \textindexlc{Bitwise XOR} Operator}

The \code{\textasciicircum} operator provides the bitwise XOR operation.

\xsubsection{expand}{\codeindex{\&}: \textindexlc{Bitwise AND} Operator}

The \code{\&} operator provides the bitwise AND operation.

\xsubsection{expshift}{\codeindex{<<} and \codeindex{>>}: \textindexlc{Bit Shift} Operators}

\code{<<} gives a bit-shift to the left, just as it does in C.
So \code{5<<3} evaluates to 5 times 8, or 40. \code{>>} gives
a bit-shift to the right; in NASM, such a shift is \emph{always}
unsigned, so that the bits shifted in from the left-hand end
are filled with zero rather than a sign-extension of the
previous highest bit.

\xsubsection{expplmi}{\index{+ opaddition}\codeindex{+} and
\index{- opsubtraction}\codeindex{-}:
\textindexlc{Addition} and \textindexlc{Subtraction} Operators}

The \code{+} and \code{-} operators do perfectly ordinary addition
and subtraction.

\xsubsection{expmul}{\codeindex{*}, \codeindex{/},
\codeindex{//} and \codeindex{\%\%}:
\textindexlc{Multiplication} and \textindexlc{Division}}

\code{*} is the multiplication operator. \code{/} and \code{//} are both
division operators: \code{/} is \textindex{unsigned division} and
\code{//} is \textindex{signed division}. Similarly, \code{\%} and
\code{\%\%} provide \index{unsigned modulo}\index{modulo operators}unsigned
and \textindex{signed modulo} operators respectively.

NASM, like ANSI C, provides no guarantees about the sensible
operation of the signed modulo operator.

Since the \code{\%} character is used extensively by the macro
\textindex{preprocessor}, you should ensure that both the signed
and unsigned modulo operators are followed by white space wherever
they appear.

\xsubsection{expunary}{\textindex{Unary Operators}}

The highest-priority operators in NASM's expression grammar are those
which only apply to one argument. These are \index{+ opunary}\code{+},
\index{- opunary}\code{-}, \codeindex{\textasciitilde},
\index{! opunary}\code{!}, \codeindex{SEG}, and the
\textindex{integer functions} operators.

\code{-} negates its operand, \code{+} does nothing (it's provided for
symmetry with \code{-}), \code{\textasciitilde} computes the
\textindex{one's complement} of its operand, \code{!} is the
\textindex{logical negation} operator.

\code{SEG} provides the \textindex{segment address}
of its operand (explained in more detail in \nref{segwrt}).

A set of additional operators with leading and trailing double
underscores are used to implement the integer functions of the
\code{ifunc} macro package, see \nref{pkgifunc}.

\xsection{segwrt}{\codeindex{SEG} and \codeindex{WRT}}

When writing large 16-bit programs, which must be split into
multiple \textindex{segments}, it is often necessary to be able
to refer to the \index{segment address}segment part of the address
of a symbol. NASM supports the \code{SEG} operator to perform
this function.

The \code{SEG} operator returns the \emph{\textindex{preferred}}
segment base of a symbol, defined as the segment base relative
to which the offset of the symbol makes sense. So the code

\begin{lstlisting}
mov     ax,seg symbol
mov     es,ax
mov     bx,symbol
\end{lstlisting}

will load \code{ES:BX} with a valid pointer to the symbol
\code{symbol}.

Things can be more complex than this: since 16-bit segments and
\textindex{groups} may \index{overlapping segments}overlap,
you might occasionally want to refer to some symbol using
a different segment base from the preferred one. NASM lets you
do this, by the use of the \code{WRT} (With Reference To) keyword.
So you can do things like

\begin{lstlisting}
mov     ax,weird_seg        ; weird_seg is a segment base
mov     es,ax
mov     bx,symbol wrt weird_seg
\end{lstlisting}

to load \code{ES:BX} with a different, but functionally equivalent,
pointer to the symbol \code{symbol}.

NASM supports far (inter-segment) calls and jumps by means of the
syntax \code{call segment:offset}, where \code{segment}
and \code{offset} both represent immediate values. So to call
a far procedure, you could code either of

\begin{lstlisting}
call    (seg procedure):procedure
call    weird_seg:(procedure wrt weird_seg)
\end{lstlisting}

(The parentheses are included for clarity, to show the intended
parsing of the above instructions. They are not necessary in
practice.)

NASM supports the syntax \indexcode{CALL FAR}\code{call far procedure}
as a synonym for the first of the above usages. \code{JMP} works
identically to \code{CALL} in these examples.

To declare a \textindex{far pointer} to a data item in a data
segment, you must code

\begin{lstlisting}
dw      symbol, seg symbol
\end{lstlisting}

NASM supports no convenient synonym for this, though you can always
invent one using the macro processor.

\xsection{strict}{\codeindex{STRICT}: Inhibiting Optimization}

When assembling with the optimizer set to level 2 or higher (see
\nref{opt-O}), NASM will use size specifiers (\code{BYTE},
\code{WORD}, \code{DWORD}, \code{QWORD}, \code{TWORD}, \code{OWORD},
\code{YWORD} or \code{ZWORD}), but will give them the smallest possible
size. The keyword \code{STRICT} can be used to inhibit optimization
and force a particular operand to be emitted in the specified size.
For example, with the optimizer on, and in \code{BITS 16} mode,

\begin{lstlisting}
push dword 33
\end{lstlisting}

is encoded in three bytes \code{66 6A 21}, whereas

\begin{lstlisting}
push strict dword 33
\end{lstlisting}

is encoded in six bytes, with a full dword immediate operand
\code{66 68 21 00 00 00}.

With the optimizer off, the same code (six bytes) is generated whether
the \code{STRICT} keyword was used or not.

\xsection{crit}{\textindexlc{Critical Expressions}}

Although NASM has an optional multi-pass optimizer, there are some
expressions which must be resolvable on the first pass. These are
called \emph{Critical Expressions}.

The first pass is used to determine the size of all the assembled
code and data, so that the second pass, when generating all the
code, knows all the symbol addresses the code refers to. So one
thing NASM can't handle is code whose size depends on the value
of a symbol declared after the code in question. For example,

\begin{lstlisting}
times (label-$) db 0
label:	db      'Where am I?'
\end{lstlisting}

The argument to \codeindex{TIMES} in this case could equally legally
evaluate to anything at all; NASM will reject this example because
it cannot tell the size of the \code{TIMES} line when it first sees it.
It will just as firmly reject the slightly \index{paradox}paradoxical
code

\begin{lstlisting}
times (label-$+1) db 0
label:  db      'NOW where am I?'
\end{lstlisting}

in which \emph{any} value for the \code{TIMES} argument
is by definition wrong!

NASM rejects these examples by means of a concept called a
\emph{critical expression}, which is defined to be an
expression whose value is required to be computable in
the first pass, and which must therefore depend only
on symbols defined before it. The argument to the \code{TIMES}
prefix is a critical expression.

\xsection{locallab}{\textindexlc{Local Labels}}

NASM gives special treatment to symbols beginning with a \textindex{period}.
A label beginning with a single period is treated as a \emph{local}
label, which means that it is associated with the previous non-local
label. So, for example:

\begin{lstlisting}
label1  ; some code

.loop
    ; some more code

    jne     .loop
    ret

label2  ; some code

.loop
    ; some more code

    jne     .loop
    ret
\end{lstlisting}

In the above code fragment, each \code{JNE} instruction jumps to the
line immediately before it, because the two definitions of
\code{.loop} are kept separate by virtue of each being associated
with the previous non-local label.

This form of local label handling is borrowed from the old Amiga
assembler \textindex{DevPac}; however, NASM goes one step further,
in allowing access to local labels from other parts of the code. This
is achieved by means of \emph{defining} a local label in terms of the
previous non-local label: the first definition of \code{.loop} above is
really defining a symbol called \code{label1.loop}, and the second
defines a symbol called \code{label2.loop}. So, if you really needed
to, you could write

\begin{lstlisting}
label3  ; some more code
    ; and some more

    jmp label1.loop
\end{lstlisting}

Sometimes it is useful - in a macro, for instance - to be able to
define a label which can be referenced from anywhere but which
doesn't interfere with the normal local-label mechanism. Such a
label can't be non-local because it would interfere with subsequent
definitions of, and references to, local labels; and it can't be
local because the macro that defined it wouldn't know the label's
full name. NASM therefore introduces a third type of label, which is
probably only useful in macro definitions: if a label begins with
the \index{label prefix}special prefix \codeindex{..@}, then it
does nothing to the local label mechanism. So you could code

\begin{lstlisting}
label1:             ; a non-local label
.local:             ; this is really label1.local
..@foo:             ; this is a special symbol
label2:             ; another non-local label
.local:             ; this is really label2.local

    jmp     ..@foo  ; this will jump three lines up
\end{lstlisting}

NASM has the capacity to define other special symbols beginning with
a double period: for example, \code{..start} is used to specify the
entry point in the \code{obj} output format (see \nref{dotdotstart}),
\code{..imagebase} is used to find out the offset from a base address
of the current image in the \code{win64} output format
(see \nref{win64pic}). So just keep in mind that symbols
beginning with a double period are special.
