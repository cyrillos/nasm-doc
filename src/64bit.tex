%
% vim: ts=4 sw=4 et
%
\xchapter{64bit}{Writing 64-bit Code (Unix, Win64)}

This chapter attempts to cover some of the common issues involved when
writing 64-bit code, to run under \textindex{Win64} or Unix. It covers
how to write assembly code to interface with 64-bit C routines, and
how to write position-independent code for shared libraries.

All 64-bit code uses a flat memory model, since segmentation is not
available in 64-bit mode. The one exception is the \code{FS} and
\code{GS} registers, which still add their bases.

Position independence in 64-bit mode is significantly simpler, since
the processor supports \code{RIP}-relative addressing directly; see the
\code{REL} keyword (\nref{effaddr}). On most 64-bit platforms,
it is probably desirable to make that the default, using the directive
\code{DEFAULT REL} (\nref{default}).

64-bit programming is relatively similar to 32-bit programming, but
of course pointers are 64 bits long; additionally, all existing
platforms pass arguments in registers rather than on the stack.
Furthermore, 64-bit platforms use SSE2 by default for floating point.
Please see the ABI documentation for your platform.

64-bit platforms differ in the sizes of the C/C++ fundamental
datatypes, not just from 32-bit platforms but from each other. If a
specific size data type is desired, it is probably best to use the
types defined in the standard C header \code{<inttypes.h>}.

All known 64-bit platforms except some embedded platforms require that
the stack is 16-byte aligned at the entry to a function. In order to
enforce that, the stack pointer (\code{RSP}) needs to be aligned on an
\code{odd} multiple of 8 bytes before the \code{CALL} instruction.

In 64-bit mode, the default instruction size is still 32 bits. When
loading a value into a 32-bit register (but not an 8- or 16-bit
register), the upper 32 bits of the corresponding 64-bit register are
set to zero.

\xsection{reg64}{Register Names in 64-bit Mode}

NASM uses the following names for general-purpose registers in 64-bit
mode, for 8-, 16-, 32- and 64-bit references, respectively:

\begin{lstlisting}
AL/AH, CL/CH, DL/DH, BL/BH, SPL, BPL, SIL, DIL, R8B-R15B
AX, CX, DX, BX, SP, BP, SI, DI, R8W-R15W
EAX, ECX, EDX, EBX, ESP, EBP, ESI, EDI, R8D-R15D
RAX, RCX, RDX, RBX, RSP, RBP, RSI, RDI, R8-R15
\end{lstlisting}

This is consistent with the AMD documentation and most other
assemblers. The Intel documentation, however, uses the names
\code{R8L-R15L} for 8-bit references to the higher registers. It is
possible to use those names by definiting them as macros; similarly,
if one wants to use numeric names for the low 8 registers, define them
as macros. The standard macro package \code{altreg} (see
\nref{pkg-altreg}) can be used for this purpose.

\xsection{id64}{Immediates and Displacements in 64-bit Mode}

In 64-bit mode, immediates and displacements are generally only 32
bits wide. NASM will therefore truncate most displacements and
immediates to 32 bits.

The only instruction which takes a full \textindex{64-bit immediate} is:

\begin{lstlisting}
mov reg64,imm64
\end{lstlisting}

NASM will produce this instruction whenever the programmer uses
\code{MOV} with an immediate into a 64-bit register. If this is not
desirable, simply specify the equivalent 32-bit register, which will
be automatically zero-extended by the processor, or specify the
immediate as \code{DWORD}:

\begin{lstlisting}
mov rax,foo         ; 64-bit immediate
mov rax,qword foo   ; (identical)
mov eax,foo         ; 32-bit immediate, zero-extended
mov rax,dword foo   ; 32-bit immediate, sign-extended
\end{lstlisting}

The length of these instructions are 10, 5 and 7 bytes, respectively.

If optimization is enabled and NASM can determine at assembly time
that a shorter instruction will suffice, the shorter instruction will
be emitted unless of course \code{STRICT QWORD} or \code{STRICT DWORD}
is specified (see \nref{strict}):

\begin{lstlisting}
mov rax,1               ; Assembles as "mov eax,1" (5 bytes)
mov rax,strict qword 1  ; Full 10-byte instruction
mov rax,strict dword 1  ; 7-byte instruction
mov rax,symbol          ; 10 bytes, not known at assembly time
lea rax,[rel symbol]    ; 7 bytes, usually preferred by the ABI
\end{lstlisting}

Note that \code{lea rax,[rel symbol]} is position-independent, whereas
\code{mov rax,symbol} is not. Most ABIs prefer or even require
position-independent code in 64-bit mode. However, the \code{MOV}
instruction is able to reference a symbol anywhere in the 64-bit
address space, whereas \code{LEA} is only able to access a symbol within
within 2 GB of the instruction itself (see below.)

The only instructions which take a full \textindex{64-bit displacement}
is loading or storing, using \code{MOV}, \code{AL}, \code{AX}, \code{EAX}
or \code{RAX} (but no other registers) to an absolute 64-bit address.
Since this is a relatively rarely used instruction (64-bit code
generally uses relative addressing), the programmer has to explicitly
declare the displacement size as \code{QWORD}:

\begin{lstlisting}
default abs

mov eax,[foo]           ; 32-bit absolute disp, sign-extended
mov eax,[a32 foo]       ; 32-bit absolute disp, zero-extended
mov eax,[qword foo]     ; 64-bit absolute disp

default rel

mov eax,[foo]           ; 32-bit relative disp
mov eax,[a32 foo]       ; d:o, address truncated to 32 bits(!)
mov eax,[qword foo]     ; error
mov eax,[abs qword foo] ; 64-bit absolute disp
\end{lstlisting}

A sign-extended absolute displacement can access from -2 GB to +2 GB;
a zero-extended absolute displacement can access from 0 to 4 GB.

\xsection{unix64}{Interfacing to 64-bit C Programs (Unix)}

On Unix, the 64-bit ABI as well as the x32 ABI (32-bit ABI with the
CPU in 64-bit mode) is defined by the documents at
\href{http://www.nasm.us/abi/unix64}{http://www.nasm.us/abi/unix64}

Although written for AT\&T-syntax assembly, the concepts apply equally
well for NASM-style assembly. What follows is a simplified summary.

The first six integer arguments (from the left) are passed in \code{RDI},
\code{RSI}, \code{RDX}, \code{RCX}, \code{R8}, and \code{R9}, in that
order. Additional integer arguments are passed on the stack. These
registers, plus \code{RAX}, \code{R10} and \code{R11} are destroyed
by function calls, and thus are available for use by the function
without saving.

Integer return values are passed in \code{RAX} and \code{RDX},
in that order.

Floating point is done using SSE registers, except for \code{long double}.
Floating-point arguments are passed in \code{XMM0} to \code{XMM7};
return is \code{XMM0} and \code{XMM1}. \code{long double} are passed
on the stack, and returned in \code{ST0} and \code{ST1}.

All SSE and x87 registers are destroyed by function calls.

On 64-bit Unix, \code{long} is 64 bits.

Integer and SSE register arguments are counted separately, so
for the case of

\begin{lstlisting}
void foo(long a, double b, int c)
\end{lstlisting}

\code{a} is passed in \code{RDI}, \code{b} in \code{XMM0},
and \code{c} in \code{ESI}.

\xsection{win64}{Interfacing to 64-bit C Programs (Win64)}

The Win64 ABI is described by the document at
\href{http://www.nasm.us/abi/win64}{http://www.nasm.us/abi/win64}

What follows is a simplified summary.

The first four integer arguments are passed in \code{RCX}, \code{RDX},
\code{R8} and \code{R9}, in that order. Additional integer arguments are
passed on the stack. These registers, plus \code{RAX}, \code{R10} and
\code{R11} are destroyed by function calls, and thus are available for
use by the function without saving.

Integer return values are passed in \code{RAX} only.

Floating point is done using SSE registers, except for \code{long
double}. Floating-point arguments are passed in \code{XMM0}
to \code{XMM3}; return is \code{XMM0} only.

On Win64, \code{long} is 32 bits; \code{long long} or \code{\_int64}
is 64 bits.

Integer and SSE register arguments are counted together, so
for the case of

\begin{lstlisting}
void foo(long long a, double b, int c)
\end{lstlisting}

\code{a} is passed in \code{RCX}, \code{b} in \code{XMM1},
and \code{c} in \code{R8D}.
