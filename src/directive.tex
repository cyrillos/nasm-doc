\chapter{\textindexlc{Assembler Directives}}
\label{ch:directive}

NASM, though it attempts to avoid the bureaucracy of assemblers like
MASM and TASM, is nevertheless forced to support a \emph{few}
directives. These are described in this chapter.

NASM's directives come in two types: \index{directives!user-level}
\emph{user-level} directives and \index{directives!primitive}
\emph{primitive} directives. Typically, each directive has a
user-level form and a primitive form. In almost all cases, we
recommend that users use the user-level forms of the directives,
which are implemented as macros which call the primitive forms.

Primitive directives are enclosed in square brackets; user-level
directives are not.

In addition to the universal directives described in this chapter,
each object file format can optionally supply extra directives in
order to control particular features of that file format. These
\index{directives!format-specific}\emph{format-specific} directives are
documented along with the formats that implement them, in
\fullref{sec:outfmt}.

\section{\codeindex{BITS}: Specifying Target \textindexlc{Processor Mode}}
\label{sec:bits}

The \code{BITS} directive specifies whether NASM should generate code
\index{16-bit mode, versus 32-bit mode}designed to run on a processor
operating in 16-bit mode, 32-bit mode or 64-bit mode. The syntax is
\code{BITS XX}, where XX is 16, 32 or 64.

In most cases, you should not need to use \code{BITS} explicitly. The
\code{aout}, \code{coff}, \code{elf32}, \code{elf64}, \code{macho32},
\code{macho64}, \code{win32} and \code{win64} object formats, which
are designed for use in 32-bit or 64-bit operating systems, all cause
NASM to select 32-bit or 64-bit mode, respectively, by default.
The \code{obj} object format allows you to specify each segment
you define as either \code{USE16} or \code{USE32}, and NASM will
set its operating mode accordingly, so the use of the \code{BITS}
directive is once again unnecessary.

The most likely reason for using the \code{BITS} directive is to write
32-bit or 64-bit code in a flat binary file; this is because the \code{bin}
output format defaults to 16-bit mode in anticipation of it being
used most frequently to write DOS \code{.COM} programs, DOS \code{.SYS}
device drivers and boot loader software.

The \code{BITS} directive can also be used to generate code for
a different mode than the standard one for the output format.

You do \emph{not} need to specify \code{BITS 32} merely in order
to use 32-bit instructions in a 16-bit DOS program; if you do, the
assembler will generate incorrect code because it will be writing
code targeted at a 32-bit platform, to be run on a 16-bit one.

When NASM is in \code{BITS 16} mode, instructions which use 32-bit
data are prefixed with an 0x66 byte, and those referring to 32-bit
addresses have an 0x67 prefix. In \code{BITS 32} mode, the reverse is
true: 32-bit instructions require no prefixes, whereas instructions
using 16-bit data need an 0x66 and those working on 16-bit
addresses need an 0x67.

When NASM is in \code{BITS 64} mode, most instructions operate the same
as they do for \code{BITS 32} mode. However, there are 8 more general and
SSE registers, and 16-bit addressing is no longer supported.

The default address size is 64 bits; 32-bit addressing can be selected
with the 0x67 prefix. The default operand size is still 32 bits,
however, and the 0x66 prefix selects 16-bit operand size.
The \code{REX} prefix is used both to select 64-bit operand size, and
to access the new registers. NASM automatically inserts REX prefixes
when necessary.

When the \code{REX} prefix is used, the processor does not know how to
address the AH, BH, CH or DH (high 8-bit legacy) registers. Instead,
it is possible to access the the low 8-bits of the SP, BP SI and DI
registers as SPL, BPL, SIL and DIL, respectively; but only when the
REX prefix is used.

The \code{BITS} directive has an exactly equivalent primitive form,
\code{[BITS 16]}, \code{[BITS 32]} and \code{[BITS 64]}. The user-level
form is a macro which has no function other than to call the primitive form.

Note that the space is neccessary, e.g. \code{BITS32} will \emph{not} work!

\subsection{\codeindex{USE16}, \codeindex{USE32} and \codeindex{USE64}: Aliases for BITS}
\label{subsec:use163264}

The \code{USE16}, \code{USE32} and \code{USE64} directives can be used
in place of \code{BITS 16}, \code{BITS 32} and \code{BITS 64}, for
compatibility with other assemblers.

\section{\codeindex{DEFAULT}: Change the assembler defaults}
\label{sec:default}

The \code{DEFAULT} directive changes the assembler defaults. Normally,
NASM defaults to a mode where the programmer is expected to explicitly
specify most features directly. However, this is occasionally obnoxious,
as the explicit form is pretty much the only one one wishes to use.

Currently, \code{DEFAULT} can be set to \code{REL}, \code{ABS}, \code{BND}
and \code{NOBND}.

\subsection{\codeindex{REL} and \codeindex{ABS}: RIP-relative addressing}
\label{subsec:relabs}

This sets whether registerless instructions in 64-bit mode are
\code{RIP}-relative or not. By default, they are absolute unless
overridden with the \codeindex{REL} specifier (see \fullref{sec:effaddr}).
However, if \code{DEFAULT REL} is specified, \code{REL} is default, unless
overridden with the \code{ABS} specifier, \emph{except when used with an
FS or GS segment override}.

The special handling of \code{FS} and \code{GS} overrides are due to the
fact that these registers are generally used as thread pointers or
other special functions in 64-bit mode, and generating
\code{RIP}-relative addresses would be extremely confusing.

\code{DEFAULT REL} is disabled with \code{DEFAULT ABS}.

\subsection{\codeindex{BND} and \codeindex{NOBND}: \code{BND} prefix}
\label{subsec:bndnobnd}

If \code{DEFAULT BND} is set, all bnd-prefix available instructions
following this directive are prefixed with bnd. To override it,
\code{NOBND} prefix can be used.

\begin{lstlisting}
DEFAULT BND
    call foo        ; BND will be prefixed
    nobnd call foo  ; BND will NOT be prefixed
\end{lstlisting}

\code{DEFAULT NOBND} can disable \code{DEFAULT BND} and then
\code{BND} prefix will be added only when explicitly specified
in code.

\code{DEFAULT BND} is expected to be the normal configuration
for writing MPX-enabled code.

\section{\codeindex{SECTION} or \codeindex{SEGMENT}: Changing and
\textindexlc{Defining Sections}}
\label{sec:section}

\index{sections!changing}\index{sections!switching between}
The \code{SECTION} directive (\code{SEGMENT} is an exactly equivalent
synonym) changes which section of the output file the code you write
will be assembled into. In some object file formats, the number and
names of sections are fixed; in others, the user may make up as many
as they wish. Hence \code{SECTION} may sometimes give an error message,
or may define a new section, if you try to switch to a section that does
not (yet) exist.

The Unix object formats, and the \code{bin} object format (but see
\fullref{sec:multisec}), all support the \index{sections!standardized names}
standardized names \code{.text}, \code{.data} and \code{.bss} for the code,
data and uninitialized-data sections. The \code{obj} format, by contrast,
does not recognize these section names as being special, and indeed will
strip off the leading period of any section name that has one.

\subsection{The \codeindex{\_\_SECT\_\_} Macro}
\label{subsec:sectmac}

The \code{SECTION} directive is unusual in that its user-level form
functions differently from its primitive form. The primitive form,
\code{[SECTION xyz]}, simply switches the current target section to the
one given. The user-level form, \code{SECTION xyz}, however, first
defines the single-line macro \code{\_\_SECT\_\_} to be the primitive
\code{[SECTION]} directive which it is about to issue, and then issues
it. So the user-level directive

\begin{lstlisting}
    SECTION .text
\end{lstlisting}

expands to the two lines

\begin{lstlisting}
%define __SECT__    [SECTION .text]
    [SECTION .text]
\end{lstlisting}

Users may find it useful to make use of this in their own macros.
For example, the \code{writefile} macro defined in \fullref{sec:mlmacgre}
can be usefully rewritten in the following more sophisticated form:

\begin{lstlisting}
%macro  writefile 2+
        [section .data]

    %%str:      db %2
    %%endstr:

        __SECT__

        mov     dx, %%str
        mov     cx, %%endstr-%%str
        mov     bx, %1
        mov     ah, 0x40
        int     0x21
%endmacro
\end{lstlisting}

This form of the macro, once passed a string to output, first
switches temporarily to the data section of the file, using the
primitive form of the \code{SECTION} directive so as not to modify
\code{\_\_SECT\_\_}. It then declares its string in the data section,
and then invokes \code{\_\_SECT\_\_} to switch back to \emph{whichever}
section the user was previously working in. It thus avoids the need,
in the previous version of the macro, to include a \code{JMP} instruction
to jump over the data, and also does not fail if, in a complicated
\code{OBJ} format module, the user could potentially be assembling the
code in any of several separate code sections.

\section{\codeindex{ABSOLUTE}: Defining Absolute Labels}
\label{subsec:absolute}

The \code{ABSOLUTE} directive can be thought of as an alternative form
of \code{SECTION}: it causes the subsequent code to be directed at no
physical section, but at the hypothetical section starting at the
given absolute address. The only instructions you can use in this
mode are the \code{RESB} family.

\code{ABSOLUTE} is used as follows:

\begin{lstlisting}
absolute 0x1A

    kbuf_chr    resw    1
    kbuf_free   resw    1
    kbuf        resw    16
\end{lstlisting}

This example describes a section of the PC BIOS data area, at
segment address 0x40: the above code defines \code{kbuf\_chr} to be
0x1A, \code{kbuf\_free} to be 0x1C, and \code{kbuf} to be 0x1E.

The user-level form of \code{ABSOLUTE}, like that of \code{SECTION},
redefines the \codeindex{\_\_SECT\_\_} macro when it is invoked.

\codeindex{STRUC} and \codeindex{ENDSTRUC} are defined as macros
which use \code{ABSOLUTE} (and also \code{\_\_SECT\_\_}).

\code{ABSOLUTE} doesn't have to take an absolute constant as an
argument: it can take an expression (actually, a \textindex{critical
expression}: see \fullref{sec:crit}) and it can be a value in a segment.
For example, a TSR can re-use its setup code as run-time BSS like this:

\begin{lstlisting}
        org     100h        ; it's a .COM program
        jmp     setup       ; setup code comes last
        ; the resident part of the TSR goes here
        ; ...
setup:
        ; now write the code that installs the TSR here
        ; ...
absolute setup

runtimevar1     resw    1
runtimevar2     resd    20

tsr_end:
\end{lstlisting}

This defines some variables ``on top of'' the setup code, so that
after the setup has finished running, the space it took up can be
re-used as data storage for the running TSR. The symbol
\code{tsr\_end} can be used to calculate the total size of
the part of the TSR that needs to be made resident.

\section{\codeindex{EXTERN}: \textindexlc{Importing Symbols} from Other Modules}
\label{sec:extern}

\code{EXTERN} is similar to the MASM directive \code{EXTRN} and
the C keyword \code{extern}: it is used to declare a symbol which
is not defined anywhere in the module being assembled, but is assumed
to be defined in some other module and needs to be referred to by this
one. Not every object-file format can support external variables:
the \code{bin} format cannot.

The \code{EXTERN} directive takes as many arguments as you like.
Each argument is the name of a symbol:

\begin{lstlisting}
extern  _printf
extern  _sscanf,_fscanf
\end{lstlisting}

Some object-file formats provide extra features to the \code{EXTERN}
directive. In all cases, the extra features are used by suffixing a
colon to the symbol name followed by object-format specific text.
For example, the \code{obj} format allows you to declare that the
default segment base of an external should be the group \code{dgroup}
by means of the directive

\begin{lstlisting}
extern  _variable:wrt dgroup
\end{lstlisting}

The primitive form of \code{EXTERN} differs from the user-level form
only in that it can take only one argument at a time: the support
for multiple arguments is implemented at the preprocessor level.

You can declare the same variable as \code{EXTERN} more than once:
NASM will quietly ignore the second and later redeclarations.
You can't declare a variable as \code{EXTERN} as well as something
else, though.
