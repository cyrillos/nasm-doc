%
% vim: ts=4 sw=4 et
%
\xchapter{ndisasm}{Ndisasm}

The Netwide Disassembler, NDISASM.

\xsection{ndisintro}{Introduction}

The Netwide Disassembler is a small companion program to the Netwide
Assembler, NASM. It seemed a shame to have an x86 assembler,
complete with a full instruction table, and not make as much use of
it as possible, so here's a disassembler which shares the
instruction table (and some other bits of code) with NASM.

The Netwide Disassembler does nothing except to produce
disassemblies of \emph{binary} source files. NDISASM does not have any
understanding of object file formats, like \code{objdump}, and it will
not understand \code{DOS .EXE} files like \code{debug} will. It just
disassembles.

\xsection{ndisrun}{Running NDISASM}

To disassemble a file, you will typically use a command of the form

\begin{lstlisting}
ndisasm -b {16|32|64} filename
\end{lstlisting}

NDISASM can disassemble 16-, 32- or 64-bit code equally easily,
provided of course that you remember to specify which it is to work
with. If no \codeindex{-b} switch is present, NDISASM works in 16-bit mode
by default. The \codeindex{-u} switch (for USE32) also invokes 32-bit mode.

Two more command line options are \codeindex{-r} which reports the version
number of NDISASM you are running, and \codeindex{-h} which gives a short
summary of command line options.

\xsubsection{ndiscom}{COM Files: Specifying an Origin}

To disassemble a \code{DOS .COM} file correctly, a disassembler must
assume that the first instruction in the file is loaded at address
\code{0x100}, rather than at zero. NDISASM, which assumes by default
that any file you give it is loaded at zero, will therefore need
to be informed of this.

The \codeindex{-o} option allows you to declare a different origin
for the file you are disassembling. Its argument may be expressed
in any of the NASM numeric formats: decimal by default, if it begins
with `\code{\$}' or `\code{0x}' or ends in `\code{H}' it's \code{hex},
if it ends in `\code{Q}' it's \code{octal}, and if it ends in
`\code{B}' it's \code{binary}.

Hence, to disassemble a \code{.COM} file:

\begin{lstlisting}
ndisasm -o100h filename.com
\end{lstlisting}

will do the trick.

\xsubsection{ndissync}{Code Following Data: Synchronisation}

Suppose you are disassembling a file which contains some data which
isn't machine code, and \emph{then} contains some machine code. NDISASM
will faithfully plough through the data section, producing machine
instructions wherever it can (although most of them will look
bizarre, and some may have unusual prefixes, e.g. `\code{FS OR AX,0x240A}'),
and generating `DB' instructions ever so often if it's totally stumped.
Then it will reach the code section.

Supposing NDISASM has just finished generating a strange machine
instruction from part of the data section, and its file position is
now one byte \emph{before} the beginning of the code section. It's
entirely possible that another spurious instruction will get
generated, starting with the final byte of the data section, and
then the correct first instruction in the code section will not be
seen because the starting point skipped over it. This isn't really
ideal.

To avoid this, you can specify a `\codeindex{synchronisation}' point, or indeed
as many synchronisation points as you like (although NDISASM can
only handle 2147483647 sync points internally). The definition of a sync
point is this: NDISASM guarantees to hit sync points exactly during
disassembly. If it is thinking about generating an instruction which
would cause it to jump over a sync point, it will discard that
instruction and output a `\code{db}' instead. So it \emph{will} start
disassembly exactly from the sync point, and so you \emph{will} see all
the instructions in your code section.

Sync points are specified using the \codeindex{-s} option: they are measured
in terms of the program origin, not the file position. So if you
want to synchronize after 32 bytes of a \codeindex{.COM} file, you would have to
do

\begin{lstlisting}
ndisasm -o100h -s120h file.com
\end{lstlisting}

rather than

\begin{lstlisting}
ndisasm -o100h -s20h file.com
\end{lstlisting}

As stated above, you can specify multiple sync markers if you need
to, just by repeating the \code{-s} option.


\xsubsection{ndisisync}{Mixed Code and Data: Automatic (Intelligent)
Synchronisation}
\indexcode{auto-sync}

Suppose you are disassembling the boot sector of a \code{DOS} floppy (maybe
it has a virus, and you need to understand the virus so that you
know what kinds of damage it might have done you). Typically, this
will contain a \code{JMP} instruction, then some data, then the rest of the
code. So there is a very good chance of NDISASM being \emph{misaligned}
when the data ends and the code begins. Hence a sync point is
needed.

On the other hand, why should you have to specify the sync point
manually? What you'd do in order to find where the sync point would
be, surely, would be to read the \code{JMP} instruction, and then to use
its target address as a sync point. So can NDISASM do that for you?

The answer, of course, is yes: using either of the synonymous
switches \codeindex{-a} (for automatic sync) or \codeindex{-i}
(for intelligent sync) will enable \code{auto-sync} mode. Auto-sync
mode automatically generates a sync point for any forward-referring
PC-relative jump or call instruction that NDISASM encounters. (Since
NDISASM is one-pass, if it encounters a PC-relative jump whose target
has already been processed, there isn't much it can do about it...)

Only PC-relative jumps are processed, since an absolute jump is
either through a register (in which case NDISASM doesn't know what
the register contains) or involves a segment address (in which case
the target code isn't in the same segment that NDISASM is working
in, and so the sync point can't be placed anywhere useful).

For some kinds of file, this mechanism will automatically put sync
points in all the right places, and save you from having to place
any sync points manually. However, it should be stressed that
auto-sync mode is \emph{not} guaranteed to catch all the sync points, and
you may still have to place some manually.

Auto-sync mode doesn't prevent you from declaring manual sync
points: it just adds automatically generated ones to the ones you
provide. It's perfectly feasible to specify \code{-i} \emph{and}
some \code{-s} options.

Another caveat with auto-sync mode is that if, by some unpleasant
fluke, something in your data section should disassemble to a
PC-relative call or jump instruction, NDISASM may obediently place a
sync point in a totally random place, for example in the middle of
one of the instructions in your code section. So you may end up with
a wrong disassembly even if you use auto-sync. Again, there isn't
much I can do about this. If you have problems, you'll have to use
manual sync points, or use the \code{-k} option (documented below) to
suppress disassembly of the data area.

\xsubsection{ndisother}{Other Options}

The \codeindex{-e} option skips a header on the file, by ignoring the first N
bytes. This means that the header is \emph{not} counted towards the
disassembly offset: if you give \code{-e10 -o10}, disassembly will start
at byte 10 in the file, and this will be given offset 10, not 20.

The \codeindex{-k} option is provided with two comma-separated numeric
arguments, the first of which is an assembly offset and the second
is a number of bytes to skip. This \emph{will} count the skipped bytes
towards the assembly offset: its use is to suppress disassembly of a
data section which wouldn't contain anything you wanted to see
anyway.
