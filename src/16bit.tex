%
% vim: ts=4 sw=4 et
%
\xchapter{16bit}{Writing 16-bit Code (DOS, Windows 3/3.1)}

This chapter attempts to cover some of the common issues encountered
when writing 16-bit code to run under \code{MS-DOS} or \code{Windows 3.x}.
It covers how to link programs to produce \code{.EXE} or \code{.COM} files,
how to write \code{.SYS} device drivers, and how to interface assembly
language code with 16-bit C compilers and with Borland Pascal.

\xsection{exefiles}{Producing \codeindex{.EXE} Files}

Any large program written under DOS needs to be built as a \code{.EXE}
file: only \code{.EXE} files have the necessary internal structure
required to span more than one 64K segment. \textindex{Windows} programs,
also, have to be built as \code{.EXE} files, since Windows does not
support the \code{.COM} format.

In general, you generate \code{.EXE} files by using the \code{obj} output
format to produce one or more \codeindex{.OBJ} files, and then linking
them together using a linker. However, NASM also supports the direct
generation of simple DOS \code{.EXE} files using the \code{bin} output
format (by using \code{DB} and \code{DW} to construct the \code{.EXE} file
header), and a macro package is supplied to do this. Thanks to
Yann Guidon for contributing the code for this.

NASM may also support \code{.EXE} natively as another output format in
future releases.

\subsection{Using the \code{obj} Format To Generate \code{.EXE} Files}
\label{subsec:objexe}

This section describes the usual method of generating \code{.EXE} files
by linking \code{.OBJ} files together.

Most 16-bit programming language packages come with a suitable
linker; if you have none of these, there is a free linker called
\textindex{VALX}\index{linker!VALX}, available as a part of
CC386 compiler on \href{http://ladsoft.tripod.com/cc386\_compiler.html}
{ladsoft.tripod.com}.

There is another `free' linker (though this one doesn't come with
sources) called \textindex{FREELINK}\index{linker!FREELINK}, available
from \href{http://www.pcorner.com/tpc/old/3-101.html}{www.pcorner.com}.

A third, \textindex{djlink}, written by DJ Delorie, is available at
\href{http://www.delorie.com/djgpp/16bit/djlink/}{www.delorie.com}.

A fourth linker, \textindex{ALINK}\index{linker!ALINK}, written by
Anthony A.J. Williams, is available at \href{http://alink.sourceforge.net}
{alink.sourceforge.net}.

When linking several \code{.OBJ} files into a \code{.EXE} file, you should
ensure that exactly one of them has a start point defined (using the
\index{program entry point}\codeindex{..start} special symbol defined by the
\code{obj} format: see \fullref{sec:dotdotstart}). If no module defines a start
point, the linker will not know what value to give the entry-point
field in the output file header; if more than one defines a start
point, the linker will not know \emph{which} value to use.

An example of a NASM source file which can be assembled to a
\code{.OBJ} file and linked on its own to a \code{.EXE} is given here. It
demonstrates the basic principles of defining a stack, initialising
the segment registers, and declaring a start point. This file is
also provided in the \index{test subdirectory}\code{test} subdirectory of
the NASM archives, under the name \code{objexe.asm}.

\begin{lstlisting}
segment code

..start:
    mov     ax,data
    mov     ds,ax
    mov     ax,stack
    mov     ss,ax
    mov     sp,stacktop
\end{lstlisting}

This initial piece of code sets up \code{DS} to point to the data
segment, and initializes \code{SS} and \code{SP} to point to the top of
the provided stack. Notice that interrupts are implicitly disabled
for one instruction after a move into \code{SS}, precisely for this
situation, so that there's no chance of an interrupt occurring
between the loads of \code{SS} and \code{SP} and not having a stack to
execute on.

Note also that the special symbol \code{..start} is defined at the
beginning of this code, which means that will be the entry point
into the resulting executable file.

\begin{lstlisting}
    mov     dx,hello
    mov     ah,9
    int     0x21
\end{lstlisting}

The above is the main program: load \code{DS:DX} with a pointer to the
greeting message (\code{hello} is implicitly relative to the segment
\code{data}, which was loaded into \code{DS} in the setup code, so the
full pointer is valid), and call the DOS print-string function.

\begin{lstlisting}
    mov     ax,0x4c00
    int     0x21
\end{lstlisting}

This terminates the program using another DOS system call.

\begin{lstlisting}
segment data

hello:  db      'hello, world', 13, 10, '$'
\end{lstlisting}

The data segment contains the string we want to display.

\begin{lstlisting}
segment stack stack
    resb 64
stacktop:
\end{lstlisting}

The above code declares a stack segment containing 64 bytes of
uninitialized stack space, and points \code{stacktop} at the top of it.
The directive \code{segment stack stack} defines a segment \emph{called}
\code{stack}, and also of \emph{type} \code{STACK}. The latter is not
necessary to the correct running of the program, but linkers are
likely to issue warnings or errors if your program has no segment of
type \code{STACK}.

The above file, when assembled into a \code{.OBJ} file, will link on
its own to a valid \code{.EXE} file, which when run will print `hello,
world' and then exit.

\subsection{Using the \code{bin} Format To Generate \code{.EXE} Files}
\label{subsec:binexe}

The \code{.EXE} file format is simple enough that it's possible to
build a \code{.EXE} file by writing a pure-binary program and sticking
a 32-byte header on the front. This header is simple enough that it
can be generated using \code{DB} and \code{DW} commands by NASM itself,
so that you can use the \code{bin} output format to directly generate
\code{.EXE} files.

Included in the NASM archives, in the \index{misc subdirectory}\code{misc}
subdirectory, is a file \codeindex{exebin.mac} of macros. It defines three
macros: \codeindex{EXE\_begin}, \codeindex{EXE\_stack} and
\codeindex{EXE\_end}.

To produce a \code{.EXE} file using this method, you should start by
using \code{\%include} to load the \code{exebin.mac} macro package into
your source file. You should then issue the \code{EXE\_begin} macro call
(which takes no arguments) to generate the file header data. Then
write code as normal for the \code{bin} format - you can use all three
standard sections \code{.text}, \code{.data} and \code{.bss}. At the end of
the file you should call the \code{EXE\_end} macro (again, no arguments),
which defines some symbols to mark section sizes, and these symbols
are referred to in the header code generated by \code{EXE\_begin}.

In this model, the code you end up writing starts at \code{0x100}, just
like a \code{.COM} file - in fact, if you strip off the 32-byte header
from the resulting \code{.EXE} file, you will have a valid \code{.COM}
program. All the segment bases are the same, so you are limited to a
64K program, again just like a \code{.COM} file. Note that an \code{ORG}
directive is issued by the \code{EXE\_begin} macro, so you should not
explicitly issue one of your own.

You can't directly refer to your segment base value, unfortunately,
since this would require a relocation in the header, and things
would get a lot more complicated. So you should get your segment
base by copying it out of \code{CS} instead.

On entry to your \code{.EXE} file, \code{SS:SP} are already set up to
point to the top of a 2Kb stack. You can adjust the default stack
size of 2Kb by calling the \code{EXE\_stack} macro. For example, to
change the stack size of your program to 64 bytes, you would call
\code{EXE\_stack 64}.

A sample program which generates a \code{.EXE} file in this way is
given in the \code{test} subdirectory of the NASM archive, as
\code{binexe.asm}.

\xsection{comfiles}{Producing \codeindex{.COM} Files}

While large DOS programs must be written as \code{.EXE} files, small
ones are often better written as \code{.COM} files. \code{.COM} files are
pure binary, and therefore most easily produced using the \code{bin}
output format.

\subsection{Using the \code{bin} Format To Generate \code{.COM} Files}
\label{subsec:combinfmt}

\code{.COM} files expect to be loaded at offset \code{100h} into their
segment (though the segment may change). Execution then begins at
\indexcode{ORG}\code{100h}, i.e. right at the start of the program.
So to write a \code{.COM} program, you would create a source file
looking like

\begin{lstlisting}
        org 100h

section .text
start:
        ; put your code here

section .data
        ; put data items here

section .bss
        ; put uninitialized data here
\end{lstlisting}

The \code{bin} format puts the \code{.text} section first in the file,
so you can declare data or BSS items before beginning to write code if
you want to and the code will still end up at the front of the file
where it belongs.

The BSS (uninitialized data) section does not take up space in the
\code{.COM} file itself: instead, addresses of BSS items are resolved
to point at space beyond the end of the file, on the grounds that
this will be free memory when the program is run. Therefore you
should not rely on your BSS being initialized to all zeros when you
run.

To assemble the above program, you should use a command line like

\begin{lstlisting}
nasm myprog.asm -fbin -o myprog.com
\end{lstlisting}

The \code{bin} format would produce a file called \code{myprog} if no
explicit output file name were specified, so you have to override it
and give the desired file name.

\subsection{Using the \code{obj} Format To Generate \code{.COM} Files}
\label{subsec:comobjfmt}

If you are writing a \code{.COM} program as more than one module, you
may wish to assemble several \code{.OBJ} files and link them together
into a \code{.COM} program. You can do this, provided you have a linker
capable of outputting \code{.COM} files directly (\textindex{TLINK} does this),
or alternatively a converter program such as \codeindex{EXE2BIN} to
transform the \code{.EXE} file output from the linker into a \code{.COM}
file.

If you do this, you need to take care of several things:

\begin{itemize}
    \item{The first object file containing code should start its code
        segment with a line like \code{RESB 100h}. This is to ensure
        that the code begins at offset \code{100h} relative to the beginning
        of the code segment, so that the linker or converter program does
        not have to adjust address references within the file when generating
        the \code{.COM} file. Other assemblers use an \codeindex{ORG} directive
        for this purpose, but \code{ORG} in NASM is a format-specific directive
        to the \code{bin} output format, and does not mean the same thing as
        it does in MASM-compatible assemblers.}
    \item{You don't need to define a stack segment.}
    \item{All your segments should be in the same group, so that every time
        your code or data references a symbol offset, all offsets are
        relative to the same segment base. This is because, when a \code{.COM}
        file is loaded, all the segment registers contain the same value.}
\end{itemize}

\xsection{sysfiles}{Producing \codeindex{.SYS} Files}

\textindex{MS-DOS device drivers} - \code{.SYS} files - are pure binary files,
similar to \code{.COM} files, except that they start at origin zero
rather than \code{100h}. Therefore, if you are writing a device driver
using the \code{bin} format, you do not need the \code{ORG} directive,
since the default origin for \code{bin} is zero. Similarly, if you are
using \code{obj}, you do not need the \code{RESB 100h} at the start of
your code segment.

\code{.SYS} files start with a header structure, containing pointers to
the various routines inside the driver which do the work. This
structure should be defined at the start of the code segment, even
though it is not actually code.

For more information on the format of \code{.SYS} files, and the data
which has to go in the header structure, a list of books is given in
the Frequently Asked Questions list for the newsgroup
\href{news:comp.os.msdos.programmer}{comp.os.msdos.programmer}.

\xsection{16c}{Interfacing to 16-bit C Programs}

This section covers the basics of writing assembly routines that
call, or are called from, C programs. To do this, you would
typically write an assembly module as a \code{.OBJ} file, and link it
with your C modules to produce a \textindex{mixed-language program}.

\subsection{External Symbol Names}
\label{subsec:16cunder}

\index{C symbol names}\index{underscore!in C symbols}C compilers have the
convention that the names of all global symbols (functions or data)
they define are formed by prefixing an underscore to the name as it
appears in the C program. So, for example, the function a C
programmer thinks of as \code{printf} appears to an assembly language
programmer as \code{\_printf}. This means that in your assembly
programs, you can define symbols without a leading underscore, and
not have to worry about name clashes with C symbols.

If you find the underscores inconvenient, you can define macros to
replace the \code{GLOBAL} and \code{EXTERN} directives as follows:

\begin{lstlisting}
%macro  cglobal 1
    global  _%1
    %define %1 _%1
%endmacro

%macro  cextern 1
    extern  _%1
    %define %1 _%1
%endmacro
\end{lstlisting}

(These forms of the macros only take one argument at a time; a
\code{\%rep} construct could solve this.)

If you then declare an external like this:

\begin{lstlisting}
cextern printf
\end{lstlisting}

then the macro will expand it as

\begin{lstlisting}
extern  _printf
%define printf _printf
\end{lstlisting}

Thereafter, you can reference \code{printf} as if it was a symbol, and
the preprocessor will put the leading underscore on where necessary.

The \code{cglobal} macro works similarly. You must use \code{cglobal}
before defining the symbol in question, but you would have had to do
that anyway if you used \code{GLOBAL}.

Also see \fullref{opt-pfix}.

\subsection{\textindexlc{Memory Models}}
\label{subsec:16cmodels}

NASM contains no mechanism to support the various C memory models
directly; you have to keep track yourself of which one you are
writing for. This means you have to keep track of the following
things:

\begin{itemize}
    \item{In models using a single code segment (tiny, small and compact),
        functions are near. This means that function pointers, when stored
        in data segments or pushed on the stack as function arguments, are
        16 bits long and contain only an offset field (the \code{CS} register
        never changes its value, and always gives the segment part of the
        full function address), and that functions are called using ordinary
        near \code{CALL} instructions and return using \code{RETN} (which, in
        NASM, is synonymous with \code{RET} anyway). This means both that you
        should write your own routines to return with \code{RETN}, and that you
        should call external C routines with near \code{CALL} instructions.}

    \item{In models using more than one code segment (medium, large and
        huge), functions are far. This means that function pointers are 32
        bits long (consisting of a 16-bit offset followed by a 16-bit
        segment), and that functions are called using \code{CALL FAR} (or
        \code{CALL seg:offset}) and return using \code{RETF}. Again, you should
        therefore write your own routines to return with \c{RETF} and use
        \code{CALL FAR} to call external routines.}

    \item{In models using a single data segment (tiny, small and medium),
        data pointers are 16 bits long, containing only an offset field (the
        \code{DS} register doesn't change its value, and always gives the
        segment part of the full data item address).}

    \item{In models using more than one data segment (compact, large and
        huge), data pointers are 32 bits long, consisting of a 16-bit offset
        followed by a 16-bit segment. You should still be careful not to
        modify \code{DS} in your routines without restoring it afterwards, but
        \code{ES} is free for you to use to access the contents of 32-bit data
        pointers you are passed.}

    \item{The huge memory model allows single data items to exceed 64K in
        size. In all other memory models, you can access the whole of a data
        item just by doing arithmetic on the offset field of the pointer you
        are given, whether a segment field is present or not; in huge model,
        you have to be more careful of your pointer arithmetic.}

    \item{In most memory models, there is a \emph{default} data segment, whose
        segment address is kept in \code{DS} throughout the program. This data
        segment is typically the same segment as the stack, kept in \code{SS},
        so that functions' local variables (which are stored on the stack)
        and global data items can both be accessed easily without changing
        \code{DS}. Particularly large data items are typically stored in other
        segments. However, some memory models (though not the standard
        ones, usually) allow the assumption that \code{SS} and \code{DS} hold the
        same value to be removed. Be careful about functions' local
        variables in this latter case.}
\end{itemize}

In models with a single code segment, the segment is called \codeindex{\_TEXT},
so your code segment must also go by this name in order to be linked into the
same place as the main code segment. In models with a single data segment,
or with a default data segment, it is called \codeindex{\_DATA}.

\subsection{Function Definitions and Function Calls}
\label{subsec:16cfunc}

\index{functions!C calling convention}The \textindex{C calling convention}
in 16-bit programs is as follows. In the following description, the
words \emph{caller} and \emph{callee} are used to denote the function
doing the calling and the function which gets called.

\begin{itemize}
    \item{The caller pushes the function's parameters on the stack, one
        after another, in reverse order (right to left, so that the first
        argument specified to the function is pushed last).}

    \item{The caller then executes a \code{CALL} instruction to pass control
        to the callee. This \code{CALL} is either near or far depending on the
        memory model.}

    \item{The callee receives control, and typically (although this is not
        actually necessary, in functions which do not need to access their
        parameters) starts by saving the value of \code{SP} in \code{BP} so as to
        be able to use \code{BP} as a base pointer to find its parameters on
        the stack. However, the caller was probably doing this too, so part
        of the calling convention states that \code{BP} must be preserved by
        any C function. Hence the callee, if it is going to set up \code{BP} as
        a \emph{\textindex{frame pointer}}, must push the previous value first.}

    \item{The callee may then access its parameters relative to \code{BP}.
        The word at \code{[BP]} holds the previous value of \code{BP} as it was
        pushed; the next word, at \code{[BP+2]}, holds the offset part of the
        return address, pushed implicitly by \code{CALL}. In a small-model
        (near) function, the parameters start after that, at \code{[BP+4]}; in
        a large-model (far) function, the segment part of the return address
        lives at \code{[BP+4]}, and the parameters begin at \code{[BP+6]}. The
        leftmost parameter of the function, since it was pushed last, is
        accessible at this offset from \code{BP}; the others follow, at
        successively greater offsets. Thus, in a function such as \code{printf}
        which takes a variable number of parameters, the pushing of the
        parameters in reverse order means that the function knows where to
        find its first parameter, which tells it the number and type of the
        remaining ones.}

    \item{The callee may also wish to decrease \code{SP} further, so as to
        allocate space on the stack for local variables, which will then be
        accessible at negative offsets from \code{BP}.}

    \item{The callee, if it wishes to return a value to the caller, should
        leave the value in \code{AL}, \code{AX} or \code{DX:AX} depending
        on the size of the value. Floating-point results are sometimes
        (depending on the compiler) returned in \code{ST0}.}

    \item{Once the callee has finished processing, it restores \code{SP} from
        \c{BP} if it had allocated local stack space, then pops the previous
        value of \code{BP}, and returns via \code{RETN} or \code{RETF} depending on
        memory model.}

    \item{When the caller regains control from the callee, the function
        parameters are still on the stack, so it typically adds an immediate
        constant to \code{SP} to remove them (instead of executing a number of
        slow \code{POP} instructions). Thus, if a function is accidentally
        called with the wrong number of parameters due to a prototype
        mismatch, the stack will still be returned to a sensible state since
        the caller, which \emph{knows} how many parameters it pushed, does the
        removing.}
\end{itemize}

It is instructive to compare this calling convention with that for
Pascal programs (described in \fullref{subsec:16bpfunc}). Pascal has
a simpler convention, since no functions have variable numbers of parameters.
Therefore the callee knows how many parameters it should have been
passed, and is able to deallocate them from the stack itself by
passing an immediate argument to the \code{RET} or \code{RETF}
instruction, so the caller does not have to do it. Also, the
parameters are pushed in left-to-right order, not right-to-left,
which means that a compiler can give better guarantees about
sequence points without performance suffering.

Thus, you would define a function in C style in the following way.
The following example is for small model:

\begin{lstlisting}
global  _myfunc

_myfunc:
    push    bp
    mov     bp,sp
    sub     sp,0x40         ; 64 bytes of local stack space
    mov     bx,[bp+4]       ; first parameter to function

    ; some more code

    mov     sp,bp           ; undo "sub sp,0x40" above
    pop     bp
    ret
\end{lstlisting}

For a large-model function, you would replace \code{RET} by \code{RETF},
and look for the first parameter at \code{[BP+6]} instead of
\code{[BP+4]}. Of course, if one of the parameters is a pointer, then
the offsets of \emph{subsequent} parameters will change depending on
the memory model as well: far pointers take up four bytes on the
stack when passed as a parameter, whereas near pointers take up two.

At the other end of the process, to call a C function from your
assembly code, you would do something like this:

\begin{lstlisting}
extern  _printf
    ; and then, further down...

    push    word [myint]        ; one of my integer variables
    push    word mystring       ; pointer into my data segment
    call    _printf
    add     sp,byte 4           ; `byte' saves space

    ; then those data items...
segment _DATA

myint       dw  1234
mystring    db  'This number -> %d <- should be 1234',10,0
\end{lstlisting}

This piece of code is the small-model assembly equivalent of the C
code

\begin{lstlisting}
    int myint = 1234;
    printf("This number -> %d <- should be 1234\n", myint);
\end{lstlisting}

In large model, the function-call code might look more like this. In
this example, it is assumed that \code{DS} already holds the segment
base of the segment \code{\_DATA}. If not, you would have to initialize
it first.

\begin{lstlisting}
    push    word [myint]
    push    word seg mystring   ; Now push the segment, and...
    push    word mystring       ; ... offset of "mystring"
    call    far _printf
    add    sp,byte 6
\end{lstlisting}

The integer value still takes up one word on the stack, since large
model does not affect the size of the \code{int} data type. The first
argument (pushed last) to \code{printf}, however, is a data pointer,
and therefore has to contain a segment and offset part. The segment
should be stored second in memory, and therefore must be pushed
first. (Of course, \code{PUSH DS} would have been a shorter instruction
than \code{PUSH WORD SEG mystring}, if \code{DS} was set up as the above
example assumed.) Then the actual call becomes a far call, since
functions expect far calls in large model; and \code{SP} has to be
increased by 6 rather than 4 afterwards to make up for the extra
word of parameters.

\xsubsection{16cdata}{Accessing Data Items}

To get at the contents of C variables, or to declare variables which
C can access, you need only declare the names as \code{GLOBAL} or
\code{EXTERN}. (Again, the names require leading underscores, as stated
in \fullref{subsec:16cunder}.) Thus, a C variable declared as \code{int i}
can be accessed from assembler as

\begin{lstlisting}
extern _i

    mov ax,[_i]
\end{lstlisting}

And to declare your own integer variable which C programs can access
as \code{extern int j}, you do this (making sure you are assembling in
the \code{\_DATA} segment, if necessary):

\begin{lstlisting}
global  _j

_j      dw  0
\end{lstlisting}

To access a C array, you need to know the size of the components of
the array. For example, \code{int} variables are two bytes long, so if
a C program declares an array as \code{int a[10]}, you can access
\code{a[3]} by coding \code{mov ax,[\_a+6]}. (The byte offset 6 is obtained
by multiplying the desired array index, 3, by the size of the array
element, 2.) The sizes of the C base types in 16-bit compilers are:
1 for \code{char}, 2 for \code{short} and \code{int}, 4 for \code{long}
and \code{float}, and 8 for \code{double}.

To access a C \textindex{data structure}, you need to know the offset from
the base of the structure to the field you are interested in. You
can either do this by converting the C structure definition into a
NASM structure definition (using \codeindex{STRUC}), or by calculating the
one offset and using just that.

To do either of these, you should read your C compiler's manual to
find out how it organizes data structures. NASM gives no special
alignment to structure members in its own \code{STRUC} macro, so you
have to specify alignment yourself if the C compiler generates it.
Typically, you might find that a structure like

\begin{lstlisting}
struct {
    char c;
    int i;
} foo;
\end{lstlisting}

might be four bytes long rather than three, since the \code{int} field
would be aligned to a two-byte boundary. However, this sort of
feature tends to be a configurable option in the C compiler, either
using command-line options or \code{\#pragma} lines, so you have to find
out how your own compiler does it.

\subsection{\codeindex{c16.mac}: Helper Macros for the 16-bit C Interface}
\label{subsec:16cmacro}

Included in the NASM archives, in the \index{misc subdirectory}\code{misc}
directory, is a file \code{c16.mac} of macros. It defines three macros:
\codeindex{proc}, \codeindex{arg} and \codeindex{endproc}. These are intended
to be used for C-style procedure definitions, and they automate a lot of
the work involved in keeping track of the calling convention.

(An alternative, TASM compatible form of \code{arg} is also now built
into NASM's preprocessor. See \fullref{sec:stackrel} for details.)

An example of an assembly function using the macro set is given
here:

\begin{lstlisting}
proc    _nearproc
%$i     arg
%$j     arg
        mov     ax,[bp + %$i]
        mov     bx,[bp + %$j]
        add     ax,[bx]
endproc
\end{lstlisting}

This defines \code{\_nearproc} to be a procedure taking two arguments,
the first (\code{i}) an integer and the second (\code{j}) a pointer to an
integer. It returns \code{i + *j}.

Note that the \code{arg} macro has an \code{EQU} as the first line of its
expansion, and since the label before the macro call gets prepended
to the first line of the expanded macro, the \code{EQU} works, defining
\code{\%\$i} to be an offset from \code{BP}. A context-local variable is
used, local to the context pushed by the \code{proc} macro and popped
by the \code{endproc} macro, so that the same argument name can be used
in later procedures. Of course, you don't \emph{have} to do that.

The macro set produces code for near functions (tiny, small and
compact-model code) by default. You can have it generate far
functions (medium, large and huge-model code) by means of coding
\indexcode{FARCODE}\code{\%define FARCODE}. This changes the kind of
return instruction generated by \code{endproc}, and also changes the
starting point for the argument offsets. The macro set contains no
intrinsic dependency on whether data pointers are far or not.

\code{arg} can take an optional parameter, giving the size of the
argument. If no size is given, 2 is assumed, since it is likely that
many function parameters will be of type \code{int}.

The large-model equivalent of the above function would look like this:

\begin{lstlisting}
%define FARCODE

proc    _farproc
%$i     arg
%$j     arg     4
        mov     ax,[bp + %$i]
        mov     bx,[bp + %$j]
        mov     es,[bp + %$j + 2]
        add     ax,[bx]
endproc
\end{lstlisting}

This makes use of the argument to the \code{arg} macro to define a
parameter of size 4, because \code{j} is now a far pointer. When we
load from \code{j}, we must load a segment and an offset.
